%% Generated by Sphinx.
\def\sphinxdocclass{report}
\documentclass[letterpaper,10pt,english]{sphinxmanual}
\ifdefined\pdfpxdimen
   \let\sphinxpxdimen\pdfpxdimen\else\newdimen\sphinxpxdimen
\fi \sphinxpxdimen=.75bp\relax
\ifdefined\pdfimageresolution
    \pdfimageresolution= \numexpr \dimexpr1in\relax/\sphinxpxdimen\relax
\fi
%% let collapsible pdf bookmarks panel have high depth per default
\PassOptionsToPackage{bookmarksdepth=5}{hyperref}

\PassOptionsToPackage{booktabs}{sphinx}
\PassOptionsToPackage{colorrows}{sphinx}

\PassOptionsToPackage{warn}{textcomp}
\usepackage[utf8]{inputenc}
\ifdefined\DeclareUnicodeCharacter
% support both utf8 and utf8x syntaxes
  \ifdefined\DeclareUnicodeCharacterAsOptional
    \def\sphinxDUC#1{\DeclareUnicodeCharacter{"#1}}
  \else
    \let\sphinxDUC\DeclareUnicodeCharacter
  \fi
  \sphinxDUC{00A0}{\nobreakspace}
  \sphinxDUC{2500}{\sphinxunichar{2500}}
  \sphinxDUC{2502}{\sphinxunichar{2502}}
  \sphinxDUC{2514}{\sphinxunichar{2514}}
  \sphinxDUC{251C}{\sphinxunichar{251C}}
  \sphinxDUC{2572}{\textbackslash}
\fi
\usepackage{cmap}
\usepackage[T1]{fontenc}
\usepackage{amsmath,amssymb,amstext}
\usepackage{babel}



\usepackage{tgtermes}
\usepackage{tgheros}
\renewcommand{\ttdefault}{txtt}



\usepackage[Bjarne]{fncychap}
\usepackage[,numfigreset=1,mathnumfig]{sphinx}

\fvset{fontsize=auto}
\usepackage{geometry}

\usepackage{nbsphinx}

% Include hyperref last.
\usepackage{hyperref}
% Fix anchor placement for figures with captions.
\usepackage{hypcap}% it must be loaded after hyperref.
% Set up styles of URL: it should be placed after hyperref.
\urlstyle{same}


\usepackage{sphinxmessages}
\setcounter{tocdepth}{1}


    \usepackage{amsmath}
    \usepackage{amsfonts}
    \usepackage{amssymb}
    \usepackage{bm}
%    \usepackage{bbm}
    \usepackage{booktabs}
%    \usepackage[table,xcdraw]{xcolor}
    \usepackage{rotating,tabularx}
    \usepackage{multirow}
  

\title{HepEmShow a compact EM shower simulation}
\date{Nov 07, 2023}
\release{0.0.0}
\author{Mihaly Novak}
\newcommand{\sphinxlogo}{\sphinxincludegraphics{logo_HepEmShow.png}\par}
\renewcommand{\releasename}{Release}
\makeindex
\begin{document}

\ifdefined\shorthandoff
  \ifnum\catcode`\=\string=\active\shorthandoff{=}\fi
  \ifnum\catcode`\"=\active\shorthandoff{"}\fi
\fi

\pagestyle{empty}

     \pagenumbering{Roman} %%% to avoid page 1 conflict with actual page 1
     \begin{titlepage}
       %% * give space from top
       \vspace*{30mm}
       \textbf{\Huge {\texttt{HepEmShow} a compact EM shower simulation}}
       \rule{1.0\linewidth}{2.4pt}\\[-3.7ex] \rule{1.0\linewidth}{0.6pt}
       %% add logo
          \vspace*{20mm}
          \begin{figure}[!h]
             \centering
             \includegraphics[scale=1.1]{logo_HepEmShow.png}
          \end{figure}
       %% add some space
       %% add space till the bottom
       \vfill
       \vspace*{-50mm}
       \centering
       \Large \textbf{Mih{\'a}ly Nov{\'a}k}\\ CERN EP-SFT\\
       \vspace*{30mm}
       \small \textbf{\today}
    \end{titlepage}
    \pagenumbering{arabic}

\pagestyle{plain}
\sphinxtableofcontents
\pagestyle{normal}
\phantomsection\label{\detokenize{index::doc}}


\sphinxstepscope


\chapter{Introduction}
\label{\detokenize{IntroAndInstall/introduction:introduction}}\label{\detokenize{IntroAndInstall/introduction:introduction-doc}}\label{\detokenize{IntroAndInstall/introduction::doc}}
\sphinxAtStartPar
\sphinxcode{\sphinxupquote{HepEmShow}} is an application for simulating electromagnetic (EM) shower in a configurable simplified sampling calorimeter.
While the EM shower is simulated by the same algorithm and physics used today for detector simulation in High Energy Physics (HEP),
including the Large Hadron Collider (LHC) experiments such as ATLAS or CMS, the entire simulation is kept very lightweight and simple
in order to enhance the clarity and transparency of the underlying computing flow and algorithms.

\sphinxAtStartPar
The \sphinxcode{\sphinxupquote{Geant4}} \sphinxcite{zzbib:agostinelli2003geant4}\sphinxcite{zzbib:allison2006geant4}\sphinxcite{zzbib:allison2016recent} simulation toolkit is the main workhorse for such simulations
today due to the flexibility provided by its carefully design interfaces, powerful geometry description, navigation and comprehensive physics
modelling capabilities. While from the applications point of view these properties are advantageous (e.g. making possible to handle even the
most complex simulation problems some of which listed above), they might cause considerable difficulties when the goal is to identify and
investigate the underlying computing flow and algorithms. This is due to the simple fact that a relatively large fraction of the \sphinxcode{\sphinxupquote{Geant4}}
codebase is involved even in a simple simulation application due to the generic internal framework of the toolkit. It already requires substantial
effort and time to become familiar enough with the toolkit for application development but even more effort and significant expertise are need
when the goal is to identify and understand the underlying computing flow and algorithms. This can cause difficulties or even pose barriers
especially when non\sphinxhyphen{}simulation experts would like to take some \sphinxcode{\sphinxupquote{Geant4}} based simulation as the application target of their own technology,
new techniques, solution, etc.. Moreover, applying the new technology to the complete, final simulation target usually has significant cost and
risk (see above) that can be largely reduced by preceding with a feasibility study on a simpler, more compact but realistic simulation, i.e.
having the key algorithmic properties identical to the native \sphinxcode{\sphinxupquote{Geant4}} based simulation.

\sphinxAtStartPar
\sphinxcode{\sphinxupquote{HepEmShow}} was developed exactly with this goal in mind: to provide a compact, \sphinxcode{\sphinxupquote{Geant4}} like but significantly simpler simulation application
with only the required components but all available locally. More details on the main components of the simulation application can be found in
\sphinxtitleref{The components of the simulation} section.

\sphinxstepscope


\chapter{Build and Install}
\label{\detokenize{IntroAndInstall/install:build-and-install}}\label{\detokenize{IntroAndInstall/install:install-doc}}\label{\detokenize{IntroAndInstall/install::doc}}
\sphinxAtStartPar
Will come later but \sphinxcode{\sphinxupquote{HepEmShow}} requires \sphinxcode{\sphinxupquote{G4HepEm}} that provides its Physics (see more at the {\hyperref[\detokenize{IntroAndInstall/components:simulation-components-doc}]{\sphinxcrossref{\DUrole{std,std-ref}{The components of the simulation}}}}).
\sphinxcode{\sphinxupquote{G4HepEm}} can be built with or without \sphinxcode{\sphinxupquote{Geant4}} though a data file is needed
in the latter case. See more later …


\section{Build \sphinxstyleliteralintitle{\sphinxupquote{G4HepEm}} with or without \sphinxstyleliteralintitle{\sphinxupquote{Geant4}} dependence:}
\label{\detokenize{IntroAndInstall/install:build-g4hepem-with-or-without-geant4-dependence}}

\subsection{With \sphinxstyleliteralintitle{\sphinxupquote{Geant4}} (\sphinxstyleliteralintitle{\sphinxupquote{G4HepEm\_GEANT4\_BUILD=ON}} default):}
\label{\detokenize{IntroAndInstall/install:with-geant4-g4hepem-geant4-build-on-default}}
\begin{sphinxVerbatim}[commandchars=\\\{\}]
\PYG{n}{cmake} \PYG{o}{.}\PYG{o}{.}\PYG{o}{/} \PYG{o}{\PYGZhy{}}\PYG{n}{DGeant4\PYGZus{}DIR}\PYG{o}{=}\PYG{n}{YOUR\PYGZus{}GEANT4\PYGZus{}INSTALL}\PYG{o}{/}\PYG{n}{lib}\PYG{p}{(}\PYG{n}{lib64}\PYG{p}{)}\PYG{o}{/}\PYG{n}{cmake}\PYG{o}{/}\PYG{n}{Geant4}\PYG{o}{/} \PYG{o}{\PYGZhy{}}\PYG{n}{DCMAKE\PYGZus{}INSTALL\PYGZus{}PREFIX}\PYG{o}{=}\PYG{n}{YOUR\PYGZus{}G4HEPEM\PYGZus{}INSTALL}
\end{sphinxVerbatim}


\subsection{Without \sphinxstyleliteralintitle{\sphinxupquote{Geant4}} (\sphinxstyleliteralintitle{\sphinxupquote{G4HepEm\_GEANT4\_BUILD=OFF}}):}
\label{\detokenize{IntroAndInstall/install:without-geant4-g4hepem-geant4-build-off}}
\begin{sphinxVerbatim}[commandchars=\\\{\}]
\PYG{n}{cmake} \PYG{o}{.}\PYG{o}{.}\PYG{o}{/} \PYG{o}{\PYGZhy{}}\PYG{n}{DG4HepEm\PYGZus{}GEANT4\PYGZus{}BUILD}\PYG{o}{=}\PYG{n}{OFF} \PYG{o}{\PYGZhy{}}\PYG{n}{DCMAKE\PYGZus{}INSTALL\PYGZus{}PREFIX}\PYG{o}{=}\PYG{n}{YOUR\PYGZus{}G4HEPEM\PYGZus{}INSTALL}
\end{sphinxVerbatim}


\section{Build \sphinxstyleliteralintitle{\sphinxupquote{HepEmShow}} with \sphinxstyleliteralintitle{\sphinxupquote{G4HepEm}} build with or without \sphinxstyleliteralintitle{\sphinxupquote{Geant4}}:}
\label{\detokenize{IntroAndInstall/install:build-hepemshow-with-g4hepem-build-with-or-without-geant4}}

\subsection{With a complete, \sphinxstyleliteralintitle{\sphinxupquote{Geant4}} dependent \sphinxstyleliteralintitle{\sphinxupquote{G4HepEm}} build:}
\label{\detokenize{IntroAndInstall/install:with-a-complete-geant4-dependent-g4hepem-build}}
\begin{sphinxVerbatim}[commandchars=\\\{\}]
\PYG{n}{cmake} \PYG{o}{.}\PYG{o}{.}\PYG{o}{/} \PYG{o}{\PYGZhy{}}\PYG{n}{DGeant4\PYGZus{}DIR}\PYG{o}{=}\PYG{n}{YOUR\PYGZus{}GEANT4\PYGZus{}INSTALL}\PYG{o}{/}\PYG{n}{lib}\PYG{p}{(}\PYG{n}{lib64}\PYG{p}{)}\PYG{o}{/}\PYG{n}{cmake}\PYG{o}{/}\PYG{n}{Geant4}\PYG{o}{/} \PYG{o}{\PYGZhy{}}\PYG{n}{DG4HepEm\PYGZus{}DIR}\PYG{o}{=}\PYG{n}{YOUR\PYGZus{}G4HEPEM\PYGZus{}INSTALL}\PYG{o}{/}\PYG{n}{lib}\PYG{p}{(}\PYG{n}{lib64}\PYG{p}{)}\PYG{o}{/}\PYG{n}{cmake}\PYG{o}{/}\PYG{n}{G4HepEm}\PYG{o}{/} \PYG{o}{\PYGZhy{}}\PYG{n}{DCMAKE\PYGZus{}BUILD\PYGZus{}TYPE}\PYG{o}{=}\PYG{n}{RELEASE}
\end{sphinxVerbatim}


\subsection{With a standalone, \sphinxstyleliteralintitle{\sphinxupquote{Geant4}} independent \sphinxstyleliteralintitle{\sphinxupquote{G4HepEm}} build:}
\label{\detokenize{IntroAndInstall/install:with-a-standalone-geant4-independent-g4hepem-build}}
\begin{sphinxVerbatim}[commandchars=\\\{\}]
\PYG{n}{cmake} \PYG{o}{.}\PYG{o}{.}\PYG{o}{/} \PYG{o}{\PYGZhy{}}\PYG{n}{DG4HepEm\PYGZus{}DIR}\PYG{o}{=}\PYG{n}{YOUR\PYGZus{}G4HEPEM\PYGZus{}INSTALL}\PYG{o}{/}\PYG{n}{lib}\PYG{p}{(}\PYG{n}{lib64}\PYG{p}{)}\PYG{o}{/}\PYG{n}{cmake}\PYG{o}{/}\PYG{n}{G4HepEm}\PYG{o}{/} \PYG{o}{\PYGZhy{}}\PYG{n}{DCMAKE\PYGZus{}BUILD\PYGZus{}TYPE}\PYG{o}{=}\PYG{n}{RELEASE}
\end{sphinxVerbatim}


\section{Requirements}
\label{\detokenize{IntroAndInstall/install:requirements}}\label{\detokenize{IntroAndInstall/install:install-requirements}}

\section{Build and install}
\label{\detokenize{IntroAndInstall/install:id1}}
\sphinxstepscope


\chapter{The components of the simulation}
\label{\detokenize{IntroAndInstall/components:the-components-of-the-simulation}}\label{\detokenize{IntroAndInstall/components:simulation-components-doc}}\label{\detokenize{IntroAndInstall/components::doc}}
\sphinxAtStartPar
The two main components of the simulation is the {\hyperref[\detokenize{IntroAndInstall/components:geometry}]{\sphinxcrossref{Geometry}}} and {\hyperref[\detokenize{IntroAndInstall/components:physics}]{\sphinxcrossref{Physics}}} providing the necessary information and functionalities fused together
to compute the individual simulation steps during the {\hyperref[\detokenize{IntroAndInstall/components:event-processing}]{\sphinxcrossref{Event processing}}} inside the {\hyperref[\detokenize{IntroAndInstall/components:stepping-loop}]{\sphinxcrossref{Stepping loop}}}, i.e. how far the particle goes
in a given simulation step and what happens with it at that post\sphinxhyphen{}step point.

\sphinxAtStartPar
The {\hyperref[\detokenize{IntroAndInstall/components:geometry}]{\sphinxcrossref{Geometry}}} describes the simulation setup, including navigation possibilities or providing information on the \sphinxstylestrong{geometrical constraints for
the simulation step computation} (e.g. given a global point, how far the nearest volume boundary is or how far the volume boundary to a given
direction is, etc.). The {\hyperref[\detokenize{IntroAndInstall/components:physics}]{\sphinxcrossref{Physics}}} component is responsible to supply the \sphinxstylestrong{physics related constraints for the simulation step computation}
(e.g. how far the given particle go till its next interaction of the given type or how far a charge particle goes till it loses all its current
energy, etc.). These constraints are required during the {\hyperref[\detokenize{IntroAndInstall/components:event-processing}]{\sphinxcrossref{Event processing}}} at the beginning of each simulation step computation in the
{\hyperref[\detokenize{IntroAndInstall/components:stepping-loop}]{\sphinxcrossref{Stepping loop}}} to decide length of the simulation step. The particle is then moved to the post\sphinxhyphen{}step point and the necessary {\hyperref[\detokenize{IntroAndInstall/components:geometry}]{\sphinxcrossref{Geometry}}}
and/or {\hyperref[\detokenize{IntroAndInstall/components:physics}]{\sphinxcrossref{Physics}}} related \sphinxstylestrong{actions are performed} on the particle. Some information regarding the given simulation step, e.g. energy deposit,
might be collected at the end of each simulation step before performing the computation of the next step till the history of the particle is
terminated (e.g. \(e^-\) lost all its kinetic energy in the last step; \(\gamma\) was absorbed in photoelectric process or destructed by
producing \(e^-/e^+\) pair, etc.).

\sphinxAtStartPar
A short description of the main components of the \sphinxcode{\sphinxupquote{HepEmShow}} simulation application, such as the {\hyperref[\detokenize{IntroAndInstall/components:geometry}]{\sphinxcrossref{Geometry}}}, {\hyperref[\detokenize{IntroAndInstall/components:physics}]{\sphinxcrossref{Physics}}}, {\hyperref[\detokenize{IntroAndInstall/components:event-processing}]{\sphinxcrossref{Event processing}}} and
{\hyperref[\detokenize{IntroAndInstall/components:stepping-loop}]{\sphinxcrossref{Stepping loop}}}, is provided in this section together with {\hyperref[\detokenize{IntroAndInstall/components:the-hepemshow-application-main}]{\sphinxcrossref{The HepEmShow application main}}} and some example configuration.


\section{Geometry}
\label{\detokenize{IntroAndInstall/components:geometry}}\label{\detokenize{IntroAndInstall/components:geometry-doc}}
\sphinxAtStartPar
The application {\hyperref[\detokenize{Simulation/SimulationCodeDoc:_CPPv48Geometry}]{\sphinxcrossref{\sphinxcode{\sphinxupquote{Geometry}}}}} is a configurable simplified sampling \sphinxcode{\sphinxupquote{calorimeter}} that is built up from \sphinxcode{\sphinxupquote{N}} \sphinxcode{\sphinxupquote{layer}}\sphinxhyphen{}s of an \sphinxcode{\sphinxupquote{absorber}} and
a \sphinxcode{\sphinxupquote{gap}} as illustrated in \hyperref[\detokenize{IntroAndInstall/components:fig-calo-layer2}]{Fig.\@ \ref{\detokenize{IntroAndInstall/components:fig-calo-layer2}}}.

\begin{figure}[htbp]
\centering
\capstart

\noindent\sphinxincludegraphics[width=604.80000\sphinxpxdimen,height=318.40000\sphinxpxdimen]{{calo_layer2}.png}
\caption{Illustration of the default \sphinxcode{\sphinxupquote{absorber}} (green) and \sphinxcode{\sphinxupquote{gap}} (blue) \sphinxcode{\sphinxupquote{layer}} structured simplified sampling calorimeter configuration (with \sphinxcode{\sphinxupquote{N = 14}} \sphinxcode{\sphinxupquote{layer}}\sphinxhyphen{}s in this case).
The red arrow from left represents the direction of the incoming primary particle beam.}\label{\detokenize{IntroAndInstall/components:geom-calo-layer2}}\label{\detokenize{IntroAndInstall/components:fig-calo-layer2}}\end{figure}

\sphinxAtStartPar
The number of \sphinxcode{\sphinxupquote{layer}}\sphinxhyphen{}s \sphinxcode{\sphinxupquote{N}}, the thickness of both the \sphinxcode{\sphinxupquote{absorber}} and \sphinxcode{\sphinxupquote{gap}} can be set at the construction of the calorimeter
(see the {\hyperref[\detokenize{IntroAndInstall/components:input-arguments}]{\sphinxcrossref{Input arguments}}} of {\hyperref[\detokenize{IntroAndInstall/components:the-hepemshow-application-main}]{\sphinxcrossref{The HepEmShow application main}}}). All thicknesses are measured along the \sphinxcode{\sphinxupquote{x}}\sphinxhyphen{}axis in \sphinxcode{\sphinxupquote{mm}} units.

\begin{sphinxadmonition}{note}{Note:}
\sphinxAtStartPar
The \sphinxcode{\sphinxupquote{gap}} thickness can be set even to zero in which case the \sphinxcode{\sphinxupquote{calorimeter}} is built up from the given number of \sphinxcode{\sphinxupquote{layer}}\sphinxhyphen{}s of
\sphinxcode{\sphinxupquote{absorber}} with the given thickness (i.e. a single material calorimeter sliced by the \sphinxcode{\sphinxupquote{layer}}\sphinxhyphen{}s) as illustrated in \hyperref[\detokenize{IntroAndInstall/components:fig-calo-layer1}]{Fig.\@ \ref{\detokenize{IntroAndInstall/components:fig-calo-layer1}}}.

\begin{figure}[H]
\centering
\capstart

\noindent\sphinxincludegraphics[width=589\sphinxpxdimen,height=325\sphinxpxdimen]{{calo_layer1}.png}
\caption{Illustration of the single material calorimeter sliced by the \sphinxcode{\sphinxupquote{layer}}\sphinxhyphen{}s (\sphinxcode{\sphinxupquote{N = 18}}). As the \sphinxcode{\sphinxupquote{gap}} thickness is set to \sphinxcode{\sphinxupquote{0}}, the
single \sphinxcode{\sphinxupquote{layer}} thickness is identical to the \sphinxcode{\sphinxupquote{absorber}} (blue) thickness while the \sphinxcode{\sphinxupquote{calorimeter}} thickness is \(\times\) \sphinxcode{\sphinxupquote{N}} of that.}\label{\detokenize{IntroAndInstall/components:geom-calo-layer1}}\label{\detokenize{IntroAndInstall/components:fig-calo-layer1}}\end{figure}
\end{sphinxadmonition}

\sphinxAtStartPar
The thickness of the single \sphinxcode{\sphinxupquote{layer}} as well as the entire \sphinxcode{\sphinxupquote{calorimeter}} is automatically computed from the given \sphinxcode{\sphinxupquote{absorber}}, \sphinxcode{\sphinxupquote{gap}}
thicknesses and number of \sphinxcode{\sphinxupquote{layer}}\sphinxhyphen{}s respectively. The transverse size, i.e. the full extent of the \sphinxcode{\sphinxupquote{absorber}}, \sphinxcode{\sphinxupquote{gap}}, \sphinxcode{\sphinxupquote{layer}}
and the \sphinxcode{\sphinxupquote{calorimeter}} along the \sphinxcode{\sphinxupquote{y}}\sphinxhyphen{} and \sphinxcode{\sphinxupquote{z}}\sphinxhyphen{}axes, can also be set as an input argument (see the {\hyperref[\detokenize{IntroAndInstall/components:input-arguments}]{\sphinxcrossref{Input arguments}}} of
{\hyperref[\detokenize{IntroAndInstall/components:the-hepemshow-application-main}]{\sphinxcrossref{The HepEmShow application main}}}). The calorimeter is places in the center of the \sphinxcode{\sphinxupquote{world}} so 5 ({\hyperref[\detokenize{Simulation/SimulationCodeDoc:_CPPv43Box}]{\sphinxcrossref{\sphinxcode{\sphinxupquote{Box}}}}} shaped) volumes are
used in total to describe the application geometry. The size of the \sphinxcode{\sphinxupquote{world}} volume is computed automatically based on the extent of the
calorimeter such that it encloses the entire calorimeter with some margin.

\begin{sphinxadmonition}{note}{Note:}
\sphinxAtStartPar
The \sphinxcode{\sphinxupquote{x}}\sphinxhyphen{}coordinate of the left hand side boundary of the calorimeter as well as an appropriate
initial \sphinxcode{\sphinxupquote{x}}\sphinxhyphen{}coordinate of the primary particles (such that they are located mid\sphinxhyphen{}way between the \sphinxcode{\sphinxupquote{calorimeter}} and
the \sphinxcode{\sphinxupquote{world}} boundaries along the negative \sphinxcode{\sphinxupquote{x}}\sphinxhyphen{}axis) are calculated. At the beginning of an event, the
{\hyperref[\detokenize{Simulation/SimulationCodeDoc:_CPPv416PrimaryGenerator}]{\sphinxcrossref{\sphinxcode{\sphinxupquote{PrimaryGenerator}}}}} will generate primary particles at this later position with the \sphinxcode{\sphinxupquote{{[}1,0,0{]}}} direction vector,
i.e. pointing toward to the left hand side boundary of the calorimeter. Each primary particle is then moved to the former position,
i.e. on the left hand side boundary of the calorimeter, in their very first step such that they will enter into the calorimeter
in the next, second simulation step.
\end{sphinxadmonition}

\sphinxAtStartPar
Each of these volumes is filled with a given material specified by the material index of the volume. These indices are the subscripts of the
(\sphinxcode{\sphinxupquote{Geant4}} predefined NIST) materials listed in the material name vector of {\hyperref[\detokenize{IntroAndInstall/components:the-hepemshow-datageneration-application-main}]{\sphinxcrossref{The HepEmShow\sphinxhyphen{}DataGeneration application main}}}. In the default case, which is the same that
was used to generate the \sphinxcode{\sphinxupquote{G4HepEm}} data shipped with the application, this material name vector is
\sphinxSetupCaptionForVerbatim{The material name vector as it is in {\hyperref[\detokenize{IntroAndInstall/components:the-hepemshow-datageneration-application-main}]{\sphinxcrossref{The HepEmShow\sphinxhyphen{}DataGeneration application main}}}.}
\def\sphinxLiteralBlockLabel{\label{\detokenize{IntroAndInstall/components:id3}}}
\begin{sphinxVerbatim}[commandchars=\\\{\}]
\PYG{c+c1}{// list of Geant4 (NIST) material names}
\PYG{n}{std}\PYG{o}{:}\PYG{o}{:}\PYG{n}{vector}\PYG{o}{\PYGZlt{}}\PYG{n}{std}\PYG{o}{:}\PYG{o}{:}\PYG{n}{string}\PYG{o}{\PYGZgt{}}\PYG{+w}{ }\PYG{n}{matList}\PYG{+w}{ }\PYG{p}{\PYGZob{}}\PYG{l+s}{\PYGZdq{}}\PYG{l+s}{G4\PYGZus{}Galactic}\PYG{l+s}{\PYGZdq{}}\PYG{p}{,}\PYG{+w}{ }\PYG{l+s}{\PYGZdq{}}\PYG{l+s}{G4\PYGZus{}PbWO4}\PYG{l+s}{\PYGZdq{}}\PYG{p}{,}\PYG{+w}{ }\PYG{l+s}{\PYGZdq{}}\PYG{l+s}{G4\PYGZus{}lAr}\PYG{l+s}{\PYGZdq{}}\PYG{p}{\PYGZcb{}}\PYG{p}{;}
\end{sphinxVerbatim}

\sphinxAtStartPar
This corresponds to the default \sphinxtitleref{index\sphinxhyphen{}to\sphinxhyphen{}material} and eventually to the \sphinxtitleref{material\sphinxhyphen{}to\sphinxhyphen{}volume} association shown
in \hyperref[\detokenize{IntroAndInstall/components:table-material-index}]{Table \ref{\detokenize{IntroAndInstall/components:table-material-index}}}. A complete list of the predefined NIST materials provided by \sphinxcode{\sphinxupquote{Geant4}} with their composition
can be found at the corresponding part of the \sphinxcode{\sphinxupquote{Geant4}} \sphinxhref{https://geant4-userdoc.web.cern.ch/UsersGuides/ForApplicationDeveloper/html/Appendix/materialNames.html}{documentation (Book For Application Developers: Geant4 Material Database)}.


\begin{savenotes}\sphinxattablestart
\sphinxthistablewithglobalstyle
\centering
\sphinxcapstartof{table}
\sphinxthecaptionisattop
\sphinxcaption{The default material to index and material to volume association.}\label{\detokenize{IntroAndInstall/components:table-material-index}}
\sphinxaftertopcaption
\begin{tabulary}{\linewidth}[t]{TTT}
\sphinxtoprule
\sphinxstyletheadfamily 
\sphinxAtStartPar
Material
&\sphinxstyletheadfamily 
\sphinxAtStartPar
Index
&\sphinxstyletheadfamily 
\sphinxAtStartPar
Used in Volume
\\
\sphinxmidrule
\sphinxtableatstartofbodyhook
\sphinxAtStartPar
\sphinxtitleref{galactic}
(low density gas)
&
\sphinxAtStartPar
0
&
\sphinxAtStartPar
\sphinxcode{\sphinxupquote{layer}}
\sphinxcode{\sphinxupquote{calorimeter}}
\sphinxcode{\sphinxupquote{world}}
\\
\sphinxhline
\sphinxAtStartPar
lead tungstate
(atolzite)
&
\sphinxAtStartPar
1
&
\sphinxAtStartPar
\sphinxcode{\sphinxupquote{absorber}}
\\
\sphinxhline
\sphinxAtStartPar
liquid\sphinxhyphen{}argon
&
\sphinxAtStartPar
2
&
\sphinxAtStartPar
\sphinxcode{\sphinxupquote{gap}}
\\
\sphinxbottomrule
\end{tabulary}
\sphinxtableafterendhook\par
\sphinxattableend\end{savenotes}

\begin{sphinxadmonition}{note}{Note:}
\sphinxAtStartPar
Changing the material name(s) in this above vector of the {\hyperref[\detokenize{IntroAndInstall/components:the-hepemshow-datageneration-application-main}]{\sphinxcrossref{The HepEmShow\sphinxhyphen{}DataGeneration application main}}} (especially at index \sphinxcode{\sphinxupquote{1}}
and/or \sphinxcode{\sphinxupquote{2}} as the vacuum is always needed to fill the \sphinxcode{\sphinxupquote{layer}}, \sphinxcode{\sphinxupquote{calorimeter}} and \sphinxcode{\sphinxupquote{world}} container volumes), regenerating the data
by executing this data generation application, then executing again the \sphinxcode{\sphinxupquote{HepEmShow}} application, corresponds to changing the material
of the \sphinxcode{\sphinxupquote{absorber}} and/or \sphinxcode{\sphinxupquote{gap}} volumes of the simulation.
\end{sphinxadmonition}

\sphinxAtStartPar
The application geometry also provides a rather simple “navigation” capability (used in the simulation stepping loops) through its {\hyperref[\detokenize{Simulation/SimulationCodeDoc:_CPPv4N8Geometry22CalculateDistanceToOutEPdPdPP3BoxPiPi}]{\sphinxcrossref{\sphinxcode{\sphinxupquote{Geometry::CalculateDistanceToOut()}}}}} method
described in details at the corresponding code documentation.

\begin{sphinxadmonition}{attention}{Attention:}
\sphinxAtStartPar
Unlike the \sphinxcode{\sphinxupquote{Geant4}} geometry modeller and navigation, that provides generic geometry description and navigation capabilities,
the {\hyperref[\detokenize{Simulation/SimulationCodeDoc:_CPPv48Geometry}]{\sphinxcrossref{\sphinxcode{\sphinxupquote{Geometry}}}}} implemented for \sphinxcode{\sphinxupquote{HepEmShow}} is specific to the configurable simplified sampling calorimeter described above. Focusing
only to this specific geometry modelling and related navigation problem made possible to provide a rather compact, simple and clear implementation
of all geometry related functionalities required during the simulation (i.e. in the {\hyperref[\detokenize{Simulation/SimulationCodeDoc:_CPPv412SteppingLoop}]{\sphinxcrossref{\sphinxcode{\sphinxupquote{SteppingLoop}}}}}).
\end{sphinxadmonition}


\section{Physics}
\label{\detokenize{IntroAndInstall/components:physics}}
\sphinxAtStartPar
Targeting only the simulation of the EM shower inherently leads to a compact simulation as it includes only \(e^-/e^+\) and \(\gamma\)
particles with their EM (i.e. without gamma\sphinxhyphen{} and lepto\sphinxhyphen{}nuclear) interactions. Focusing to the descriptions of these interactions, that ensures
sufficient details and accuracy for HEP detector simulations, leads to an even more specific set of interactions and underlying models that
the physics component of the simulation needs to provide. This well defined, important but small subset of the very rich physics offered by
the \sphinxcode{\sphinxupquote{Geant4}} toolkit, can then be implemented in a very compact form.

\sphinxAtStartPar
The \sphinxcode{\sphinxupquote{G4HepEm}} R\&D project \sphinxcite{zzbib:g4hepem} offers such an implementation with several attractive properties. Separation of data definition,
initialisation and run\sphinxhyphen{}time functionalities results in a rather small, \sphinxcode{\sphinxupquote{Geant4}} independent, stateless, header based implementation of all
physics related run\sphinxhyphen{}time functionalities required for such EM shower simulations. Furthermore, all the data, extracted from \sphinxcode{\sphinxupquote{Geant4}} during
the initialisation, can be exported/imported into/from a single file making possible to skip the \sphinxcode{\sphinxupquote{Geant4}} dependent initialisation phase
in subsequent executions of the application. Therefore, \sphinxcode{\sphinxupquote{G4HepEm}} offers the possibility of a \sphinxcode{\sphinxupquote{Geant4}} like but \sphinxcode{\sphinxupquote{Geant4}}
independent EM physics component for developing particle transport simulations. Further information on \sphinxcode{\sphinxupquote{G4HepEm}}, including the
\sphinxhref{https://g4hepem.readthedocs.io/en/latest/IntroAndInstall/introduction.html\#physics-modelling-capability}{physics interactions included},
can be found in the corresponding part of the \sphinxcode{\sphinxupquote{G4HepEm}} \sphinxhref{https://g4hepem.readthedocs.io/en/latest/}{documentation}.

\begin{sphinxadmonition}{note}{Note:}
\sphinxAtStartPar
The \sphinxcode{\sphinxupquote{hepemshow}} repository includes the pre\sphinxhyphen{}generated data file (\sphinxcode{\sphinxupquote{/data/hepem\_data.json}}) that has been extracted by using the
\sphinxcode{\sphinxupquote{HepEmShow\sphinxhyphen{}DataGeneration}} with the default material configuration settings. Providing this data file makes possible
to initialise the \sphinxcode{\sphinxupquote{G4HepEm}} data component from this file making \sphinxcode{\sphinxupquote{HepEmShow}} independent from \sphinxcode{\sphinxupquote{Geant4}}. The \sphinxcode{\sphinxupquote{HepEmShow\sphinxhyphen{}DataGeneration}}
application is also available in the \sphinxcode{\sphinxupquote{hepemshow}} repository. This can be used to re\sphinxhyphen{}generate the above data file when the goal is to change
the default material configuration (see above at the {\hyperref[\detokenize{IntroAndInstall/components:geometry}]{\sphinxcrossref{Geometry}}} section). However, as the data extraction requires the \sphinxcode{\sphinxupquote{Geant4}} dependent
initialisation of \sphinxcode{\sphinxupquote{G4HepEm}}, it requires a \sphinxcode{\sphinxupquote{Geant4}} dependent build of \sphinxcode{\sphinxupquote{G4HepEm}}. See more details in the
\sphinxhref{install\_Requirements}{Requirements} subsection of the \sphinxhref{install\_doc}{Build and Install} section.
\end{sphinxadmonition}

\sphinxAtStartPar
As mentioned above, the entire physics of the \sphinxcode{\sphinxupquote{HepEmShow}} simulation application is provided by \sphinxcode{\sphinxupquote{G4HepEm}} \sphinxcite{zzbib:g4hepem}. The required,
definitions (\sphinxcode{\sphinxupquote{.hh}} files) of the \sphinxcode{\sphinxupquote{G4HepEm}} run\sphinxhyphen{}time functionalities are pulled in by the \sphinxcode{\sphinxupquote{Physics.hh}} header while the corresponding
implementations (\sphinxcode{\sphinxupquote{.icc}} files) are in the \sphinxcode{\sphinxupquote{Physics.cc}}. The only missing implementation, that the client needs to provide, is a uniform
random number generator that needs to be utilised to complete the implementation of the \sphinxcode{\sphinxupquote{G4HepEmRandomEngine}}. This is also done in the
\sphinxcode{\sphinxupquote{Physics.cc}} file by using the local {\hyperref[\detokenize{Simulation/SimulationCodeDoc:_CPPv47URandom}]{\sphinxcrossref{\sphinxcode{\sphinxupquote{URandom}}}}} uniform random number generator. More information can be found in the code
documentation of the {\hyperref[\detokenize{Simulation/SimulationCodeDoc:_CPPv47Physics}]{\sphinxcrossref{\sphinxcode{\sphinxupquote{Physics}}}}}.

\sphinxAtStartPar
\sphinxcode{\sphinxupquote{G4HepEm}} provides two top level methods, \sphinxcode{\sphinxupquote{HowFar}} and \sphinxcode{\sphinxupquote{Perform}} in its \sphinxcode{\sphinxupquote{G4HepEmGammaManager}} and \sphinxcode{\sphinxupquote{G4HepEmElectronManager}} for \(\gamma\) and \(e^-/e^+\) particles respectively:
\begin{itemize}
\item {} 
\sphinxAtStartPar
\sphinxcode{\sphinxupquote{HowFar}}: the physics constrained step length of the given input track, i.e. how far the particle goes e.g. till the next physics
interaction takes place or it loses all its kinetic energy or due to any other physics related constraints.

\item {} 
\sphinxAtStartPar
\sphinxcode{\sphinxupquote{Perform}}: performs all necessary physics related actions and updates on the given input track, including the production of secondary
tracks in the given physics interaction (if any).

\end{itemize}

\sphinxAtStartPar
These two top level methods are utilised in the {\hyperref[\detokenize{IntroAndInstall/components:stepping-loop}]{\sphinxcrossref{Stepping loop}}} during the computation of the individual simulation step. \sphinxcode{\sphinxupquote{HowFar}} is invoked
at the pre\sphinxhyphen{}step point, i.e. at the step limit evaluation, while \sphinxcode{\sphinxupquote{Perform}} is utilised at the post\sphinxhyphen{}step point of each individual simulation step
computation inside the {\hyperref[\detokenize{Simulation/SimulationCodeDoc:_CPPv4N12SteppingLoop12GammaStepperER13G4HepEmTLDataR12G4HepEmStateR10TrackStackR8GeometryR7Resultsi}]{\sphinxcrossref{\sphinxcode{\sphinxupquote{SteppingLoop::GammaStepper()}}}}} and {\hyperref[\detokenize{Simulation/SimulationCodeDoc:_CPPv4N12SteppingLoop15ElectronStepperER13G4HepEmTLDataR12G4HepEmStateR10TrackStackR8GeometryR7Resultsi}]{\sphinxcrossref{\sphinxcode{\sphinxupquote{SteppingLoop::ElectronStepper()}}}}} methods.

\begin{sphinxadmonition}{attention}{Attention:}
\sphinxAtStartPar
Unlike the {\hyperref[\detokenize{IntroAndInstall/components:geometry}]{\sphinxcrossref{Geometry}}} of the application, the {\hyperref[\detokenize{IntroAndInstall/components:physics}]{\sphinxcrossref{Physics}}} is fully generic as \sphinxcode{\sphinxupquote{G4HepEm}} provides an application independent, generic
EM physics component similarly to the corresponding native \sphinxcode{\sphinxupquote{Geant4}} implementation. However, the \sphinxcode{\sphinxupquote{Geant4}} dependent initialisation
phase of \sphinxcode{\sphinxupquote{G4HepEm}}, i.e. the data extraction, has been eliminated from \sphinxcode{\sphinxupquote{HepEmShow}} by separating it to the additional \sphinxcode{\sphinxupquote{HepEmShow\sphinxhyphen{}DataGeneration}}
application in order to make \sphinxcode{\sphinxupquote{HepEmShow}} independent from \sphinxcode{\sphinxupquote{Geant4}}. As a consequence, the corresponding generated data file is specific to a given
material configuration and needs to be re\sphinxhyphen{}generated whenever one would like to change that material configuration as discussed above.
\end{sphinxadmonition}


\section{Event processing}
\label{\detokenize{IntroAndInstall/components:event-processing}}

\subsection{Event loop}
\label{\detokenize{IntroAndInstall/components:event-loop}}

\subsection{Stepping loop}
\label{\detokenize{IntroAndInstall/components:stepping-loop}}

\section{The \sphinxstyleliteralintitle{\sphinxupquote{HepEmShow}} application main}
\label{\detokenize{IntroAndInstall/components:the-hepemshow-application-main}}

\subsection{Input arguments}
\label{\detokenize{IntroAndInstall/components:input-arguments}}

\section{The \sphinxstyleliteralintitle{\sphinxupquote{HepEmShow\sphinxhyphen{}DataGeneration}} application main}
\label{\detokenize{IntroAndInstall/components:the-hepemshow-datageneration-application-main}}

\section{Example configurations}
\label{\detokenize{IntroAndInstall/components:example-configurations}}
\sphinxstepscope


\chapter{Code documentation}
\label{\detokenize{CodeDoc:code-documentation}}\label{\detokenize{CodeDoc::doc}}
\sphinxstepscope


\section{Code documentation of the \sphinxstyleliteralintitle{\sphinxupquote{Simulation}} part of the \sphinxstyleliteralintitle{\sphinxupquote{HepEmShow}} application.}
\label{\detokenize{Simulation/SimulationCodeDoc:code-documentation-of-the-simulation-part-of-the-hepemshow-application}}\label{\detokenize{Simulation/SimulationCodeDoc::doc}}

\subsection{The main of the \sphinxstyleliteralintitle{\sphinxupquote{HepEmShow}} application}
\label{\detokenize{Simulation/SimulationCodeDoc:the-main-of-the-hepemshow-application}}
\sphinxAtStartPar
The main funtion of the \sphinxcode{\sphinxupquote{HepEmShow}} application. 

\sphinxAtStartPar
\begin{description}
\sphinxlineitem{\sphinxstylestrong{Author}}
\sphinxAtStartPar
M. Novak 

\sphinxlineitem{\sphinxstylestrong{Date}}
\sphinxAtStartPar
July 2023

\end{description}


\sphinxAtStartPar
The main of the \sphinxcode{\sphinxupquote{HepEmShow}} application is responsible for setting up the environment, lunch the simulation and write the results. This is done by:\begin{itemize}
\item {} 
\sphinxAtStartPar
reading the input arguments provided at the execution of the application into an \sphinxcode{\sphinxupquote{{\hyperref[\detokenize{Simulation/SimulationCodeDoc:struct_input_parameters}]{\sphinxcrossref{\DUrole{std,std-ref}{InputParameters}}}}}} object. (Note, these arguments provide configuration options).

\item {} 
\sphinxAtStartPar
loading the \sphinxcode{\sphinxupquote{G4HepEm}} data and parameters from file into a \sphinxcode{\sphinxupquote{G4HepEmState}} (see the note below)

\item {} 
\sphinxAtStartPar
constructing a \sphinxcode{\sphinxupquote{G4HepEmTLData}} (also required by \sphinxcode{\sphinxupquote{G4HepEm}} and encapsulates the random number generator and some track buffers) with its random number generator (utilising the local \sphinxcode{\sphinxupquote{{\hyperref[\detokenize{Simulation/SimulationCodeDoc:class_u_random}]{\sphinxcrossref{\DUrole{std,std-ref}{URandom}}}}}} generator)

\item {} 
\sphinxAtStartPar
constructing and setting up the application \sphinxcode{\sphinxupquote{{\hyperref[\detokenize{Simulation/SimulationCodeDoc:class_geometry}]{\sphinxcrossref{\DUrole{std,std-ref}{Geometry}}}}}} according to the provided configuration input arguments (in \sphinxcode{\sphinxupquote{{\hyperref[\detokenize{Simulation/SimulationCodeDoc:struct_input_parameters}]{\sphinxcrossref{\DUrole{std,std-ref}{InputParameters}}}}}})

\item {} 
\sphinxAtStartPar
constructing and setting up a \sphinxcode{\sphinxupquote{{\hyperref[\detokenize{Simulation/SimulationCodeDoc:struct_results}]{\sphinxcrossref{\DUrole{std,std-ref}{Results}}}}}} structure that will be used to collect some data during the simulation

\item {} 
\sphinxAtStartPar
constructing and setting up the \sphinxcode{\sphinxupquote{{\hyperref[\detokenize{Simulation/SimulationCodeDoc:class_primary_generator}]{\sphinxcrossref{\DUrole{std,std-ref}{PrimaryGenerator}}}}}} of the application according to the provided configuration input arguments (in \sphinxcode{\sphinxupquote{{\hyperref[\detokenize{Simulation/SimulationCodeDoc:struct_input_parameters}]{\sphinxcrossref{\DUrole{std,std-ref}{InputParameters}}}}}})

\item {} 
\sphinxAtStartPar
the \sphinxcode{\sphinxupquote{{\hyperref[\detokenize{Simulation/SimulationCodeDoc:class_event_loop_1a7b1d4b512c5fa676bd990cfaa6d561c9}]{\sphinxcrossref{\DUrole{std,std-ref}{EventLoop::ProcessEvents}}}}}} method is invoked then to \sphinxstylestrong{perform the simulation}

\item {} 
\sphinxAtStartPar
the simulation results are witten to file (and to the standard output) by invoking \sphinxcode{\sphinxupquote{{\hyperref[\detokenize{Simulation/SimulationCodeDoc:_results_8hh_1a2e94780b3b797ba10abaf1154db56016}]{\sphinxcrossref{\DUrole{std,std-ref}{WriteResults()}}}}}} (from the \sphinxcode{\sphinxupquote{{\hyperref[\detokenize{Simulation/SimulationCodeDoc:struct_results}]{\sphinxcrossref{\DUrole{std,std-ref}{Results}}}}}})

\end{itemize}


\sphinxAtStartPar

\begin{sphinxadmonition}{note}{Note:}
\sphinxAtStartPar
The \sphinxcode{\sphinxupquote{G4HepEm}} data file is either the one included in the \sphinxcode{\sphinxupquote{HepEmShow}} repository (under \sphinxcode{\sphinxupquote{hepemshow/data/}}) or generated by the auxiliary \sphinxcode{\sphinxupquote{HepEmShow\sphinxhyphen{}DataGeneration}} application. In the former case, the data file contains all data that \sphinxcode{\sphinxupquote{G4HepEm}} needs for the simulation for the 3 default (\sphinxcode{\sphinxupquote{\{"G4\_Galactic", "G4\_PbWO4", "G4\_lAr"\}}}) materials, i.e. those used in the default \sphinxcode{\sphinxupquote{{\hyperref[\detokenize{Simulation/SimulationCodeDoc:class_geometry}]{\sphinxcrossref{\DUrole{std,std-ref}{Geometry}}}}}} configuration. 
\end{sphinxadmonition}


\begin{sphinxuseclass}{breathe-sectiondef}\subsubsection*{Functions}
\index{main (C++ function)@\spxentry{main}\spxextra{C++ function}}

\begin{fulllineitems}
\phantomsection\label{\detokenize{Simulation/SimulationCodeDoc:_CPPv44mainiA_Pc}}
\pysigstartsignatures
\pysigstartmultiline
\pysiglinewithargsret{\phantomsection\label{\detokenize{Simulation/SimulationCodeDoc:_hep_em_show_8cc_1a0ddf1224851353fc92bfbff6f499fa97}}\DUrole{kt,kt}{int}\DUrole{w,w}{  }\sphinxbfcode{\sphinxupquote{\DUrole{n,n}{main}}}}{\DUrole{kt,kt}{int}\DUrole{w,w}{  }\DUrole{n,sig-param,n}{argc}\sphinxparamcomma \DUrole{kt,kt}{char}\DUrole{w,w}{  }\DUrole{p,p}{*}\DUrole{n,sig-param,n}{argv}\DUrole{p,p}{{[}}\DUrole{p,p}{{]}}}{}
\pysigstopmultiline
\pysigstopsignatures
\sphinxAtStartPar
The main function of the \sphinxcode{\sphinxupquote{HepEmShow}} simulation application (see more in the description). 

\end{fulllineitems}


\end{sphinxuseclass}

\subsection{The \sphinxstyleliteralintitle{\sphinxupquote{Geometry}} description related code documentation}
\label{\detokenize{Simulation/SimulationCodeDoc:the-geometry-description-related-code-documentation}}
\sphinxAtStartPar
The {\hyperref[\detokenize{Simulation/SimulationCodeDoc:_CPPv48Geometry}]{\sphinxcrossref{\sphinxcode{\sphinxupquote{Geometry}}}}} of this application is built up from 5 {\hyperref[\detokenize{Simulation/SimulationCodeDoc:_CPPv43Box}]{\sphinxcrossref{\sphinxcode{\sphinxupquote{Box}}}}}
objects (by default, i.e. at non\sphinxhyphen{}zero \sphinxcode{\sphinxupquote{gap}} thickness): an \sphinxcode{\sphinxupquote{absorber}} and
a \sphinxcode{\sphinxupquote{gap}} building up a \sphinxcode{\sphinxupquote{layer}} that is repeated \sphinxtitleref{N} times along the \sphinxtitleref{x}\sphinxhyphen{}axis
building up and filling in the \sphinxcode{\sphinxupquote{calorimeter}} that is centered at the origin
and placed in the \sphinxcode{\sphinxupquote{world}}.
\index{Geometry (C++ class)@\spxentry{Geometry}\spxextra{C++ class}}

\begin{fulllineitems}
\phantomsection\label{\detokenize{Simulation/SimulationCodeDoc:_CPPv48Geometry}}
\pysigstartsignatures
\pysigstartmultiline
\pysigline{\phantomsection\label{\detokenize{Simulation/SimulationCodeDoc:class_geometry}}\DUrole{k,k}{class}\DUrole{w,w}{  }\sphinxbfcode{\sphinxupquote{\DUrole{n,n}{Geometry}}}}
\pysigstopmultiline
\pysigstopsignatures
\sphinxAtStartPar
{\hyperref[\detokenize{Simulation/SimulationCodeDoc:class_geometry}]{\sphinxcrossref{\DUrole{std,std-ref}{Geometry}}}} description for this simple simulation setup. 

\sphinxAtStartPar
\begin{description}
\sphinxlineitem{\sphinxstylestrong{Author}}
\sphinxAtStartPar
M. Novak 

\sphinxlineitem{\sphinxstylestrong{Date}}
\sphinxAtStartPar
July 2023

\end{description}


\sphinxAtStartPar
The simulation setup is a \sphinxstylestrong{configurable simplified sampling calorimeter} built up from \sphinxcode{\sphinxupquote{N}} layers of an \sphinxcode{\sphinxupquote{absorber}} and a \sphinxcode{\sphinxupquote{gap}} (both by default). The number of layers \sphinxcode{\sphinxupquote{N}}, the thickness of both the \sphinxcode{\sphinxupquote{absorber}} and \sphinxcode{\sphinxupquote{gap}} along the x\sphinxhyphen{}axes can be set and changed dynamically.

\sphinxAtStartPar
\begin{itemize}
\item {} 
\sphinxAtStartPar
\sphinxcode{\sphinxupquote{layer}}:\begin{itemize}
\item {} 
\sphinxAtStartPar
number : \sphinxcode{\sphinxupquote{fNumLayers}}

\item {} 
\sphinxAtStartPar
set/get : \sphinxcode{\sphinxupquote{{\hyperref[\detokenize{Simulation/SimulationCodeDoc:class_geometry_1a47d793bed18a5599b656e85b502109ac}]{\sphinxcrossref{\DUrole{std,std-ref}{SetNumLayers(int)}}}}}}/\sphinxcode{\sphinxupquote{{\hyperref[\detokenize{Simulation/SimulationCodeDoc:class_geometry_1a5b246547a35881d7eb2b6980952f4004}]{\sphinxcrossref{\DUrole{std,std-ref}{GetNumLayers()}}}}}}

\item {} 
\sphinxAtStartPar
thickness: \sphinxcode{\sphinxupquote{fLayerThick}} (calculated automatically from the \sphinxcode{\sphinxupquote{absorber}} and \sphinxcode{\sphinxupquote{gap}} thicknesses)

\end{itemize}


\item {} 
\sphinxAtStartPar
\sphinxcode{\sphinxupquote{absorber}}:\begin{itemize}
\item {} 
\sphinxAtStartPar
thickness: \sphinxcode{\sphinxupquote{fAbsThick}}

\item {} 
\sphinxAtStartPar
set/get : \sphinxcode{\sphinxupquote{{\hyperref[\detokenize{Simulation/SimulationCodeDoc:class_geometry_1a24eaa81106907bbef4404407b4093bbf}]{\sphinxcrossref{\DUrole{std,std-ref}{SetAbsThick(double)}}}}}}/\sphinxcode{\sphinxupquote{{\hyperref[\detokenize{Simulation/SimulationCodeDoc:class_geometry_1a45b4edf0b8cd260cf3487e5831cdb57e}]{\sphinxcrossref{\DUrole{std,std-ref}{GetAbsThick()}}}}}}

\item {} 
\sphinxAtStartPar
material : lead tungstate/atolzite (\sphinxcode{\sphinxupquote{"G4\_PbWO4"}}) with material \sphinxcode{\sphinxupquote{index = 1}} (by default)

\end{itemize}


\item {} 
\sphinxAtStartPar
\sphinxcode{\sphinxupquote{gap}}:\begin{itemize}
\item {} 
\sphinxAtStartPar
thickness: \sphinxcode{\sphinxupquote{fGapThick}}

\item {} 
\sphinxAtStartPar
set/get : \sphinxcode{\sphinxupquote{{\hyperref[\detokenize{Simulation/SimulationCodeDoc:class_geometry_1a07629f80bdfada5f95ce3657aa2887b7}]{\sphinxcrossref{\DUrole{std,std-ref}{SetGapThick(double)}}}}}}/\sphinxcode{\sphinxupquote{{\hyperref[\detokenize{Simulation/SimulationCodeDoc:class_geometry_1af925634f36ca10963e9c58f8b675fbeb}]{\sphinxcrossref{\DUrole{std,std-ref}{GetGapThick()}}}}}}

\item {} 
\sphinxAtStartPar
material : liquid argon (\sphinxcode{\sphinxupquote{"G4\_lAr"}}) with material \sphinxcode{\sphinxupquote{index = 2}} (by default)

\end{itemize}


\end{itemize}


\sphinxAtStartPar

All thickness measured along the \sphinxcode{\sphinxupquote{x}} axes while the \sphinxcode{\sphinxupquote{yz}} extent is the same both for the \sphinxcode{\sphinxupquote{absorber}} and \sphinxcode{\sphinxupquote{gap}} determined by the \sphinxcode{\sphinxupquote{fCaloSizeYZ}} which can be set dynamically by \sphinxcode{\sphinxupquote{SetCaloSizeYZ(val)/GetCaloSizeYZ()}}.
\begin{sphinxadmonition}{note}{Note:}
\sphinxAtStartPar
The default length unit is {[}mm{]} so all thicknesses and sizes are assumed to be give in {[}mm{]} units.
\end{sphinxadmonition}

\begin{sphinxadmonition}{note}{Note:}
\sphinxAtStartPar
The \sphinxcode{\sphinxupquote{gap}} thickness can be set even to zero in which case the \sphinxcode{\sphinxupquote{calorimeter}} is built up from the given number of \sphinxcode{\sphinxupquote{layer}}s of \sphinxcode{\sphinxupquote{absorber}} with the given thickness (i.e. a single material calorimeter sliced by the \sphinxcode{\sphinxupquote{layer}}s).
\end{sphinxadmonition}

\begin{sphinxadmonition}{note}{Note:}
\sphinxAtStartPar
The material indices are determined by the order of the corresponding \sphinxcode{\sphinxupquote{Geant4}} (predefined NIST) material names listed in the material name vector of the data extraction application (i.e. in \sphinxcode{\sphinxupquote{DataGeneration.cc}}). This application is used beforehand to extract the material (and cuts) dependent data required during the simulation. The default vector in \sphinxcode{\sphinxupquote{DataGeneration.cc}}, that was used to extract the provided data files, is 
\begin{sphinxVerbatim}[commandchars=\\\{\}]
\PYG{o}{/}\PYG{o}{/} \PYG{n+nb}{list} \PYG{n}{of} \PYG{n}{Geant4} \PYG{p}{(}\PYG{n}{NIST}\PYG{p}{)} \PYG{n}{material} \PYG{n}{names}
\PYG{n}{std}\PYG{p}{:}\PYG{p}{:}\PYG{n}{vector}\PYG{o}{\PYGZlt{}}\PYG{n}{std}\PYG{p}{:}\PYG{p}{:}\PYG{n}{string}\PYG{o}{\PYGZgt{}} \PYG{n}{matList} \PYG{p}{\PYGZob{}}\PYG{l+s+s2}{\PYGZdq{}}\PYG{l+s+s2}{G4\PYGZus{}Galactic}\PYG{l+s+s2}{\PYGZdq{}}\PYG{p}{,} \PYG{l+s+s2}{\PYGZdq{}}\PYG{l+s+s2}{G4\PYGZus{}PbWO4}\PYG{l+s+s2}{\PYGZdq{}}\PYG{p}{,} \PYG{l+s+s2}{\PYGZdq{}}\PYG{l+s+s2}{G4\PYGZus{}lAr}\PYG{l+s+s2}{\PYGZdq{}}\PYG{p}{\PYGZcb{}}\PYG{p}{;}
\end{sphinxVerbatim}
 hence the above \sphinxcode{\sphinxupquote{material \sphinxhyphen{} index}} mapping. Changing the material name(s) in this above vector (especially at index \sphinxcode{\sphinxupquote{1}} and/or \sphinxcode{\sphinxupquote{2}} as the vacuum is always needed to fill the container volumes like the \sphinxcode{\sphinxupquote{layer}}, \sphinxcode{\sphinxupquote{calorimeter}} or the \sphinxcode{\sphinxupquote{world}} volumes), regenerating the data, then executing the same \sphinxcode{\sphinxupquote{Simulation}} application, corresponds to changing the mateiral (with the given index) in the simulation. A \sphinxhref{https://geant4-userdoc.web.cern.ch/UsersGuides/ForApplicationDeveloper/html/Appendix/materialNames.html}{list of the available predefined Geant4 NIST material names} can be found in the corresponding part of the Geant4 Documentation.
\end{sphinxadmonition}


\sphinxAtStartPar
A single layer is composed from the above \sphinxcode{\sphinxupquote{absorber}} and \sphinxcode{\sphinxupquote{gap}} while the entire \sphinxcode{\sphinxupquote{calorimeter}} is built up form the given number of identical \sphinxcode{\sphinxupquote{layer}}s shifted along the \sphinxcode{\sphinxupquote{x}} axes. The \sphinxcode{\sphinxupquote{calorimeter}} center, i.e. the {[}0,0,0{]} position of the corresponding {\hyperref[\detokenize{Simulation/SimulationCodeDoc:class_box}]{\sphinxcrossref{\DUrole{std,std-ref}{Box}}}} shape local coordinate, is at the global origin (i.e. no translation nor rotation is applied). The entire calorimeter is placed inside the \sphinxcode{\sphinxupquote{world}} that is the limit of our simulation universe. The \sphinxcode{\sphinxupquote{layer}}, \sphinxcode{\sphinxupquote{calorimeter}} and \sphinxcode{\sphinxupquote{world}} is filled with vacuum (very low density hydrogen), so only the \sphinxcode{\sphinxupquote{absorber}} and the \sphinxcode{\sphinxupquote{gap}} have non\sphinxhyphen{}vacuum like materials.

\sphinxAtStartPar
The shape of all objects (\sphinxcode{\sphinxupquote{absorber}}, \sphinxcode{\sphinxupquote{gap}}, \sphinxcode{\sphinxupquote{layer}}, \sphinxcode{\sphinxupquote{calorimeter}}, \sphinxcode{\sphinxupquote{world}}) is \sphinxcode{\sphinxupquote{{\hyperref[\detokenize{Simulation/SimulationCodeDoc:class_box}]{\sphinxcrossref{\DUrole{std,std-ref}{Box}}}}}}. A box object is constructed for each in the constructor by setting the appropriate name and material index fields. Their proper sizes are calculated and updated automatically whenever one of the above setters, affecting any of the thicknesses or sizes, is invoked.

\sphinxAtStartPar
The geometry can also provide an apropriate initial \sphinxcode{\sphinxupquote{x}} position for the primary particles locating in between the \sphinxcode{\sphinxupquote{world}} and the \sphinxcode{\sphinxupquote{calorimeter}} on the left hand side (\sphinxcode{\sphinxupquote{{\hyperref[\detokenize{Simulation/SimulationCodeDoc:class_geometry_1a69f072bb3a6ecda07dd7f8c907fc03c9}]{\sphinxcrossref{\DUrole{std,std-ref}{GetPrimaryXposition()}}}}}}). The \sphinxcode{\sphinxupquote{x}} position, where the \sphinxcode{\sphinxupquote{calorimeter}} starts on the left hand side, can also be obtained (\sphinxcode{\sphinxupquote{{\hyperref[\detokenize{Simulation/SimulationCodeDoc:class_geometry_1a783dfbcefde35a75bee9d08d33fb3908}]{\sphinxcrossref{\DUrole{std,std-ref}{GetCaloStartXposition()}}}}}}). These method will always give an appropriate value as the corresponding data are also updated dynamically whenever any of the thicknesses or sizes are modified.

\sphinxAtStartPar
The geometry also provides a very simple “navigation” through its \sphinxcode{\sphinxupquote{{\hyperref[\detokenize{Simulation/SimulationCodeDoc:class_geometry_1af34c4d9e94d5026ed64104c3accbb354}]{\sphinxcrossref{\DUrole{std,std-ref}{CalculateDistanceToOut(double*, double*, Box**, int*, int*)}}}}}} method that determines:\begin{itemize}
\item {} 
\sphinxAtStartPar
the (deepest) volume/box in which the given global \sphinxcode{\sphinxupquote{position}} is located

\item {} 
\sphinxAtStartPar
the index of the layer (only if the point is located inside the \sphinxcode{\sphinxupquote{calorimeter}})

\item {} 
\sphinxAtStartPar
and the index of the absorber: 0 for the \sphinxcode{\sphinxupquote{absorber}} and 1 for the \sphinxcode{\sphinxupquote{gap}} (but only in case the point is inside the calorimeter)

\end{itemize}


\sphinxAtStartPar
At the end, it returns:\begin{itemize}
\item {} 
\sphinxAtStartPar
a large (1E+20 {[}mm{]}) value whenever the \sphinxcode{\sphinxupquote{position}} and \sphinxcode{\sphinxupquote{direction}} is such that the particle is about leaving the \sphinxcode{\sphinxupquote{calorimeter}} (i.e. going to vacuum then would hit the boundary of the \sphinxcode{\sphinxupquote{world}} at the end of that step)

\item {} 
\sphinxAtStartPar
otherwise: the distance to the boundary of the volume in which the given \sphinxcode{\sphinxupquote{position}} was located: from the given \sphinxcode{\sphinxupquote{position}} along the given \sphinxcode{\sphinxupquote{direction}}.

\end{itemize}


\sphinxAtStartPar
Note, that this distance is computed by using the corresponding box/volume method (namely \sphinxcode{\sphinxupquote{{\hyperref[\detokenize{Simulation/SimulationCodeDoc:class_box_1aad17ace7ec8e5b684b09833a3d35a2bc}]{\sphinxcrossref{\DUrole{std,std-ref}{Box::DistanceToOut(double*,double*) const}}}}}}) and the \sphinxcode{\sphinxupquote{box}} object in which the given \sphinxcode{\sphinxupquote{position}} was located:\begin{itemize}
\item {} 
\sphinxAtStartPar
distance to out is 0 if a given \sphinxcode{\sphinxupquote{position}} is outside of that volume/box

\item {} 
\sphinxAtStartPar
volume boundaries, closer to \sphinxcode{\sphinxupquote{position}} than half of the \sphinxcode{\sphinxupquote{{\hyperref[\detokenize{Simulation/SimulationCodeDoc:class_box_1a92e002367dac759cdb49991eccb80869}]{\sphinxcrossref{\DUrole{std,std-ref}{Box::kCarTolerance}}}}}} (= 1E\sphinxhyphen{}9 {[}mm{]}), are ignored when the direction is pointing out, i.e. these positions are considered to be outside, leading to 0 distance to the volume boundary from inside. (see more on \sphinxcode{\sphinxupquote{inside}}, \sphinxcode{\sphinxupquote{surface}} and \sphinxcode{\sphinxupquote{outside}} at the \sphinxcode{\sphinxupquote{{\hyperref[\detokenize{Simulation/SimulationCodeDoc:class_box}]{\sphinxcrossref{\DUrole{std,std-ref}{Box}}}}}} documentation)

\end{itemize}


\sphinxAtStartPar
A small (e.g. 1E\sphinxhyphen{}6 {[}mm{]}) push along the current direction might be applied in the steppers when the distance to out was found to be 0.

\sphinxAtStartPar
This is because it’s assumed, that the 0 distance is due to the employed simple location calculation, that ignors the tolerance, instead of using an apropriate navigator. Namely, the point was located in volume A. but it’s actually on its \sphinxcode{\sphinxupquote{surface}}: closer than \sphinxcode{\sphinxupquote{{\hyperref[\detokenize{Simulation/SimulationCodeDoc:class_box_1a92e002367dac759cdb49991eccb80869}]{\sphinxcrossref{\DUrole{std,std-ref}{Box::kCarTolerance}}}}}}/2 to its boundary. Morover, the direction is pointing toward to the next, volume B., that is just on the other side of this boundary. This correctly gives 0 distance to the boundary of volume A as actually the given step will be done in volume B. Therefore, the small push is to overcome the volume boundary, i.e. to push the point to be on the other side of the boundary between volume A and B. Then, the point is calulated to be in volume B now when relocating and as the direction is pointing inside volume B, the expected distance to the next boundary of volume B is computed. 

\begin{sphinxuseclass}{breathe-sectiondef}\subsubsection*{Public Functions}
\index{Geometry::Geometry (C++ function)@\spxentry{Geometry::Geometry}\spxextra{C++ function}}

\begin{fulllineitems}
\phantomsection\label{\detokenize{Simulation/SimulationCodeDoc:_CPPv4N8Geometry8GeometryEv}}
\pysigstartsignatures
\pysigstartmultiline
\pysiglinewithargsret{\phantomsection\label{\detokenize{Simulation/SimulationCodeDoc:class_geometry_1a4c301c163c63d21ed08c17b0f4e131d3}}\sphinxbfcode{\sphinxupquote{\DUrole{n,n}{Geometry}}}}{}{}
\pysigstopmultiline
\pysigstopsignatures
\sphinxAtStartPar
Constructor: sets the default configuration, creates the {\hyperref[\detokenize{Simulation/SimulationCodeDoc:class_box}]{\sphinxcrossref{\DUrole{std,std-ref}{Box}}}} objects for all components. 

\end{fulllineitems}

\index{Geometry::\textasciitilde{}Geometry (C++ function)@\spxentry{Geometry::\textasciitilde{}Geometry}\spxextra{C++ function}}

\begin{fulllineitems}
\phantomsection\label{\detokenize{Simulation/SimulationCodeDoc:_CPPv4N8GeometryD0Ev}}
\pysigstartsignatures
\pysigstartmultiline
\pysiglinewithargsret{\phantomsection\label{\detokenize{Simulation/SimulationCodeDoc:class_geometry_1ad55e832122ab3a2833dcaa6507867678}}\sphinxbfcode{\sphinxupquote{\DUrole{n,n}{\textasciitilde{}Geometry}}}}{}{}
\pysigstopmultiline
\pysigstopsignatures
\sphinxAtStartPar
Destructor: deletes the {\hyperref[\detokenize{Simulation/SimulationCodeDoc:class_box}]{\sphinxcrossref{\DUrole{std,std-ref}{Box}}}} objects that represents the volume of the components. 

\end{fulllineitems}

\index{Geometry::SetNumLayers (C++ function)@\spxentry{Geometry::SetNumLayers}\spxextra{C++ function}}

\begin{fulllineitems}
\phantomsection\label{\detokenize{Simulation/SimulationCodeDoc:_CPPv4N8Geometry12SetNumLayersEi}}
\pysigstartsignatures
\pysigstartmultiline
\pysiglinewithargsret{\phantomsection\label{\detokenize{Simulation/SimulationCodeDoc:class_geometry_1a47d793bed18a5599b656e85b502109ac}}\DUrole{k,k}{inline}\DUrole{w,w}{  }\DUrole{kt,kt}{void}\DUrole{w,w}{  }\sphinxbfcode{\sphinxupquote{\DUrole{n,n}{SetNumLayers}}}}{\DUrole{kt,kt}{int}\DUrole{w,w}{  }\DUrole{n,sig-param,n}{nlayers}}{}
\pysigstopmultiline
\pysigstopsignatures
\sphinxAtStartPar
Sets the number of layers the entire calorimeter should be built up. 

\sphinxAtStartPar
\begin{quote}\begin{description}
\sphinxlineitem{param nlayers}
\sphinxAtStartPar
\sphinxstylestrong{{[}in{]}} Number of layers (must be \textgreater{} 0) requested (all parameters are recalculated). 

\end{description}\end{quote}


\end{fulllineitems}

\index{Geometry::GetNumLayers (C++ function)@\spxentry{Geometry::GetNumLayers}\spxextra{C++ function}}

\begin{fulllineitems}
\phantomsection\label{\detokenize{Simulation/SimulationCodeDoc:_CPPv4NK8Geometry12GetNumLayersEv}}
\pysigstartsignatures
\pysigstartmultiline
\pysiglinewithargsret{\phantomsection\label{\detokenize{Simulation/SimulationCodeDoc:class_geometry_1a5b246547a35881d7eb2b6980952f4004}}\DUrole{k,k}{inline}\DUrole{w,w}{  }\DUrole{kt,kt}{int}\DUrole{w,w}{  }\sphinxbfcode{\sphinxupquote{\DUrole{n,n}{GetNumLayers}}}}{}{\DUrole{w,w}{  }\DUrole{k,k}{const}}
\pysigstopmultiline
\pysigstopsignatures
\sphinxAtStartPar
Gives the number of layers the calorimeter is built up. 

\sphinxAtStartPar
\begin{quote}\begin{description}
\sphinxlineitem{return}
\sphinxAtStartPar
number of layers. 

\end{description}\end{quote}


\end{fulllineitems}

\index{Geometry::GetCaloThick (C++ function)@\spxentry{Geometry::GetCaloThick}\spxextra{C++ function}}

\begin{fulllineitems}
\phantomsection\label{\detokenize{Simulation/SimulationCodeDoc:_CPPv4NK8Geometry12GetCaloThickEv}}
\pysigstartsignatures
\pysigstartmultiline
\pysiglinewithargsret{\phantomsection\label{\detokenize{Simulation/SimulationCodeDoc:class_geometry_1a957b5b3bb7f9a971b691f0741d7ca172}}\DUrole{k,k}{inline}\DUrole{w,w}{  }\DUrole{kt,kt}{double}\DUrole{w,w}{  }\sphinxbfcode{\sphinxupquote{\DUrole{n,n}{GetCaloThick}}}}{}{\DUrole{w,w}{  }\DUrole{k,k}{const}}
\pysigstopmultiline
\pysigstopsignatures
\sphinxAtStartPar
Gives the thickness of the \sphinxcode{\sphinxupquote{calorimeter}} (i.e. 

\sphinxAtStartPar
full size along the x\sphinxhyphen{}axis). \begin{quote}\begin{description}
\sphinxlineitem{return}
\sphinxAtStartPar
thickness of the \sphinxcode{\sphinxupquote{calorimeter}} in {[}mm{]} units. 

\end{description}\end{quote}


\end{fulllineitems}

\index{Geometry::SetAbsThick (C++ function)@\spxentry{Geometry::SetAbsThick}\spxextra{C++ function}}

\begin{fulllineitems}
\phantomsection\label{\detokenize{Simulation/SimulationCodeDoc:_CPPv4N8Geometry11SetAbsThickEd}}
\pysigstartsignatures
\pysigstartmultiline
\pysiglinewithargsret{\phantomsection\label{\detokenize{Simulation/SimulationCodeDoc:class_geometry_1a24eaa81106907bbef4404407b4093bbf}}\DUrole{k,k}{inline}\DUrole{w,w}{  }\DUrole{kt,kt}{void}\DUrole{w,w}{  }\sphinxbfcode{\sphinxupquote{\DUrole{n,n}{SetAbsThick}}}}{\DUrole{kt,kt}{double}\DUrole{w,w}{  }\DUrole{n,sig-param,n}{thickness}}{}
\pysigstopmultiline
\pysigstopsignatures
\sphinxAtStartPar
Sets the required absorber thickness (i.e. 

\sphinxAtStartPar
full size along the x\sphinxhyphen{}axis). \begin{quote}\begin{description}
\sphinxlineitem{param thickness}
\sphinxAtStartPar
\sphinxstylestrong{{[}in{]}} Required thickness of the \sphinxcode{\sphinxupquote{absorber}} in {[}mm{]}. 

\end{description}\end{quote}


\end{fulllineitems}

\index{Geometry::GetAbsThick (C++ function)@\spxentry{Geometry::GetAbsThick}\spxextra{C++ function}}

\begin{fulllineitems}
\phantomsection\label{\detokenize{Simulation/SimulationCodeDoc:_CPPv4NK8Geometry11GetAbsThickEv}}
\pysigstartsignatures
\pysigstartmultiline
\pysiglinewithargsret{\phantomsection\label{\detokenize{Simulation/SimulationCodeDoc:class_geometry_1a45b4edf0b8cd260cf3487e5831cdb57e}}\DUrole{k,k}{inline}\DUrole{w,w}{  }\DUrole{kt,kt}{double}\DUrole{w,w}{  }\sphinxbfcode{\sphinxupquote{\DUrole{n,n}{GetAbsThick}}}}{}{\DUrole{w,w}{  }\DUrole{k,k}{const}}
\pysigstopmultiline
\pysigstopsignatures
\sphinxAtStartPar
Gives the thickness of the \sphinxcode{\sphinxupquote{absorber}} (i.e. 

\sphinxAtStartPar
full size along the x\sphinxhyphen{}axis). \begin{quote}\begin{description}
\sphinxlineitem{return}
\sphinxAtStartPar
thickness of the \sphinxcode{\sphinxupquote{absorber}} in {[}mm{]} units. 

\end{description}\end{quote}


\end{fulllineitems}

\index{Geometry::SetGapThick (C++ function)@\spxentry{Geometry::SetGapThick}\spxextra{C++ function}}

\begin{fulllineitems}
\phantomsection\label{\detokenize{Simulation/SimulationCodeDoc:_CPPv4N8Geometry11SetGapThickEd}}
\pysigstartsignatures
\pysigstartmultiline
\pysiglinewithargsret{\phantomsection\label{\detokenize{Simulation/SimulationCodeDoc:class_geometry_1a07629f80bdfada5f95ce3657aa2887b7}}\DUrole{k,k}{inline}\DUrole{w,w}{  }\DUrole{kt,kt}{void}\DUrole{w,w}{  }\sphinxbfcode{\sphinxupquote{\DUrole{n,n}{SetGapThick}}}}{\DUrole{kt,kt}{double}\DUrole{w,w}{  }\DUrole{n,sig-param,n}{thickness}}{}
\pysigstopmultiline
\pysigstopsignatures
\sphinxAtStartPar
Sets the required \sphinxcode{\sphinxupquote{gap}} thickness (i.e. 

\sphinxAtStartPar
full size along the x\sphinxhyphen{}axis).

\sphinxAtStartPar
Note, that the \sphinxcode{\sphinxupquote{gap}} thickness can also be set to zero. The calorimeter is built up from a single material layers, i.e. a block of material sliced along the x\sphinxhyphen{}axis.

\sphinxAtStartPar
\begin{quote}\begin{description}
\sphinxlineitem{param thickness}
\sphinxAtStartPar
\sphinxstylestrong{{[}in{]}} Required thickness of the \sphinxcode{\sphinxupquote{gap}} in {[}mm{]} (can be set to 0). 

\end{description}\end{quote}


\end{fulllineitems}

\index{Geometry::GetGapThick (C++ function)@\spxentry{Geometry::GetGapThick}\spxextra{C++ function}}

\begin{fulllineitems}
\phantomsection\label{\detokenize{Simulation/SimulationCodeDoc:_CPPv4NK8Geometry11GetGapThickEv}}
\pysigstartsignatures
\pysigstartmultiline
\pysiglinewithargsret{\phantomsection\label{\detokenize{Simulation/SimulationCodeDoc:class_geometry_1af925634f36ca10963e9c58f8b675fbeb}}\DUrole{k,k}{inline}\DUrole{w,w}{  }\DUrole{kt,kt}{double}\DUrole{w,w}{  }\sphinxbfcode{\sphinxupquote{\DUrole{n,n}{GetGapThick}}}}{}{\DUrole{w,w}{  }\DUrole{k,k}{const}}
\pysigstopmultiline
\pysigstopsignatures
\sphinxAtStartPar
Gives the thickness of the \sphinxcode{\sphinxupquote{gap}} (i.e. 

\sphinxAtStartPar
full size along the x\sphinxhyphen{}axis). \begin{quote}\begin{description}
\sphinxlineitem{return}
\sphinxAtStartPar
thickness of the \sphinxcode{\sphinxupquote{gap}} in {[}mm{]} units. 

\end{description}\end{quote}


\end{fulllineitems}

\index{Geometry::SetCaloSizeYZ (C++ function)@\spxentry{Geometry::SetCaloSizeYZ}\spxextra{C++ function}}

\begin{fulllineitems}
\phantomsection\label{\detokenize{Simulation/SimulationCodeDoc:_CPPv4N8Geometry13SetCaloSizeYZEd}}
\pysigstartsignatures
\pysigstartmultiline
\pysiglinewithargsret{\phantomsection\label{\detokenize{Simulation/SimulationCodeDoc:class_geometry_1a788be363c5ccfd2ccc25e68c404fec3b}}\DUrole{k,k}{inline}\DUrole{w,w}{  }\DUrole{kt,kt}{void}\DUrole{w,w}{  }\sphinxbfcode{\sphinxupquote{\DUrole{n,n}{SetCaloSizeYZ}}}}{\DUrole{kt,kt}{double}\DUrole{w,w}{  }\DUrole{n,sig-param,n}{val}}{}
\pysigstopmultiline
\pysigstopsignatures
\sphinxAtStartPar
Sets the transverse size (i.e. 

\sphinxAtStartPar
full size along the yz\sphinxhyphen{}axes).

\sphinxAtStartPar
Note, this also determines \sphinxcode{\sphinxupquote{yz}} sizes of the corresponding the \sphinxcode{\sphinxupquote{absorber}}, \sphinxcode{\sphinxupquote{gap}} and \sphinxcode{\sphinxupquote{layer}} volumes/shapes.

\sphinxAtStartPar
\begin{quote}\begin{description}
\sphinxlineitem{param val}
\sphinxAtStartPar
\sphinxstylestrong{{[}in{]}} Required full transvers size of the \sphinxcode{\sphinxupquote{calorimeter}} in {[}mm{]}. 

\end{description}\end{quote}


\end{fulllineitems}

\index{Geometry::GetCaloSizeYZ (C++ function)@\spxentry{Geometry::GetCaloSizeYZ}\spxextra{C++ function}}

\begin{fulllineitems}
\phantomsection\label{\detokenize{Simulation/SimulationCodeDoc:_CPPv4NK8Geometry13GetCaloSizeYZEv}}
\pysigstartsignatures
\pysigstartmultiline
\pysiglinewithargsret{\phantomsection\label{\detokenize{Simulation/SimulationCodeDoc:class_geometry_1a60cf831680c87637e3efdbab06559688}}\DUrole{k,k}{inline}\DUrole{w,w}{  }\DUrole{kt,kt}{double}\DUrole{w,w}{  }\sphinxbfcode{\sphinxupquote{\DUrole{n,n}{GetCaloSizeYZ}}}}{}{\DUrole{w,w}{  }\DUrole{k,k}{const}}
\pysigstopmultiline
\pysigstopsignatures
\sphinxAtStartPar
Gives the transverse size of the \sphinxcode{\sphinxupquote{calorimeter}} (i.e. 

\sphinxAtStartPar
full size along the yz\sphinxhyphen{}axis). \begin{quote}\begin{description}
\sphinxlineitem{return}
\sphinxAtStartPar
full transverse size of the \sphinxcode{\sphinxupquote{calorimeter}} volume/shape in {[}mm{]} units. 

\end{description}\end{quote}


\end{fulllineitems}

\index{Geometry::GetPrimaryXposition (C++ function)@\spxentry{Geometry::GetPrimaryXposition}\spxextra{C++ function}}

\begin{fulllineitems}
\phantomsection\label{\detokenize{Simulation/SimulationCodeDoc:_CPPv4NK8Geometry19GetPrimaryXpositionEv}}
\pysigstartsignatures
\pysigstartmultiline
\pysiglinewithargsret{\phantomsection\label{\detokenize{Simulation/SimulationCodeDoc:class_geometry_1a69f072bb3a6ecda07dd7f8c907fc03c9}}\DUrole{k,k}{inline}\DUrole{w,w}{  }\DUrole{kt,kt}{double}\DUrole{w,w}{  }\sphinxbfcode{\sphinxupquote{\DUrole{n,n}{GetPrimaryXposition}}}}{}{\DUrole{w,w}{  }\DUrole{k,k}{const}}
\pysigstopmultiline
\pysigstopsignatures
\sphinxAtStartPar
Provides the x\sphinxhyphen{}coordinate of the mid\sphinxhyphen{}position between the \sphinxcode{\sphinxupquote{world}} and \sphinxcode{\sphinxupquote{calorimeter}} boundaries on the left hand side. 

\sphinxAtStartPar
Note that this is only for the primary generator, the primary tracks should be inside the calorimeter or its boundary but pointing inside (see more at \sphinxcode{\sphinxupquote{{\hyperref[\detokenize{Simulation/SimulationCodeDoc:class_geometry_1af34c4d9e94d5026ed64104c3accbb354}]{\sphinxcrossref{\DUrole{std,std-ref}{CalculateDistanceToOut()}}}}}})

\sphinxAtStartPar
\begin{quote}\begin{description}
\sphinxlineitem{return}
\sphinxAtStartPar
the x\sphinxhyphen{}coordinate of the mid\sphinxhyphen{}point between the \sphinxcode{\sphinxupquote{world}} and \sphinxcode{\sphinxupquote{calorimeter}} boundaries on the left. 

\end{description}\end{quote}


\end{fulllineitems}

\index{Geometry::GetCaloStartXposition (C++ function)@\spxentry{Geometry::GetCaloStartXposition}\spxextra{C++ function}}

\begin{fulllineitems}
\phantomsection\label{\detokenize{Simulation/SimulationCodeDoc:_CPPv4NK8Geometry21GetCaloStartXpositionEv}}
\pysigstartsignatures
\pysigstartmultiline
\pysiglinewithargsret{\phantomsection\label{\detokenize{Simulation/SimulationCodeDoc:class_geometry_1a783dfbcefde35a75bee9d08d33fb3908}}\DUrole{k,k}{inline}\DUrole{w,w}{  }\DUrole{kt,kt}{double}\DUrole{w,w}{  }\sphinxbfcode{\sphinxupquote{\DUrole{n,n}{GetCaloStartXposition}}}}{}{\DUrole{w,w}{  }\DUrole{k,k}{const}}
\pysigstopmultiline
\pysigstopsignatures
\sphinxAtStartPar
Provides the x\sphinxhyphen{}coordinate on the \sphinxcode{\sphinxupquote{calorimeter}} boundary on the left hand side. 

\sphinxAtStartPar
Note this is the initial x\sphinxhyphen{}coordinate of each primary track while their direction should point toward the calorimeter (i.e. having positive x\sphinxhyphen{}coordiante).

\sphinxAtStartPar
\begin{quote}\begin{description}
\sphinxlineitem{return}
\sphinxAtStartPar
the x\sphinxhyphen{}coordinate of the \sphinxcode{\sphinxupquote{calorimeter}} boundary on the left hand side. 

\end{description}\end{quote}


\end{fulllineitems}

\index{Geometry::CalculateDistanceToOut (C++ function)@\spxentry{Geometry::CalculateDistanceToOut}\spxextra{C++ function}}

\begin{fulllineitems}
\phantomsection\label{\detokenize{Simulation/SimulationCodeDoc:_CPPv4N8Geometry22CalculateDistanceToOutEPdPdPP3BoxPiPi}}
\pysigstartsignatures
\pysigstartmultiline
\pysiglinewithargsret{\phantomsection\label{\detokenize{Simulation/SimulationCodeDoc:class_geometry_1af34c4d9e94d5026ed64104c3accbb354}}\DUrole{kt,kt}{double}\DUrole{w,w}{  }\sphinxbfcode{\sphinxupquote{\DUrole{n,n}{CalculateDistanceToOut}}}}{\DUrole{kt,kt}{double}\DUrole{w,w}{  }\DUrole{p,p}{*}\DUrole{n,sig-param,n}{r}\sphinxparamcomma \DUrole{kt,kt}{double}\DUrole{w,w}{  }\DUrole{p,p}{*}\DUrole{n,sig-param,n}{v}\sphinxparamcomma {\hyperref[\detokenize{Simulation/SimulationCodeDoc:_CPPv43Box}]{\sphinxcrossref{\DUrole{n,n,n}{Box}}}}\DUrole{w,w}{  }\DUrole{p,p}{*}\DUrole{p,p}{*}\DUrole{n,sig-param,n}{currentVolume}\sphinxparamcomma \DUrole{kt,kt}{int}\DUrole{w,w}{  }\DUrole{p,p}{*}\DUrole{n,sig-param,n}{indxLayer}\sphinxparamcomma \DUrole{kt,kt}{int}\DUrole{w,w}{  }\DUrole{p,p}{*}\DUrole{n,sig-param,n}{indxAbs}}{}
\pysigstopmultiline
\pysigstopsignatures
\sphinxAtStartPar
Locates a point in the geometry and calculates the distance till the next boundary. 

\sphinxAtStartPar
This method is supposed to be called at the pre\sphinxhyphen{}step point of the simulation step with the global pre\sphinxhyphen{}step ponit coordinates and actual direction in order to:\begin{itemize}
\item {} 
\sphinxAtStartPar
determine the volume in which this simulation step will be done (and more importantly, the material as everything depends on that (at least))

\item {} 
\sphinxAtStartPar
the distance to the boundary of that volume along the given diretion (as the material might change on the other side of that boundary)

\end{itemize}


\sphinxAtStartPar
The pre\sphinxhyphen{}step point is supposed to be inside the \sphinxcode{\sphinxupquote{calorimeter}} volume, i.e. either\begin{itemize}
\item {} 
\sphinxAtStartPar
\sphinxcode{\sphinxupquote{inside}}: deeper than \sphinxcode{\sphinxupquote{{\hyperref[\detokenize{Simulation/SimulationCodeDoc:class_box_1a92e002367dac759cdb49991eccb80869}]{\sphinxcrossref{\DUrole{std,std-ref}{Box::kCarTolerance}}}}}}/2 form any of its boundaries

\item {} 
\sphinxAtStartPar
on \sphinxcode{\sphinxupquote{surface}}: closer to a boundary than \sphinxcode{\sphinxupquote{{\hyperref[\detokenize{Simulation/SimulationCodeDoc:class_box_1a92e002367dac759cdb49991eccb80869}]{\sphinxcrossref{\DUrole{std,std-ref}{Box::kCarTolerance}}}}}}/2 While the distance to the \sphinxcode{\sphinxupquote{calorimeter}} volume boundary is \textgreater{} zero in the first case, this depends on the actual direction in the second case:

\item {} 
\sphinxAtStartPar
zero: when the direction is pointing outside of that boundary, i.e. particle is about leaving the volume

\item {} 
\sphinxAtStartPar
positive: as the particle is about moving in the volume otherwise These are true for all (and not only for the \sphinxcode{\sphinxupquote{calorimeter}}) volumes!

\end{itemize}


\sphinxAtStartPar
During the simulation, each primary track starts from the \sphinxcode{\sphinxupquote{calorimeter}} volume boundary with a direction that is pointing inside, i.e. ensured to be inside (on \sphinxcode{\sphinxupquote{surface}} but pointing in). All tracks are terminated when the particle is about leaving the \sphinxcode{\sphinxupquote{calorimeter}}, i.e. the particle is on \sphinxcode{\sphinxupquote{surface}} and pointing out (as the step would be done in the vaccum ending on the boundary of our \sphinxcode{\sphinxupquote{world}}). Therefore, all step pints, and secondary tracks created at some of these points, are also ensured to be inside the \sphinxcode{\sphinxupquote{calorimeter}} (as defined above).

\sphinxAtStartPar
In order to achive the above, this method returns with a large (1E+20 {[}mm{]}) distance to boundary whenever the particle is about leaving the \sphinxcode{\sphinxupquote{calorimeter}}. The point (the step) is located to be in the \sphinxcode{\sphinxupquote{world}} volume (\sphinxcode{\sphinxupquote{layer}} and \sphinxcode{\sphinxupquote{absorber}} indices are set to \sphinxhyphen{}1). Otherwise, the point is located, i.e. the deepest volume inside the \sphinxcode{\sphinxupquote{calorimeter}} in which the point is located, is determined, the \sphinxcode{\sphinxupquote{layer}} and \sphinxcode{\sphinxupquote{absorber}} indices are set.

\sphinxAtStartPar
However, this is done based on a simply computation of the \sphinxcode{\sphinxupquote{layer}} index (based on its thickness) then the same within the layer. In other words, this is done without considering the tolerance or the direction (unlike in Geant4, having a robust but complex navigator for this). Therefore, it might be the case that the point is calculated to be inside a given volume but actually it’s on the \sphinxcode{\sphinxupquote{surface}} while moving out. The corresponding simulation step should actually be performed in the next volume (that is just on the other side of that boundary). This is detected during the simulation step computation, as this method returns zero distance in this case, and:\begin{itemize}
\item {} 
\sphinxAtStartPar
a small push of 1E\sphinxhyphen{}6 {[}mm{]} is applied along the current direction (just to push the point to the other side of the boundary)

\item {} 
\sphinxAtStartPar
this method is called again with the new position: calculated to be in the good volume now

\end{itemize}


\sphinxAtStartPar
The input position, given in global coordinates, always transformed to the local system of the given volume. This is an identity transformation for the \sphinxcode{\sphinxupquote{calorimeter}} (as not translated nor rotated). Then for the \sphinxcode{\sphinxupquote{layer}}, the translation vector (having non\sphinxhyphen{}zero only its x\sphinxhyphen{}component) is determined based on the first layer x\sphinxhyphen{}position (where the \sphinxcode{\sphinxupquote{calorimeter}} starts), the thickness of the \sphinxcode{\sphinxupquote{layer}} and the current x\sphinxhyphen{}position of the point. After transforming the point to \sphinxcode{\sphinxupquote{layer}} local coordinates, the position is transformed further either to \sphinxcode{\sphinxupquote{absorber}} or \sphinxcode{\sphinxupquote{gap}} local coorinates depending on which the point was calculated to be in. At the end, the input position vector contains the position of the point in the local system of the volume/shape that the pint was calculated to be located. Therefore, this local coorinates can be used later directly in any shape ({\hyperref[\detokenize{Simulation/SimulationCodeDoc:class_box}]{\sphinxcrossref{\DUrole{std,std-ref}{Box}}}}) methods, e.g. for computing the safety.

\sphinxAtStartPar
\begin{quote}\begin{description}
\sphinxlineitem{param r}
\sphinxAtStartPar
\sphinxstylestrong{{[}inout{]}} pointer to a 3D array that stores the x, y and z coorinates of the position in global coordiantes at input. These will be updated to be local coordinates in the system of the volume ({\hyperref[\detokenize{Simulation/SimulationCodeDoc:class_box}]{\sphinxcrossref{\DUrole{std,std-ref}{Box}}}}) in which the point was located in. 

\sphinxlineitem{param v}
\sphinxAtStartPar
\sphinxstylestrong{{[}in{]}} pointer to a 3D array that stores the x, y and z coorinates of the current normalised direction vector 

\sphinxlineitem{param currentVolume}
\sphinxAtStartPar
\sphinxstylestrong{{[}inout{]}} address of a pointer to a {\hyperref[\detokenize{Simulation/SimulationCodeDoc:class_box}]{\sphinxcrossref{\DUrole{std,std-ref}{Box}}}} object that can be anything at input, while at output the corresponding pointer is set to the \sphinxcode{\sphinxupquote{{\hyperref[\detokenize{Simulation/SimulationCodeDoc:class_box}]{\sphinxcrossref{\DUrole{std,std-ref}{Box}}}}}} object that represents the volume in which the given point was calculated to be located 

\sphinxlineitem{param indxLayer}
\sphinxAtStartPar
\sphinxstylestrong{{[}inout{]}} pointer to an integer that can be anything at input, while at output it will be the indx of tha \sphinxcode{\sphinxupquote{layer}} in which the given point was calculated to be located (\sphinxhyphen{}1 when the particle is leaving the \sphinxcode{\sphinxupquote{calorimeter}} or no layers) 

\sphinxlineitem{param indxAbs}
\sphinxAtStartPar
\sphinxstylestrong{{[}inout{]}} pointer to an integer that can be anything at input, while at output it will be 0 or 1 that corresponds to the \sphinxcode{\sphinxupquote{absorber}} and \sphinxcode{\sphinxupquote{gap}} depending on which the given point was calculated to be located (\sphinxhyphen{}1 when the particle is leaving the \sphinxcode{\sphinxupquote{calorimeter}} or no layers) 

\end{description}\end{quote}
\begin{quote}\begin{description}
\sphinxlineitem{return}
\sphinxAtStartPar
the distance, from the given position along the given direction, to the boundary of the volume in which the given point was calculated to be located. It might be zero (the step actually shouldn’t be done in the located volume) or 1E+20 {[}mm{]} (the particle about leaving the \sphinxcode{\sphinxupquote{calorimeter}}). 

\end{description}\end{quote}


\end{fulllineitems}


\end{sphinxuseclass}
\begin{sphinxuseclass}{breathe-sectiondef}\subsubsection*{Private Functions}
\index{Geometry::UpdateParameters (C++ function)@\spxentry{Geometry::UpdateParameters}\spxextra{C++ function}}

\begin{fulllineitems}
\phantomsection\label{\detokenize{Simulation/SimulationCodeDoc:_CPPv4N8Geometry16UpdateParametersEv}}
\pysigstartsignatures
\pysigstartmultiline
\pysiglinewithargsret{\phantomsection\label{\detokenize{Simulation/SimulationCodeDoc:class_geometry_1a3b86bcc854f3a33e73fa52b5ac880a15}}\DUrole{kt,kt}{void}\DUrole{w,w}{  }\sphinxbfcode{\sphinxupquote{\DUrole{n,n}{UpdateParameters}}}}{}{}
\pysigstopmultiline
\pysigstopsignatures
\sphinxAtStartPar
Privite method that clculates the apropriate positions and volume/shape sizes whever any related parameters is updated. 

\end{fulllineitems}


\end{sphinxuseclass}
\begin{sphinxuseclass}{breathe-sectiondef}\subsubsection*{Private Members}
\index{Geometry::fNumLayers (C++ member)@\spxentry{Geometry::fNumLayers}\spxextra{C++ member}}

\begin{fulllineitems}
\phantomsection\label{\detokenize{Simulation/SimulationCodeDoc:_CPPv4N8Geometry10fNumLayersE}}
\pysigstartsignatures
\pysigstartmultiline
\pysigline{\phantomsection\label{\detokenize{Simulation/SimulationCodeDoc:class_geometry_1a5780a2301d1cc122f19d49743a96a411}}\DUrole{kt,kt}{int}\DUrole{w,w}{  }\sphinxbfcode{\sphinxupquote{\DUrole{n,n}{fNumLayers}}}}
\pysigstopmultiline
\pysigstopsignatures
\sphinxAtStartPar
Number of layers the calorimeter is built up. 

\sphinxAtStartPar
Can be set (even to zero: calorimeter is just a single volume/box) 

\end{fulllineitems}

\index{Geometry::fAbsThick (C++ member)@\spxentry{Geometry::fAbsThick}\spxextra{C++ member}}

\begin{fulllineitems}
\phantomsection\label{\detokenize{Simulation/SimulationCodeDoc:_CPPv4N8Geometry9fAbsThickE}}
\pysigstartsignatures
\pysigstartmultiline
\pysigline{\phantomsection\label{\detokenize{Simulation/SimulationCodeDoc:class_geometry_1a7939a17bb7abf32957f902e4143349ff}}\DUrole{kt,kt}{double}\DUrole{w,w}{  }\sphinxbfcode{\sphinxupquote{\DUrole{n,n}{fAbsThick}}}}
\pysigstopmultiline
\pysigstopsignatures
\sphinxAtStartPar
\sphinxcode{\sphinxupquote{Absorber}} thickness measured along the \sphinxcode{\sphinxupquote{x}}\sphinxhyphen{}axis in {[}mm{]} (can be set) 

\end{fulllineitems}

\index{Geometry::fGapThick (C++ member)@\spxentry{Geometry::fGapThick}\spxextra{C++ member}}

\begin{fulllineitems}
\phantomsection\label{\detokenize{Simulation/SimulationCodeDoc:_CPPv4N8Geometry9fGapThickE}}
\pysigstartsignatures
\pysigstartmultiline
\pysigline{\phantomsection\label{\detokenize{Simulation/SimulationCodeDoc:class_geometry_1afcaed22992d54b01abfba01dd09fa2d9}}\DUrole{kt,kt}{double}\DUrole{w,w}{  }\sphinxbfcode{\sphinxupquote{\DUrole{n,n}{fGapThick}}}}
\pysigstopmultiline
\pysigstopsignatures
\sphinxAtStartPar
\sphinxcode{\sphinxupquote{Gap}} thickness measured along the \sphinxcode{\sphinxupquote{x}}\sphinxhyphen{}axis in {[}mm{]} (can be set; even to zero: single material \sphinxcode{\sphinxupquote{calorimeter}} sliced by layers along the \sphinxcode{\sphinxupquote{x}}\sphinxhyphen{}axis) 

\end{fulllineitems}

\index{Geometry::fLayerThick (C++ member)@\spxentry{Geometry::fLayerThick}\spxextra{C++ member}}

\begin{fulllineitems}
\phantomsection\label{\detokenize{Simulation/SimulationCodeDoc:_CPPv4N8Geometry11fLayerThickE}}
\pysigstartsignatures
\pysigstartmultiline
\pysigline{\phantomsection\label{\detokenize{Simulation/SimulationCodeDoc:class_geometry_1a442075c15e051aefeba6648f89207688}}\DUrole{kt,kt}{double}\DUrole{w,w}{  }\sphinxbfcode{\sphinxupquote{\DUrole{n,n}{fLayerThick}}}}
\pysigstopmultiline
\pysigstopsignatures
\sphinxAtStartPar
\sphinxcode{\sphinxupquote{Layer}} thickness measured along the \sphinxcode{\sphinxupquote{x}} axes in {[}mm{]}. 

\sphinxAtStartPar
Computed automatically (whenever the \sphinxcode{\sphinxupquote{absorber}} or \sphinxcode{\sphinxupquote{gap}} thickness is updated 

\end{fulllineitems}

\index{Geometry::fCaloThick (C++ member)@\spxentry{Geometry::fCaloThick}\spxextra{C++ member}}

\begin{fulllineitems}
\phantomsection\label{\detokenize{Simulation/SimulationCodeDoc:_CPPv4N8Geometry10fCaloThickE}}
\pysigstartsignatures
\pysigstartmultiline
\pysigline{\phantomsection\label{\detokenize{Simulation/SimulationCodeDoc:class_geometry_1af96f94a6b86482ffa34e8a48ddfb416f}}\DUrole{kt,kt}{double}\DUrole{w,w}{  }\sphinxbfcode{\sphinxupquote{\DUrole{n,n}{fCaloThick}}}}
\pysigstopmultiline
\pysigstopsignatures
\sphinxAtStartPar
The thickness of the entire \sphinxcode{\sphinxupquote{calorimeter}} measured along the \sphinxcode{\sphinxupquote{x}} axes in {[}mm{]} Computed automatically whenever the \sphinxcode{\sphinxupquote{layer}} thickness (i.e. 

\sphinxAtStartPar
absorber and/or gap thickness) or number of layer is updated. 

\end{fulllineitems}

\index{Geometry::fCaloSizeYZ (C++ member)@\spxentry{Geometry::fCaloSizeYZ}\spxextra{C++ member}}

\begin{fulllineitems}
\phantomsection\label{\detokenize{Simulation/SimulationCodeDoc:_CPPv4N8Geometry11fCaloSizeYZE}}
\pysigstartsignatures
\pysigstartmultiline
\pysigline{\phantomsection\label{\detokenize{Simulation/SimulationCodeDoc:class_geometry_1a3a114ec53aed78cd5e04d08b4daa3127}}\DUrole{kt,kt}{double}\DUrole{w,w}{  }\sphinxbfcode{\sphinxupquote{\DUrole{n,n}{fCaloSizeYZ}}}}
\pysigstopmultiline
\pysigstopsignatures
\sphinxAtStartPar
The transverse size (i.e. 

\sphinxAtStartPar
full size along the \sphinxcode{\sphinxupquote{yz}} axes) of the \sphinxcode{\sphinxupquote{calorimeter}} in {[}mm{]} units (same for \sphinxcode{\sphinxupquote{asborber}}, \sphinxcode{\sphinxupquote{gap}} and \sphinxcode{\sphinxupquote{layer}} volumes/shapes) 

\end{fulllineitems}

\index{Geometry::fCaloStartX (C++ member)@\spxentry{Geometry::fCaloStartX}\spxextra{C++ member}}

\begin{fulllineitems}
\phantomsection\label{\detokenize{Simulation/SimulationCodeDoc:_CPPv4N8Geometry11fCaloStartXE}}
\pysigstartsignatures
\pysigstartmultiline
\pysigline{\phantomsection\label{\detokenize{Simulation/SimulationCodeDoc:class_geometry_1aea951dfb90df1521866737945ed706cf}}\DUrole{kt,kt}{double}\DUrole{w,w}{  }\sphinxbfcode{\sphinxupquote{\DUrole{n,n}{fCaloStartX}}}}
\pysigstopmultiline
\pysigstopsignatures
\sphinxAtStartPar
The \sphinxcode{\sphinxupquote{x}}\sphinxhyphen{}coordinate of the \sphinxcode{\sphinxupquote{calorimeter}} boundary on the left hand size. 

\sphinxAtStartPar
Calculated automatically (whenever the related parameters are updated) 

\end{fulllineitems}

\index{Geometry::fPrimaryXPosition (C++ member)@\spxentry{Geometry::fPrimaryXPosition}\spxextra{C++ member}}

\begin{fulllineitems}
\phantomsection\label{\detokenize{Simulation/SimulationCodeDoc:_CPPv4N8Geometry17fPrimaryXPositionE}}
\pysigstartsignatures
\pysigstartmultiline
\pysigline{\phantomsection\label{\detokenize{Simulation/SimulationCodeDoc:class_geometry_1a801f4fb769893443a3801e6b87da8fa6}}\DUrole{kt,kt}{double}\DUrole{w,w}{  }\sphinxbfcode{\sphinxupquote{\DUrole{n,n}{fPrimaryXPosition}}}}
\pysigstopmultiline
\pysigstopsignatures
\sphinxAtStartPar
The \sphinxcode{\sphinxupquote{x}}\sphinxhyphen{}coordinate of the mid\sphinxhyphen{}point between the \sphinxcode{\sphinxupquote{calorimeter}} and \sphinxcode{\sphinxupquote{world}} boundaries on the left hand size. 

\sphinxAtStartPar
Calculated automatically (whenever the related parameters are updated) 

\end{fulllineitems}

\index{Geometry::fBoxWorld (C++ member)@\spxentry{Geometry::fBoxWorld}\spxextra{C++ member}}

\begin{fulllineitems}
\phantomsection\label{\detokenize{Simulation/SimulationCodeDoc:_CPPv4N8Geometry9fBoxWorldE}}
\pysigstartsignatures
\pysigstartmultiline
\pysigline{\phantomsection\label{\detokenize{Simulation/SimulationCodeDoc:class_geometry_1a79a7642cf113cceb8ebff39f17a559c1}}{\hyperref[\detokenize{Simulation/SimulationCodeDoc:_CPPv43Box}]{\sphinxcrossref{\DUrole{n,n,n}{Box}}}}\DUrole{w,w}{  }\DUrole{p,p}{*}\sphinxbfcode{\sphinxupquote{\DUrole{n,n}{fBoxWorld}}}}
\pysigstopmultiline
\pysigstopsignatures
\sphinxAtStartPar
Pointer to the \sphinxcode{\sphinxupquote{{\hyperref[\detokenize{Simulation/SimulationCodeDoc:class_box}]{\sphinxcrossref{\DUrole{std,std-ref}{Box}}}}}} shape representing the \sphinxcode{\sphinxupquote{world}} volume. 

\end{fulllineitems}

\index{Geometry::fBoxCalo (C++ member)@\spxentry{Geometry::fBoxCalo}\spxextra{C++ member}}

\begin{fulllineitems}
\phantomsection\label{\detokenize{Simulation/SimulationCodeDoc:_CPPv4N8Geometry8fBoxCaloE}}
\pysigstartsignatures
\pysigstartmultiline
\pysigline{\phantomsection\label{\detokenize{Simulation/SimulationCodeDoc:class_geometry_1ae1927e8a7db773595b8bd9038ab06c72}}{\hyperref[\detokenize{Simulation/SimulationCodeDoc:_CPPv43Box}]{\sphinxcrossref{\DUrole{n,n,n}{Box}}}}\DUrole{w,w}{  }\DUrole{p,p}{*}\sphinxbfcode{\sphinxupquote{\DUrole{n,n}{fBoxCalo}}}}
\pysigstopmultiline
\pysigstopsignatures
\sphinxAtStartPar
Pointer to the \sphinxcode{\sphinxupquote{{\hyperref[\detokenize{Simulation/SimulationCodeDoc:class_box}]{\sphinxcrossref{\DUrole{std,std-ref}{Box}}}}}} shape representing the \sphinxcode{\sphinxupquote{calorimeter}} volume. 

\end{fulllineitems}

\index{Geometry::fBoxLayer (C++ member)@\spxentry{Geometry::fBoxLayer}\spxextra{C++ member}}

\begin{fulllineitems}
\phantomsection\label{\detokenize{Simulation/SimulationCodeDoc:_CPPv4N8Geometry9fBoxLayerE}}
\pysigstartsignatures
\pysigstartmultiline
\pysigline{\phantomsection\label{\detokenize{Simulation/SimulationCodeDoc:class_geometry_1af7ef69ba1fde966b851634fc3ab132bc}}{\hyperref[\detokenize{Simulation/SimulationCodeDoc:_CPPv43Box}]{\sphinxcrossref{\DUrole{n,n,n}{Box}}}}\DUrole{w,w}{  }\DUrole{p,p}{*}\sphinxbfcode{\sphinxupquote{\DUrole{n,n}{fBoxLayer}}}}
\pysigstopmultiline
\pysigstopsignatures
\sphinxAtStartPar
Pointer to the \sphinxcode{\sphinxupquote{{\hyperref[\detokenize{Simulation/SimulationCodeDoc:class_box}]{\sphinxcrossref{\DUrole{std,std-ref}{Box}}}}}} shape representing the \sphinxcode{\sphinxupquote{layer}} volume. 

\end{fulllineitems}

\index{Geometry::fBoxAbs (C++ member)@\spxentry{Geometry::fBoxAbs}\spxextra{C++ member}}

\begin{fulllineitems}
\phantomsection\label{\detokenize{Simulation/SimulationCodeDoc:_CPPv4N8Geometry7fBoxAbsE}}
\pysigstartsignatures
\pysigstartmultiline
\pysigline{\phantomsection\label{\detokenize{Simulation/SimulationCodeDoc:class_geometry_1aa094c984f2bc46d4edd4f68fa665826c}}{\hyperref[\detokenize{Simulation/SimulationCodeDoc:_CPPv43Box}]{\sphinxcrossref{\DUrole{n,n,n}{Box}}}}\DUrole{w,w}{  }\DUrole{p,p}{*}\sphinxbfcode{\sphinxupquote{\DUrole{n,n}{fBoxAbs}}}}
\pysigstopmultiline
\pysigstopsignatures
\sphinxAtStartPar
Pointer to the \sphinxcode{\sphinxupquote{{\hyperref[\detokenize{Simulation/SimulationCodeDoc:class_box}]{\sphinxcrossref{\DUrole{std,std-ref}{Box}}}}}} shape representing the \sphinxcode{\sphinxupquote{absorber}} volume. 

\end{fulllineitems}

\index{Geometry::fBoxGap (C++ member)@\spxentry{Geometry::fBoxGap}\spxextra{C++ member}}

\begin{fulllineitems}
\phantomsection\label{\detokenize{Simulation/SimulationCodeDoc:_CPPv4N8Geometry7fBoxGapE}}
\pysigstartsignatures
\pysigstartmultiline
\pysigline{\phantomsection\label{\detokenize{Simulation/SimulationCodeDoc:class_geometry_1a62b4059f63d26b91d8503707c7b2d860}}{\hyperref[\detokenize{Simulation/SimulationCodeDoc:_CPPv43Box}]{\sphinxcrossref{\DUrole{n,n,n}{Box}}}}\DUrole{w,w}{  }\DUrole{p,p}{*}\sphinxbfcode{\sphinxupquote{\DUrole{n,n}{fBoxGap}}}}
\pysigstopmultiline
\pysigstopsignatures
\sphinxAtStartPar
Pointer to the \sphinxcode{\sphinxupquote{{\hyperref[\detokenize{Simulation/SimulationCodeDoc:class_box}]{\sphinxcrossref{\DUrole{std,std-ref}{Box}}}}}} shape representing the \sphinxcode{\sphinxupquote{gap}} volume. 

\end{fulllineitems}


\end{sphinxuseclass}
\end{fulllineitems}

\index{Box (C++ class)@\spxentry{Box}\spxextra{C++ class}}

\begin{fulllineitems}
\phantomsection\label{\detokenize{Simulation/SimulationCodeDoc:_CPPv43Box}}
\pysigstartsignatures
\pysigstartmultiline
\pysigline{\phantomsection\label{\detokenize{Simulation/SimulationCodeDoc:class_box}}\DUrole{k,k}{class}\DUrole{w,w}{  }\sphinxbfcode{\sphinxupquote{\DUrole{n,n}{Box}}}}
\pysigstopmultiline
\pysigstopsignatures
\sphinxAtStartPar
A simplified version of the G4Box shape. 

\sphinxAtStartPar
\begin{description}
\sphinxlineitem{\sphinxstylestrong{Author}}
\sphinxAtStartPar
M. Novak 

\sphinxlineitem{\sphinxstylestrong{Date}}
\sphinxAtStartPar
July 2023

\end{description}


\sphinxAtStartPar
This is a simple version of the G4Box shape to describe geometry objects and use them in the simulation to calculate distance to their boundary from a point inside. Note, that the calculations include a tolerance, e.g. a point is on the surface if closer to a boundary than 1/2 tolerance (kCarTolerance). The two most important methods, used during this simplified simulation, are:

\sphinxAtStartPar
\begin{itemize}
\item {} 
\sphinxAtStartPar
DistanceToOut(position, direction): distance to boundary from a local position (inside the box) along the given direction. The boundary is ignored if the position is closer to it than 1/2 tolerance (i.e. point is on the surface). The distance to boundary is zero in this case whenever the direction is pointing outisde (i.e. the particle is moving away/out).

\item {} 
\sphinxAtStartPar
DistanceToOut(postition): this is the \sphinxcode{\sphinxupquote{safety}}, i.e. the distance to the nearest boundary from the given local point inside (zero if on the surface or outside).

\end{itemize}


\sphinxAtStartPar
This version of the {\hyperref[\detokenize{Simulation/SimulationCodeDoc:class_box}]{\sphinxcrossref{\DUrole{std,std-ref}{Box}}}} stores an index to the material that fills the shape (therefore closer to the Geant4 \sphinxcode{\sphinxupquote{logical}} volume concept than to a shape).

\sphinxAtStartPar
{\hyperref[\detokenize{Simulation/SimulationCodeDoc:class_box}]{\sphinxcrossref{\DUrole{std,std-ref}{Box}}}} shapes are constructed for each geometry object in the \sphinxcode{\sphinxupquote{{\hyperref[\detokenize{Simulation/SimulationCodeDoc:class_geometry}]{\sphinxcrossref{\DUrole{std,std-ref}{Geometry}}}}}} and the above methods are utilised during the simulation step computation in the \sphinxcode{\sphinxupquote{GammaStepper}} and \sphinxcode{\sphinxupquote{ElectronStepper}} (in some case indirectly by calling \sphinxcode{\sphinxupquote{Geometry::DistanceToOut}} that first locates the point, i.e. finds the {\hyperref[\detokenize{Simulation/SimulationCodeDoc:class_box}]{\sphinxcrossref{\DUrole{std,std-ref}{Box}}}} object which the given global point is located in).

\sphinxAtStartPar
NOTE: a point given in local coordiantes can locate\begin{itemize}
\item {} 
\sphinxAtStartPar
\sphinxcode{\sphinxupquote{inside}} : if deeper inside than \sphinxcode{\sphinxupquote{kCarTolerance/2}} from any boundary

\item {} 
\sphinxAtStartPar
\sphinxcode{\sphinxupquote{surface}} : if within \sphinxcode{\sphinxupquote{kCarTolerance/2}} from any boundary (in\sphinxhyphen{} or outside)

\item {} 
\sphinxAtStartPar
\sphinxcode{\sphinxupquote{outside}} : if further away than \sphinxcode{\sphinxupquote{kCarTolerance/2}} from any boundary outside

\end{itemize}


\sphinxAtStartPar
NOTE: distance to volume boundary from a point along a given direction is zero when the point is not \sphinxcode{\sphinxupquote{inside}} and the direction is pointing away. Therefore, a point located on the surface gives distance to boundary:\begin{itemize}
\item {} 
\sphinxAtStartPar
zero : if the direction is pointing outside of that boundary

\item {} 
\sphinxAtStartPar
non\sphinxhyphen{}zero: if the direction is pointing inside of that boundary 

\end{itemize}


\begin{sphinxuseclass}{breathe-sectiondef}\subsubsection*{Public Functions}
\index{Box::Box (C++ function)@\spxentry{Box::Box}\spxextra{C++ function}}

\begin{fulllineitems}
\phantomsection\label{\detokenize{Simulation/SimulationCodeDoc:_CPPv4N3Box3BoxERKNSt6stringEiddd}}
\pysigstartsignatures
\pysigstartmultiline
\pysiglinewithargsret{\phantomsection\label{\detokenize{Simulation/SimulationCodeDoc:class_box_1a7ac872372c33f5aecf06374c843dceca}}\sphinxbfcode{\sphinxupquote{\DUrole{n,n}{Box}}}}{\DUrole{k,k}{const}\DUrole{w,w}{  }\DUrole{n,n,n}{std}\DUrole{p,p}{::}\DUrole{n,n,n}{string}\DUrole{w,w}{  }\DUrole{p,p}{\&}\DUrole{n,sig-param,n}{name}\sphinxparamcomma \DUrole{kt,kt}{int}\DUrole{w,w}{  }\DUrole{n,sig-param,n}{indxMat}\sphinxparamcomma \DUrole{kt,kt}{double}\DUrole{w,w}{  }\DUrole{n,sig-param,n}{pX}\sphinxparamcomma \DUrole{kt,kt}{double}\DUrole{w,w}{  }\DUrole{n,sig-param,n}{pY}\sphinxparamcomma \DUrole{kt,kt}{double}\DUrole{w,w}{  }\DUrole{n,sig-param,n}{pZ}}{}
\pysigstopmultiline
\pysigstopsignatures
\sphinxAtStartPar
Constructor. 

\sphinxAtStartPar
\begin{quote}\begin{description}
\sphinxlineitem{param name}
\sphinxAtStartPar
\sphinxstylestrong{{[}in{]}} Name of this volume. 

\sphinxlineitem{param indxMat}
\sphinxAtStartPar
\sphinxstylestrong{{[}in{]}} Index of the material this volume is filled with. 

\sphinxlineitem{param pX}
\sphinxAtStartPar
\sphinxstylestrong{{[}in{]}} Half length of the box along the x\sphinxhyphen{}axis. 

\sphinxlineitem{param pY}
\sphinxAtStartPar
\sphinxstylestrong{{[}in{]}} Half length of the box along the y\sphinxhyphen{}axis. 

\sphinxlineitem{param pZ}
\sphinxAtStartPar
\sphinxstylestrong{{[}in{]}} Half length of the box along the z\sphinxhyphen{}axis. 

\end{description}\end{quote}


\end{fulllineitems}

\index{Box::\textasciitilde{}Box (C++ function)@\spxentry{Box::\textasciitilde{}Box}\spxextra{C++ function}}

\begin{fulllineitems}
\phantomsection\label{\detokenize{Simulation/SimulationCodeDoc:_CPPv4N3BoxD0Ev}}
\pysigstartsignatures
\pysigstartmultiline
\pysiglinewithargsret{\phantomsection\label{\detokenize{Simulation/SimulationCodeDoc:class_box_1a6a5e09398e85d602a046b429062fb9c2}}\DUrole{k,k}{inline}\DUrole{w,w}{  }\sphinxbfcode{\sphinxupquote{\DUrole{n,n}{\textasciitilde{}Box}}}}{}{}
\pysigstopmultiline
\pysigstopsignatures
\sphinxAtStartPar
Destructor (nothig to do) 

\end{fulllineitems}

\index{Box::GetName (C++ function)@\spxentry{Box::GetName}\spxextra{C++ function}}

\begin{fulllineitems}
\phantomsection\label{\detokenize{Simulation/SimulationCodeDoc:_CPPv4NK3Box7GetNameEv}}
\pysigstartsignatures
\pysigstartmultiline
\pysiglinewithargsret{\phantomsection\label{\detokenize{Simulation/SimulationCodeDoc:class_box_1ae360d3e3aa5921e52596d0ff36535913}}\DUrole{k,k}{inline}\DUrole{w,w}{  }\DUrole{k,k}{const}\DUrole{w,w}{  }\DUrole{n,n,n}{std}\DUrole{p,p}{::}\DUrole{n,n,n}{string}\DUrole{w,w}{  }\DUrole{p,p}{\&}\sphinxbfcode{\sphinxupquote{\DUrole{n,n}{GetName}}}}{}{\DUrole{w,w}{  }\DUrole{k,k}{const}}
\pysigstopmultiline
\pysigstopsignatures
\sphinxAtStartPar
Get the name of this volume. 

\end{fulllineitems}

\index{Box::SetMaterialIndx (C++ function)@\spxentry{Box::SetMaterialIndx}\spxextra{C++ function}}

\begin{fulllineitems}
\phantomsection\label{\detokenize{Simulation/SimulationCodeDoc:_CPPv4N3Box15SetMaterialIndxEi}}
\pysigstartsignatures
\pysigstartmultiline
\pysiglinewithargsret{\phantomsection\label{\detokenize{Simulation/SimulationCodeDoc:class_box_1ab2e8383851eeaa396268241399073081}}\DUrole{k,k}{inline}\DUrole{w,w}{  }\DUrole{kt,kt}{void}\DUrole{w,w}{  }\sphinxbfcode{\sphinxupquote{\DUrole{n,n}{SetMaterialIndx}}}}{\DUrole{kt,kt}{int}\DUrole{w,w}{  }\DUrole{n,sig-param,n}{indx}}{}
\pysigstopmultiline
\pysigstopsignatures
\sphinxAtStartPar
Set the material this volume is filled with. 

\sphinxAtStartPar
\begin{quote}\begin{description}
\sphinxlineitem{param indx}
\sphinxAtStartPar
\sphinxstylestrong{{[}in{]}} Index of the material. 

\end{description}\end{quote}


\end{fulllineitems}

\index{Box::GetMaterialIndx (C++ function)@\spxentry{Box::GetMaterialIndx}\spxextra{C++ function}}

\begin{fulllineitems}
\phantomsection\label{\detokenize{Simulation/SimulationCodeDoc:_CPPv4NK3Box15GetMaterialIndxEv}}
\pysigstartsignatures
\pysigstartmultiline
\pysiglinewithargsret{\phantomsection\label{\detokenize{Simulation/SimulationCodeDoc:class_box_1a0ae7ae21a5e8b6fab545ccb5d67b9763}}\DUrole{k,k}{inline}\DUrole{w,w}{  }\DUrole{kt,kt}{int}\DUrole{w,w}{  }\sphinxbfcode{\sphinxupquote{\DUrole{n,n}{GetMaterialIndx}}}}{}{\DUrole{w,w}{  }\DUrole{k,k}{const}}
\pysigstopmultiline
\pysigstopsignatures
\sphinxAtStartPar
Get the material this volume is filled with. 

\sphinxAtStartPar
\begin{quote}\begin{description}
\sphinxlineitem{return}
\sphinxAtStartPar
Index of the material. 

\end{description}\end{quote}


\end{fulllineitems}

\index{Box::SetHalfLength (C++ function)@\spxentry{Box::SetHalfLength}\spxextra{C++ function}}

\begin{fulllineitems}
\phantomsection\label{\detokenize{Simulation/SimulationCodeDoc:_CPPv4N3Box13SetHalfLengthEdi}}
\pysigstartsignatures
\pysigstartmultiline
\pysiglinewithargsret{\phantomsection\label{\detokenize{Simulation/SimulationCodeDoc:class_box_1ab6cccf2b74be881f887f118eb1cdffaf}}\DUrole{kt,kt}{void}\DUrole{w,w}{  }\sphinxbfcode{\sphinxupquote{\DUrole{n,n}{SetHalfLength}}}}{\DUrole{kt,kt}{double}\DUrole{w,w}{  }\DUrole{n,sig-param,n}{val}\sphinxparamcomma \DUrole{kt,kt}{int}\DUrole{w,w}{  }\DUrole{n,sig-param,n}{idx}}{}
\pysigstopmultiline
\pysigstopsignatures
\sphinxAtStartPar
Set the half length of the box along the given axis. 

\sphinxAtStartPar
\begin{quote}\begin{description}
\sphinxlineitem{param val}
\sphinxAtStartPar
\sphinxstylestrong{{[}in{]}} Half length in {[}mm{]} units. 

\sphinxlineitem{param idx}
\sphinxAtStartPar
\sphinxstylestrong{{[}in{]}} Encodes the axis along the half length is given (idx=0 \textendash{}\textgreater{} x; idx=1 \textendash{}\textgreater{} y; idx=2 \textendash{}\textgreater{} z). 

\end{description}\end{quote}


\end{fulllineitems}

\index{Box::GetHalfLength (C++ function)@\spxentry{Box::GetHalfLength}\spxextra{C++ function}}

\begin{fulllineitems}
\phantomsection\label{\detokenize{Simulation/SimulationCodeDoc:_CPPv4NK3Box13GetHalfLengthEi}}
\pysigstartsignatures
\pysigstartmultiline
\pysiglinewithargsret{\phantomsection\label{\detokenize{Simulation/SimulationCodeDoc:class_box_1a57bcebb25b0e07f573225bbc1bd4b1d1}}\DUrole{kt,kt}{double}\DUrole{w,w}{  }\sphinxbfcode{\sphinxupquote{\DUrole{n,n}{GetHalfLength}}}}{\DUrole{kt,kt}{int}\DUrole{w,w}{  }\DUrole{n,sig-param,n}{idx}}{\DUrole{w,w}{  }\DUrole{k,k}{const}}
\pysigstopmultiline
\pysigstopsignatures
\sphinxAtStartPar
Get the half length of the box along the given axis. 

\sphinxAtStartPar
\begin{quote}\begin{description}
\sphinxlineitem{param idx}
\sphinxAtStartPar
\sphinxstylestrong{{[}in{]}} Encodes the axis along the half length is required (idx=0 \textendash{}\textgreater{} x; idx=1 \textendash{}\textgreater{} y; idx=2 \textendash{}\textgreater{} z). 

\end{description}\end{quote}
\begin{quote}\begin{description}
\sphinxlineitem{return}
\sphinxAtStartPar
Half length of this box along the required axis. 

\end{description}\end{quote}


\end{fulllineitems}

\index{Box::DistanceToOut (C++ function)@\spxentry{Box::DistanceToOut}\spxextra{C++ function}}

\begin{fulllineitems}
\phantomsection\label{\detokenize{Simulation/SimulationCodeDoc:_CPPv4NK3Box13DistanceToOutEPdPd}}
\pysigstartsignatures
\pysigstartmultiline
\pysiglinewithargsret{\phantomsection\label{\detokenize{Simulation/SimulationCodeDoc:class_box_1aad17ace7ec8e5b684b09833a3d35a2bc}}\DUrole{kt,kt}{double}\DUrole{w,w}{  }\sphinxbfcode{\sphinxupquote{\DUrole{n,n}{DistanceToOut}}}}{\DUrole{kt,kt}{double}\DUrole{w,w}{  }\DUrole{p,p}{*}\DUrole{n,sig-param,n}{r}\sphinxparamcomma \DUrole{kt,kt}{double}\DUrole{w,w}{  }\DUrole{p,p}{*}\DUrole{n,sig-param,n}{v}}{\DUrole{w,w}{  }\DUrole{k,k}{const}}
\pysigstopmultiline
\pysigstopsignatures
\sphinxAtStartPar
Calculates distance to the volume boundary from inside along the given direction. 

\sphinxAtStartPar
Returns the distance along the normalised direction vector \sphinxcode{\sphinxupquote{v}} to the volume boundary, from the given point \sphinxcode{\sphinxupquote{p}} inside or on the surface of the box. Intersections with surfaces, when the point is within half tolerance (\sphinxcode{\sphinxupquote{kCarTolerance/2}}) from a surface, is ignored.

\sphinxAtStartPar
\begin{quote}\begin{description}
\sphinxlineitem{param r}
\sphinxAtStartPar
\sphinxstylestrong{{[}in{]}} 3D position of the point in local coordinates 

\sphinxlineitem{param v}
\sphinxAtStartPar
\sphinxstylestrong{{[}in{]}} 3D normalised direction 

\end{description}\end{quote}
\begin{quote}\begin{description}
\sphinxlineitem{return}
\sphinxAtStartPar
Distance to the surface boundary from inside (see above). 

\end{description}\end{quote}


\end{fulllineitems}

\index{Box::DistanceToOut (C++ function)@\spxentry{Box::DistanceToOut}\spxextra{C++ function}}

\begin{fulllineitems}
\phantomsection\label{\detokenize{Simulation/SimulationCodeDoc:_CPPv4NK3Box13DistanceToOutEPd}}
\pysigstartsignatures
\pysigstartmultiline
\pysiglinewithargsret{\phantomsection\label{\detokenize{Simulation/SimulationCodeDoc:class_box_1a29345325f9e40e34dc345b51a48004dd}}\DUrole{kt,kt}{double}\DUrole{w,w}{  }\sphinxbfcode{\sphinxupquote{\DUrole{n,n}{DistanceToOut}}}}{\DUrole{kt,kt}{double}\DUrole{w,w}{  }\DUrole{p,p}{*}\DUrole{n,sig-param,n}{r}}{\DUrole{w,w}{  }\DUrole{k,k}{const}}
\pysigstopmultiline
\pysigstopsignatures
\sphinxAtStartPar
Calculates the distance to the nearest boundary of a shape from inside (safety). 

\sphinxAtStartPar
While the above considers the direction, this finds the nearest boundary.

\sphinxAtStartPar
\begin{quote}\begin{description}
\sphinxlineitem{param r}
\sphinxAtStartPar
\sphinxstylestrong{{[}in{]}} 3D position of the point in local coordinates 

\end{description}\end{quote}
\begin{quote}\begin{description}
\sphinxlineitem{return}
\sphinxAtStartPar
Distance to the nearest surface boundary from inside (see above). 

\end{description}\end{quote}


\end{fulllineitems}


\end{sphinxuseclass}
\begin{sphinxuseclass}{breathe-sectiondef}\subsubsection*{Private Members}
\index{Box::fName (C++ member)@\spxentry{Box::fName}\spxextra{C++ member}}

\begin{fulllineitems}
\phantomsection\label{\detokenize{Simulation/SimulationCodeDoc:_CPPv4N3Box5fNameE}}
\pysigstartsignatures
\pysigstartmultiline
\pysigline{\phantomsection\label{\detokenize{Simulation/SimulationCodeDoc:class_box_1a2c24355779e7b714cab4581547ea9580}}\DUrole{k,k}{const}\DUrole{w,w}{  }\DUrole{n,n,n}{std}\DUrole{p,p}{::}\DUrole{n,n,n}{string}\DUrole{w,w}{  }\sphinxbfcode{\sphinxupquote{\DUrole{n,n}{fName}}}}
\pysigstopmultiline
\pysigstopsignatures
\sphinxAtStartPar
Name of this volume. 

\end{fulllineitems}

\index{Box::fMaterialIndx (C++ member)@\spxentry{Box::fMaterialIndx}\spxextra{C++ member}}

\begin{fulllineitems}
\phantomsection\label{\detokenize{Simulation/SimulationCodeDoc:_CPPv4N3Box13fMaterialIndxE}}
\pysigstartsignatures
\pysigstartmultiline
\pysigline{\phantomsection\label{\detokenize{Simulation/SimulationCodeDoc:class_box_1ae9dbbb9d4a4143cbe2febc01af577f53}}\DUrole{kt,kt}{int}\DUrole{w,w}{  }\sphinxbfcode{\sphinxupquote{\DUrole{n,n}{fMaterialIndx}}}}
\pysigstopmultiline
\pysigstopsignatures
\sphinxAtStartPar
Index of the material this volume is filled with. 

\end{fulllineitems}

\index{Box::fDx (C++ member)@\spxentry{Box::fDx}\spxextra{C++ member}}

\begin{fulllineitems}
\phantomsection\label{\detokenize{Simulation/SimulationCodeDoc:_CPPv4N3Box3fDxE}}
\pysigstartsignatures
\pysigstartmultiline
\pysigline{\phantomsection\label{\detokenize{Simulation/SimulationCodeDoc:class_box_1a456992b909ed45dd99e1f1b9f246994c}}\DUrole{kt,kt}{double}\DUrole{w,w}{  }\sphinxbfcode{\sphinxupquote{\DUrole{n,n}{fDx}}}}
\pysigstopmultiline
\pysigstopsignatures
\sphinxAtStartPar
Half length of the box along the x\sphinxhyphen{}axis. 

\end{fulllineitems}

\index{Box::fDy (C++ member)@\spxentry{Box::fDy}\spxextra{C++ member}}

\begin{fulllineitems}
\phantomsection\label{\detokenize{Simulation/SimulationCodeDoc:_CPPv4N3Box3fDyE}}
\pysigstartsignatures
\pysigstartmultiline
\pysigline{\phantomsection\label{\detokenize{Simulation/SimulationCodeDoc:class_box_1a341d1fb2d8102035698dfa6a4de4a557}}\DUrole{kt,kt}{double}\DUrole{w,w}{  }\sphinxbfcode{\sphinxupquote{\DUrole{n,n}{fDy}}}}
\pysigstopmultiline
\pysigstopsignatures
\sphinxAtStartPar
Half length of the box along the y\sphinxhyphen{}axis. 

\end{fulllineitems}

\index{Box::fDz (C++ member)@\spxentry{Box::fDz}\spxextra{C++ member}}

\begin{fulllineitems}
\phantomsection\label{\detokenize{Simulation/SimulationCodeDoc:_CPPv4N3Box3fDzE}}
\pysigstartsignatures
\pysigstartmultiline
\pysigline{\phantomsection\label{\detokenize{Simulation/SimulationCodeDoc:class_box_1a5e868b153e22d588dd482e45fc69ca7d}}\DUrole{kt,kt}{double}\DUrole{w,w}{  }\sphinxbfcode{\sphinxupquote{\DUrole{n,n}{fDz}}}}
\pysigstopmultiline
\pysigstopsignatures
\sphinxAtStartPar
Half length of the box along the z\sphinxhyphen{}axis. 

\end{fulllineitems}

\index{Box::kCarTolerance (C++ member)@\spxentry{Box::kCarTolerance}\spxextra{C++ member}}

\begin{fulllineitems}
\phantomsection\label{\detokenize{Simulation/SimulationCodeDoc:_CPPv4N3Box13kCarToleranceE}}
\pysigstartsignatures
\pysigstartmultiline
\pysigline{\phantomsection\label{\detokenize{Simulation/SimulationCodeDoc:class_box_1a92e002367dac759cdb49991eccb80869}}\DUrole{k,k}{const}\DUrole{w,w}{  }\DUrole{kt,kt}{double}\DUrole{w,w}{  }\sphinxbfcode{\sphinxupquote{\DUrole{n,n}{kCarTolerance}}}\DUrole{w,w}{  }\DUrole{p,p}{=}\DUrole{w,w}{  }\DUrole{m,m}{1.0E\sphinxhyphen{}9}}
\pysigstopmultiline
\pysigstopsignatures
\sphinxAtStartPar
Value of the tolerance in {[}mm{]}. 

\end{fulllineitems}

\index{Box::fDelta (C++ member)@\spxentry{Box::fDelta}\spxextra{C++ member}}

\begin{fulllineitems}
\phantomsection\label{\detokenize{Simulation/SimulationCodeDoc:_CPPv4N3Box6fDeltaE}}
\pysigstartsignatures
\pysigstartmultiline
\pysigline{\phantomsection\label{\detokenize{Simulation/SimulationCodeDoc:class_box_1a167d1f1fa5733a0ad9ffed8b492a30c9}}\DUrole{kt,kt}{double}\DUrole{w,w}{  }\sphinxbfcode{\sphinxupquote{\DUrole{n,n}{fDelta}}}}
\pysigstopmultiline
\pysigstopsignatures
\sphinxAtStartPar
Half of the above tolerance. 

\end{fulllineitems}


\end{sphinxuseclass}
\end{fulllineitems}



\subsection{The \sphinxstyleliteralintitle{\sphinxupquote{Physics}} code documentation}
\label{\detokenize{Simulation/SimulationCodeDoc:the-physics-code-documentation}}\index{Physics (C++ class)@\spxentry{Physics}\spxextra{C++ class}}

\begin{fulllineitems}
\phantomsection\label{\detokenize{Simulation/SimulationCodeDoc:_CPPv47Physics}}
\pysigstartsignatures
\pysigstartmultiline
\pysigline{\phantomsection\label{\detokenize{Simulation/SimulationCodeDoc:class_physics}}\DUrole{k,k}{class}\DUrole{w,w}{  }\sphinxbfcode{\sphinxupquote{\DUrole{n,n}{Physics}}}}
\pysigstopmultiline
\pysigstopsignatures
\sphinxAtStartPar
The entire physics of the simulation is provided by \sphinxcode{\sphinxupquote{G4HepEm}} {[}1{]} and pulled\sphinxhyphen{}in to the \sphinxcode{\sphinxupquote{HepEmShow}} application by the \sphinxcode{\sphinxupquote{Physics.hh}} and \sphinxcode{\sphinxupquote{Physics.cc}} files. 

\sphinxAtStartPar
\begin{description}
\sphinxlineitem{\sphinxstylestrong{Author}}
\sphinxAtStartPar
M. Novak 

\sphinxlineitem{\sphinxstylestrong{Date}}
\sphinxAtStartPar
July 2023

\end{description}


\sphinxAtStartPar
The \sphinxcode{\sphinxupquote{Physics.hh}} header file includes the \sphinxcode{\sphinxupquote{G4HepEmRun}} headers that give the complete set of run\sphinxhyphen{}time functioinalities required for the EM physics modelling. The corresponding implementations are pulled\sphinxhyphen{}in all together in the\sphinxcode{\sphinxupquote{Physics.cc}} implentation file.

\sphinxAtStartPar
The only ingredient of \sphinxcode{\sphinxupquote{G4HepEmRun}}, that a client application (such as \sphinxcode{\sphinxupquote{HepEmShow}}), needs to provide is an implementation of a uniform random number generator. An object from such genertor must be plugged\sphinxhyphen{}in to the \sphinxcode{\sphinxupquote{G4HepEmRandomEngine}} by implementing the two missing \sphinxcode{\sphinxupquote{G4HepEmRandomEngine::flat()}} and \sphinxcode{\sphinxupquote{G4HepEmRandomEngine::flatArray(const int, double *)}} methods. This is also done in the \sphinxcode{\sphinxupquote{Physics.cc}} implementation file that completes the implementation of \sphinxcode{\sphinxupquote{G4HepEmRun}}.

\sphinxAtStartPar
\sphinxcode{\sphinxupquote{{\hyperref[\detokenize{Simulation/SimulationCodeDoc:class_u_random}]{\sphinxcrossref{\DUrole{std,std-ref}{URandom}}}}}} is the uniform random number generator implemented in \sphinxcode{\sphinxupquote{HepEmShow}} based on the 64\sphinxhyphen{}bit verson of the Mersenne Twister generator provided by c++11. An obejct of this is utilised in the \sphinxcode{\sphinxupquote{Physics.cc}} file to complete the implementation of the \sphinxcode{\sphinxupquote{G4HepEmRandomEngine}} as mentioned above. Then the actual uniform random number generator and the random engine objects are constructed (and set to the \sphinxcode{\sphinxupquote{G4HepEmTLData}} obejct) in the \sphinxcode{\sphinxupquote{HepEmShow.cc}} main function of the application.

\sphinxAtStartPar
\sphinxcode{\sphinxupquote{G4HepEm}} implements two top level methods, \sphinxcode{\sphinxupquote{HowFar}} and \sphinxcode{\sphinxupquote{Perform}}, in its \sphinxcode{\sphinxupquote{G4HepEmGammaManager}} and \sphinxcode{\sphinxupquote{G4HepEmElectronManager}}:\begin{itemize}
\item {} 
\sphinxAtStartPar
to provide the information on \sphinxcode{\sphinxupquote{HowFar}} a given input \(\gamma\) or \(e^-/e^+\) track goes according to their physics related constraints (e.g. till their next physics interaction takes place or other physics related constraints).

\item {} 
\sphinxAtStartPar
to \sphinxcode{\sphinxupquote{Perform}} all necessary physics related updates on the given input \(\gamma\) or \(e^-/e^+\) track, including the production of secondary tracks in the given physics interaction (if any).

\end{itemize}


\sphinxAtStartPar
The first (\sphinxcode{\sphinxupquote{HowFar}}) is invoked at the pre\sphinxhyphen{}step point while the second (\sphinxcode{\sphinxupquote{Perform}}) is at the post\sphinxhyphen{}step point of each individual simulation step computation inside the \sphinxcode{\sphinxupquote{{\hyperref[\detokenize{Simulation/SimulationCodeDoc:class_stepping_loop_1a1d7aa9c14da7b327829dc318d13fc2ed}]{\sphinxcrossref{\DUrole{std,std-ref}{SteppingLoop::GammaStepper()}}}}}} and \sphinxcode{\sphinxupquote{{\hyperref[\detokenize{Simulation/SimulationCodeDoc:class_stepping_loop_1ad0de81b62ac3ba13532464c2c01f1b39}]{\sphinxcrossref{\DUrole{std,std-ref}{SteppingLoop::ElectronStepper()}}}}}}. 

\end{fulllineitems}

\index{URandom (C++ class)@\spxentry{URandom}\spxextra{C++ class}}

\begin{fulllineitems}
\phantomsection\label{\detokenize{Simulation/SimulationCodeDoc:_CPPv47URandom}}
\pysigstartsignatures
\pysigstartmultiline
\pysigline{\phantomsection\label{\detokenize{Simulation/SimulationCodeDoc:class_u_random}}\DUrole{k,k}{class}\DUrole{w,w}{  }\sphinxbfcode{\sphinxupquote{\DUrole{n,n}{URandom}}}}
\pysigstopmultiline
\pysigstopsignatures
\sphinxAtStartPar
A uniform random number generator. 

\sphinxAtStartPar
\begin{description}
\sphinxlineitem{\sphinxstylestrong{Author}}
\sphinxAtStartPar
M. Novak 

\sphinxlineitem{\sphinxstylestrong{Date}}
\sphinxAtStartPar
July 2023

\end{description}


\sphinxAtStartPar
This is the uniform random number generator, i.e. the only thing that is need to make the \sphinxcode{\sphinxupquote{G4HepEm}} physics implementation complete (see more at the \sphinxcode{\sphinxupquote{{\hyperref[\detokenize{Simulation/SimulationCodeDoc:class_physics}]{\sphinxcrossref{\DUrole{std,std-ref}{Physics}}}}}} documentation). This random number generator relies on the c++11 implementation of the 64\sphinxhyphen{}bit Mersenne Twister engine. The \sphinxcode{\sphinxupquote{{\hyperref[\detokenize{Simulation/SimulationCodeDoc:class_u_random_1a869185702aeec4f37a4f788dcd5ea838}]{\sphinxcrossref{\DUrole{std,std-ref}{URandom::flat()}}}}}} method can be used to provide uniform random numbers on the \((0,1)\). An object from this class is constructed in the \sphinxcode{\sphinxupquote{HepEmShow}} main and set to be used in the \sphinxcode{\sphinxupquote{G4HepEmRandomEngine}}.

\sphinxAtStartPar

\begin{sphinxadmonition}{note}{Note:}
\sphinxAtStartPar
This random number generator can be replaced with anything that can provide uniform fandom numbers on \((0,1)\). One need to modify the corresponding implementations in \sphinxcode{\sphinxupquote{{\hyperref[\detokenize{Simulation/SimulationCodeDoc:class_physics}]{\sphinxcrossref{\DUrole{std,std-ref}{Physics}}}}}} (namely, one line in the \sphinxcode{\sphinxupquote{G4HepEmRandomEngine::flat()}} and \sphinxcode{\sphinxupquote{G4HepEmRandomEngine::flatArray()}} implementations in \sphinxcode{\sphinxupquote{Physics.cc}}) and replace the \sphinxcode{\sphinxupquote{{\hyperref[\detokenize{Simulation/SimulationCodeDoc:class_u_random}]{\sphinxcrossref{\DUrole{std,std-ref}{URandom}}}}}} object construction in the \sphinxcode{\sphinxupquote{HepEmShow}} main. 
\end{sphinxadmonition}


\begin{sphinxuseclass}{breathe-sectiondef}\subsubsection*{Public Functions}
\index{URandom::URandom (C++ function)@\spxentry{URandom::URandom}\spxextra{C++ function}}

\begin{fulllineitems}
\phantomsection\label{\detokenize{Simulation/SimulationCodeDoc:_CPPv4N7URandom7URandomEi}}
\pysigstartsignatures
\pysigstartmultiline
\pysiglinewithargsret{\phantomsection\label{\detokenize{Simulation/SimulationCodeDoc:class_u_random_1a4a68d6fbc770bb3b69b84eb9a1698d94}}\sphinxbfcode{\sphinxupquote{\DUrole{n,n}{URandom}}}}{\DUrole{kt,kt}{int}\DUrole{w,w}{  }\DUrole{n,sig-param,n}{seed}\DUrole{w,w}{  }\DUrole{p,p}{=}\DUrole{w,w}{  }\DUrole{m,m}{123}}{}
\pysigstopmultiline
\pysigstopsignatures
\sphinxAtStartPar
CTR. 

\sphinxAtStartPar
\begin{quote}\begin{description}
\sphinxlineitem{param seed}
\sphinxAtStartPar
seed of the random number generator. 

\end{description}\end{quote}


\end{fulllineitems}

\index{URandom::\textasciitilde{}URandom (C++ function)@\spxentry{URandom::\textasciitilde{}URandom}\spxextra{C++ function}}

\begin{fulllineitems}
\phantomsection\label{\detokenize{Simulation/SimulationCodeDoc:_CPPv4N7URandomD0Ev}}
\pysigstartsignatures
\pysigstartmultiline
\pysiglinewithargsret{\phantomsection\label{\detokenize{Simulation/SimulationCodeDoc:class_u_random_1a31e88fa4b0610950a82d87c535e3d33f}}\sphinxbfcode{\sphinxupquote{\DUrole{n,n}{\textasciitilde{}URandom}}}}{}{}
\pysigstopmultiline
\pysigstopsignatures
\sphinxAtStartPar
DTR. 

\end{fulllineitems}

\index{URandom::flat (C++ function)@\spxentry{URandom::flat}\spxextra{C++ function}}

\begin{fulllineitems}
\phantomsection\label{\detokenize{Simulation/SimulationCodeDoc:_CPPv4N7URandom4flatEv}}
\pysigstartsignatures
\pysigstartmultiline
\pysiglinewithargsret{\phantomsection\label{\detokenize{Simulation/SimulationCodeDoc:class_u_random_1a869185702aeec4f37a4f788dcd5ea838}}\DUrole{kt,kt}{double}\DUrole{w,w}{  }\sphinxbfcode{\sphinxupquote{\DUrole{n,n}{flat}}}}{}{}
\pysigstopmultiline
\pysigstopsignatures
\sphinxAtStartPar
Method to provide uniform random numbers on \((0,1)\). 

\end{fulllineitems}


\end{sphinxuseclass}
\begin{sphinxuseclass}{breathe-sectiondef}\subsubsection*{Public Members}
\index{URandom::fEngine (C++ member)@\spxentry{URandom::fEngine}\spxextra{C++ member}}

\begin{fulllineitems}
\phantomsection\label{\detokenize{Simulation/SimulationCodeDoc:_CPPv4N7URandom7fEngineE}}
\pysigstartsignatures
\pysigstartmultiline
\pysigline{\phantomsection\label{\detokenize{Simulation/SimulationCodeDoc:class_u_random_1a091edcd5c38bb2b2a4c0960d4f56c793}}\DUrole{n,n,n}{std}\DUrole{p,p}{::}\DUrole{n,n,n}{mt19937\_64}\DUrole{w,w}{  }\sphinxbfcode{\sphinxupquote{\DUrole{n,n}{fEngine}}}}
\pysigstopmultiline
\pysigstopsignatures
\sphinxAtStartPar
c++11 implementation of the 64\sphinxhyphen{}bit Mersenne Twister engine 

\end{fulllineitems}

\index{URandom::fDist (C++ member)@\spxentry{URandom::fDist}\spxextra{C++ member}}

\begin{fulllineitems}
\phantomsection\label{\detokenize{Simulation/SimulationCodeDoc:_CPPv4N7URandom5fDistE}}
\pysigstartsignatures
\pysigstartmultiline
\pysigline{\phantomsection\label{\detokenize{Simulation/SimulationCodeDoc:class_u_random_1aa6c951503d67b9be7b2e4fd6de4415b2}}\DUrole{n,n,n}{std}\DUrole{p,p}{::}\DUrole{n,n,n}{uniform\_real\_distribution}\DUrole{p,p}{\textless{}}\DUrole{kt,kt}{double}\DUrole{p,p}{\textgreater{}}\DUrole{w,w}{  }\DUrole{p,p}{*}\sphinxbfcode{\sphinxupquote{\DUrole{n,n}{fDist}}}}
\pysigstopmultiline
\pysigstopsignatures
\sphinxAtStartPar
uniform distribution: utilises the above random engine to provide random numbers on \((0,1)\)

\end{fulllineitems}


\end{sphinxuseclass}
\end{fulllineitems}



\subsection{The \sphinxstyleliteralintitle{\sphinxupquote{PrimaryGenerator}} code documentation}
\label{\detokenize{Simulation/SimulationCodeDoc:the-primarygenerator-code-documentation}}\index{PrimaryGenerator (C++ class)@\spxentry{PrimaryGenerator}\spxextra{C++ class}}

\begin{fulllineitems}
\phantomsection\label{\detokenize{Simulation/SimulationCodeDoc:_CPPv416PrimaryGenerator}}
\pysigstartsignatures
\pysigstartmultiline
\pysigline{\phantomsection\label{\detokenize{Simulation/SimulationCodeDoc:class_primary_generator}}\DUrole{k,k}{class}\DUrole{w,w}{  }\sphinxbfcode{\sphinxupquote{\DUrole{n,n}{PrimaryGenerator}}}}
\pysigstopmultiline
\pysigstopsignatures
\sphinxAtStartPar
Generates primary particles for an event. 

\sphinxAtStartPar
\begin{description}
\sphinxlineitem{\sphinxstylestrong{Author}}
\sphinxAtStartPar
M. Novak 

\sphinxlineitem{\sphinxstylestrong{Date}}
\sphinxAtStartPar
July 2023

\end{description}


\sphinxAtStartPar
This is a simple primary particle generator. The kinetic energy, position, direction and the particle type (through its charge) can be configured. Note, that we simulate only \(e^-/e^+\) and \(\gamma\) particles with \sphinxhyphen{}1, +1 and 0 charge respectively.

\sphinxAtStartPar
The \sphinxcode{\sphinxupquote{{\hyperref[\detokenize{Simulation/SimulationCodeDoc:class_primary_generator_1a92f67d7b500019b07eea258420c119ed}]{\sphinxcrossref{\DUrole{std,std-ref}{GenerateOne()}}}}}} method is invoked at the beginning of each event. This generates one primary particle/track by setting the properties of the provided \sphinxcode{\sphinxupquote{G4HepEmTrack}} based on the stored configuration. 

\begin{sphinxuseclass}{breathe-sectiondef}\subsubsection*{Public Functions}
\index{PrimaryGenerator::PrimaryGenerator (C++ function)@\spxentry{PrimaryGenerator::PrimaryGenerator}\spxextra{C++ function}}

\begin{fulllineitems}
\phantomsection\label{\detokenize{Simulation/SimulationCodeDoc:_CPPv4N16PrimaryGenerator16PrimaryGeneratorEv}}
\pysigstartsignatures
\pysigstartmultiline
\pysiglinewithargsret{\phantomsection\label{\detokenize{Simulation/SimulationCodeDoc:class_primary_generator_1a4de4401188cb2a20b360307550dac97a}}\sphinxbfcode{\sphinxupquote{\DUrole{n,n}{PrimaryGenerator}}}}{}{}
\pysigstopmultiline
\pysigstopsignatures
\sphinxAtStartPar
Constructor. 

\end{fulllineitems}

\index{PrimaryGenerator::\textasciitilde{}PrimaryGenerator (C++ function)@\spxentry{PrimaryGenerator::\textasciitilde{}PrimaryGenerator}\spxextra{C++ function}}

\begin{fulllineitems}
\phantomsection\label{\detokenize{Simulation/SimulationCodeDoc:_CPPv4N16PrimaryGeneratorD0Ev}}
\pysigstartsignatures
\pysigstartmultiline
\pysiglinewithargsret{\phantomsection\label{\detokenize{Simulation/SimulationCodeDoc:class_primary_generator_1addbe32e71548024d3efeda6d56fa9179}}\DUrole{k,k}{inline}\DUrole{w,w}{  }\sphinxbfcode{\sphinxupquote{\DUrole{n,n}{\textasciitilde{}PrimaryGenerator}}}}{}{}
\pysigstopmultiline
\pysigstopsignatures
\sphinxAtStartPar
Destructor (nothing to do). 

\end{fulllineitems}

\index{PrimaryGenerator::GenerateOne (C++ function)@\spxentry{PrimaryGenerator::GenerateOne}\spxextra{C++ function}}

\begin{fulllineitems}
\phantomsection\label{\detokenize{Simulation/SimulationCodeDoc:_CPPv4N16PrimaryGenerator11GenerateOneER12G4HepEmTrack}}
\pysigstartsignatures
\pysigstartmultiline
\pysiglinewithargsret{\phantomsection\label{\detokenize{Simulation/SimulationCodeDoc:class_primary_generator_1a92f67d7b500019b07eea258420c119ed}}\DUrole{kt,kt}{void}\DUrole{w,w}{  }\sphinxbfcode{\sphinxupquote{\DUrole{n,n}{GenerateOne}}}}{\DUrole{n,n,n}{G4HepEmTrack}\DUrole{w,w}{  }\DUrole{p,p}{\&}\DUrole{n,sig-param,n}{primTrack}}{}
\pysigstopmultiline
\pysigstopsignatures
\sphinxAtStartPar
Generates one primary particle into the provided track. 

\sphinxAtStartPar
The \sphinxcode{\sphinxupquote{{\hyperref[\detokenize{Simulation/SimulationCodeDoc:class_primary_generator_1a92f67d7b500019b07eea258420c119ed}]{\sphinxcrossref{\DUrole{std,std-ref}{GenerateOne()}}}}}} method is invoked at the beginning of each event. This generates one primary particle/track by setting the properties of the provided \sphinxcode{\sphinxupquote{G4HepEmTrack}} based on the stored configuration.

\sphinxAtStartPar
\begin{quote}\begin{description}
\sphinxlineitem{param primTrack}
\sphinxAtStartPar
\sphinxstylestrong{{[}inout{]}} a track to fill in the primary particle properties 

\end{description}\end{quote}


\end{fulllineitems}

\index{PrimaryGenerator::SetKinEnergy (C++ function)@\spxentry{PrimaryGenerator::SetKinEnergy}\spxextra{C++ function}}

\begin{fulllineitems}
\phantomsection\label{\detokenize{Simulation/SimulationCodeDoc:_CPPv4N16PrimaryGenerator12SetKinEnergyEd}}
\pysigstartsignatures
\pysigstartmultiline
\pysiglinewithargsret{\phantomsection\label{\detokenize{Simulation/SimulationCodeDoc:class_primary_generator_1a7973f44f8f33e011625011bc581b86a9}}\DUrole{k,k}{inline}\DUrole{w,w}{  }\DUrole{kt,kt}{void}\DUrole{w,w}{  }\sphinxbfcode{\sphinxupquote{\DUrole{n,n}{SetKinEnergy}}}}{\DUrole{kt,kt}{double}\DUrole{w,w}{  }\DUrole{n,sig-param,n}{ekin}}{}
\pysigstopmultiline
\pysigstopsignatures
\sphinxAtStartPar
Sets kinetic energy of the primary particle. 

\sphinxAtStartPar
\begin{quote}\begin{description}
\sphinxlineitem{param ekin}
\sphinxAtStartPar
\sphinxstylestrong{{[}in{]}} kinetic energy of the primary particle in {[}MeV{]} units. 

\end{description}\end{quote}


\end{fulllineitems}

\index{PrimaryGenerator::GetKinEnergy (C++ function)@\spxentry{PrimaryGenerator::GetKinEnergy}\spxextra{C++ function}}

\begin{fulllineitems}
\phantomsection\label{\detokenize{Simulation/SimulationCodeDoc:_CPPv4NK16PrimaryGenerator12GetKinEnergyEv}}
\pysigstartsignatures
\pysigstartmultiline
\pysiglinewithargsret{\phantomsection\label{\detokenize{Simulation/SimulationCodeDoc:class_primary_generator_1ab153cb15576651798f93b3a43b4f5ce2}}\DUrole{k,k}{inline}\DUrole{w,w}{  }\DUrole{kt,kt}{double}\DUrole{w,w}{  }\sphinxbfcode{\sphinxupquote{\DUrole{n,n}{GetKinEnergy}}}}{}{\DUrole{w,w}{  }\DUrole{k,k}{const}}
\pysigstopmultiline
\pysigstopsignatures
\sphinxAtStartPar
Privides the kinetic energy of the primary particle. 

\sphinxAtStartPar
\begin{quote}\begin{description}
\sphinxlineitem{return}
\sphinxAtStartPar
kinetic energy in {[}MeV{]} units. 

\end{description}\end{quote}


\end{fulllineitems}

\index{PrimaryGenerator::SetPosition (C++ function)@\spxentry{PrimaryGenerator::SetPosition}\spxextra{C++ function}}

\begin{fulllineitems}
\phantomsection\label{\detokenize{Simulation/SimulationCodeDoc:_CPPv4N16PrimaryGenerator11SetPositionEPd}}
\pysigstartsignatures
\pysigstartmultiline
\pysiglinewithargsret{\phantomsection\label{\detokenize{Simulation/SimulationCodeDoc:class_primary_generator_1ae5cb43ec31588f62873d9372b19d7788}}\DUrole{kt,kt}{void}\DUrole{w,w}{  }\sphinxbfcode{\sphinxupquote{\DUrole{n,n}{SetPosition}}}}{\DUrole{kt,kt}{double}\DUrole{w,w}{  }\DUrole{p,p}{*}\DUrole{n,sig-param,n}{pos}}{}
\pysigstopmultiline
\pysigstopsignatures
\sphinxAtStartPar
Sets the position of the primary particle. 

\sphinxAtStartPar
\begin{quote}\begin{description}
\sphinxlineitem{param pos}
\sphinxAtStartPar
\sphinxstylestrong{{[}in{]}} pointer to a 3D (global) position vector (legth is in {[}mm{]}) 

\end{description}\end{quote}


\end{fulllineitems}

\index{PrimaryGenerator::SetPosition (C++ function)@\spxentry{PrimaryGenerator::SetPosition}\spxextra{C++ function}}

\begin{fulllineitems}
\phantomsection\label{\detokenize{Simulation/SimulationCodeDoc:_CPPv4N16PrimaryGenerator11SetPositionEddd}}
\pysigstartsignatures
\pysigstartmultiline
\pysiglinewithargsret{\phantomsection\label{\detokenize{Simulation/SimulationCodeDoc:class_primary_generator_1a54b1885f7aa821d23e70952cf8e6e553}}\DUrole{kt,kt}{void}\DUrole{w,w}{  }\sphinxbfcode{\sphinxupquote{\DUrole{n,n}{SetPosition}}}}{\DUrole{kt,kt}{double}\DUrole{w,w}{  }\DUrole{n,sig-param,n}{x}\sphinxparamcomma \DUrole{kt,kt}{double}\DUrole{w,w}{  }\DUrole{n,sig-param,n}{y}\sphinxparamcomma \DUrole{kt,kt}{double}\DUrole{w,w}{  }\DUrole{n,sig-param,n}{z}}{}
\pysigstopmultiline
\pysigstopsignatures
\sphinxAtStartPar
Sets the position of the primary particle. 

\sphinxAtStartPar
\begin{quote}\begin{description}
\sphinxlineitem{param x}
\sphinxAtStartPar
\sphinxstylestrong{{[}in{]}} x\sphinxhyphen{}coordinate of the positon vector. 

\sphinxlineitem{param y}
\sphinxAtStartPar
\sphinxstylestrong{{[}in{]}} y\sphinxhyphen{}coordinate of the positon vector. 

\sphinxlineitem{param z}
\sphinxAtStartPar
\sphinxstylestrong{{[}in{]}} z\sphinxhyphen{}coordinate of the positon vector. 

\end{description}\end{quote}


\end{fulllineitems}

\index{PrimaryGenerator::GetPosition (C++ function)@\spxentry{PrimaryGenerator::GetPosition}\spxextra{C++ function}}

\begin{fulllineitems}
\phantomsection\label{\detokenize{Simulation/SimulationCodeDoc:_CPPv4NK16PrimaryGenerator11GetPositionEv}}
\pysigstartsignatures
\pysigstartmultiline
\pysiglinewithargsret{\phantomsection\label{\detokenize{Simulation/SimulationCodeDoc:class_primary_generator_1aa7aaba72eed498d9f7246fdc2a4b2bdc}}\DUrole{k,k}{inline}\DUrole{w,w}{  }\DUrole{k,k}{const}\DUrole{w,w}{  }\DUrole{kt,kt}{double}\DUrole{w,w}{  }\DUrole{p,p}{*}\sphinxbfcode{\sphinxupquote{\DUrole{n,n}{GetPosition}}}}{}{\DUrole{w,w}{  }\DUrole{k,k}{const}}
\pysigstopmultiline
\pysigstopsignatures
\sphinxAtStartPar
Provides the 3D position vector of the primary particle. 

\sphinxAtStartPar
\begin{quote}\begin{description}
\sphinxlineitem{return}
\sphinxAtStartPar
pointer to a 3D array that stores the x, y and z\sphinxhyphen{}coordinates of the position vector. 

\end{description}\end{quote}


\end{fulllineitems}

\index{PrimaryGenerator::SetDirection (C++ function)@\spxentry{PrimaryGenerator::SetDirection}\spxextra{C++ function}}

\begin{fulllineitems}
\phantomsection\label{\detokenize{Simulation/SimulationCodeDoc:_CPPv4N16PrimaryGenerator12SetDirectionEPd}}
\pysigstartsignatures
\pysigstartmultiline
\pysiglinewithargsret{\phantomsection\label{\detokenize{Simulation/SimulationCodeDoc:class_primary_generator_1a5821f1896d6ae1e5ac9cb535d812f739}}\DUrole{kt,kt}{void}\DUrole{w,w}{  }\sphinxbfcode{\sphinxupquote{\DUrole{n,n}{SetDirection}}}}{\DUrole{kt,kt}{double}\DUrole{w,w}{  }\DUrole{p,p}{*}\DUrole{n,sig-param,n}{dir}}{}
\pysigstopmultiline
\pysigstopsignatures
\sphinxAtStartPar
Sets the direction of the primary particle. 

\sphinxAtStartPar
\begin{quote}\begin{description}
\sphinxlineitem{param dir}
\sphinxAtStartPar
\sphinxstylestrong{{[}in{]}} pointer to a 3D normalised direction vector 

\end{description}\end{quote}


\end{fulllineitems}

\index{PrimaryGenerator::SetDirection (C++ function)@\spxentry{PrimaryGenerator::SetDirection}\spxextra{C++ function}}

\begin{fulllineitems}
\phantomsection\label{\detokenize{Simulation/SimulationCodeDoc:_CPPv4N16PrimaryGenerator12SetDirectionEddd}}
\pysigstartsignatures
\pysigstartmultiline
\pysiglinewithargsret{\phantomsection\label{\detokenize{Simulation/SimulationCodeDoc:class_primary_generator_1a1c960da9771f4c27722dd31f881381ff}}\DUrole{kt,kt}{void}\DUrole{w,w}{  }\sphinxbfcode{\sphinxupquote{\DUrole{n,n}{SetDirection}}}}{\DUrole{kt,kt}{double}\DUrole{w,w}{  }\DUrole{n,sig-param,n}{x}\sphinxparamcomma \DUrole{kt,kt}{double}\DUrole{w,w}{  }\DUrole{n,sig-param,n}{y}\sphinxparamcomma \DUrole{kt,kt}{double}\DUrole{w,w}{  }\DUrole{n,sig-param,n}{z}}{}
\pysigstopmultiline
\pysigstopsignatures
\sphinxAtStartPar
Sets the direction of the primary particle. 

\sphinxAtStartPar
\begin{quote}\begin{description}
\sphinxlineitem{param x}
\sphinxAtStartPar
\sphinxstylestrong{{[}in{]}} x\sphinxhyphen{}coordinate of the normalised direction vector. 

\sphinxlineitem{param y}
\sphinxAtStartPar
\sphinxstylestrong{{[}in{]}} y\sphinxhyphen{}coordinate of the normalised direction vector. 

\sphinxlineitem{param z}
\sphinxAtStartPar
\sphinxstylestrong{{[}in{]}} z\sphinxhyphen{}coordinate of the normalised direction vector. 

\end{description}\end{quote}


\end{fulllineitems}

\index{PrimaryGenerator::GetDirection (C++ function)@\spxentry{PrimaryGenerator::GetDirection}\spxextra{C++ function}}

\begin{fulllineitems}
\phantomsection\label{\detokenize{Simulation/SimulationCodeDoc:_CPPv4NK16PrimaryGenerator12GetDirectionEv}}
\pysigstartsignatures
\pysigstartmultiline
\pysiglinewithargsret{\phantomsection\label{\detokenize{Simulation/SimulationCodeDoc:class_primary_generator_1a8da669cce14bcc039ae99a52fc24ebbf}}\DUrole{k,k}{inline}\DUrole{w,w}{  }\DUrole{k,k}{const}\DUrole{w,w}{  }\DUrole{kt,kt}{double}\DUrole{w,w}{  }\DUrole{p,p}{*}\sphinxbfcode{\sphinxupquote{\DUrole{n,n}{GetDirection}}}}{}{\DUrole{w,w}{  }\DUrole{k,k}{const}}
\pysigstopmultiline
\pysigstopsignatures
\sphinxAtStartPar
Provides the 3D normalised direction vector of the primary particle. 

\sphinxAtStartPar
\begin{quote}\begin{description}
\sphinxlineitem{return}
\sphinxAtStartPar
pointer to a 3D array that stores the x, y and z\sphinxhyphen{}coordinates of the normalised direction. 

\end{description}\end{quote}


\end{fulllineitems}

\index{PrimaryGenerator::SetCharge (C++ function)@\spxentry{PrimaryGenerator::SetCharge}\spxextra{C++ function}}

\begin{fulllineitems}
\phantomsection\label{\detokenize{Simulation/SimulationCodeDoc:_CPPv4N16PrimaryGenerator9SetChargeEd}}
\pysigstartsignatures
\pysigstartmultiline
\pysiglinewithargsret{\phantomsection\label{\detokenize{Simulation/SimulationCodeDoc:class_primary_generator_1aefe91e38900dce2993b54705f9603564}}\DUrole{k,k}{inline}\DUrole{w,w}{  }\DUrole{kt,kt}{void}\DUrole{w,w}{  }\sphinxbfcode{\sphinxupquote{\DUrole{n,n}{SetCharge}}}}{\DUrole{kt,kt}{double}\DUrole{w,w}{  }\DUrole{n,sig-param,n}{ch}}{}
\pysigstopmultiline
\pysigstopsignatures
\sphinxAtStartPar
Sets the charge of the primary particle that also determines its type. 

\sphinxAtStartPar
\begin{quote}\begin{description}
\sphinxlineitem{param ch}
\sphinxAtStartPar
\sphinxstylestrong{{[}in{]}} the charge in e+ change units: \sphinxhyphen{}1 e\sphinxhyphen{}; 0 gamma; +1 e+. 

\end{description}\end{quote}


\end{fulllineitems}

\index{PrimaryGenerator::GetCharge (C++ function)@\spxentry{PrimaryGenerator::GetCharge}\spxextra{C++ function}}

\begin{fulllineitems}
\phantomsection\label{\detokenize{Simulation/SimulationCodeDoc:_CPPv4NK16PrimaryGenerator9GetChargeEv}}
\pysigstartsignatures
\pysigstartmultiline
\pysiglinewithargsret{\phantomsection\label{\detokenize{Simulation/SimulationCodeDoc:class_primary_generator_1ac42eead57d8826337acd46257ff16bdc}}\DUrole{k,k}{inline}\DUrole{w,w}{  }\DUrole{kt,kt}{double}\DUrole{w,w}{  }\sphinxbfcode{\sphinxupquote{\DUrole{n,n}{GetCharge}}}}{}{\DUrole{w,w}{  }\DUrole{k,k}{const}}
\pysigstopmultiline
\pysigstopsignatures
\sphinxAtStartPar
Privides the charge of the primary particle. 

\sphinxAtStartPar
\begin{quote}\begin{description}
\sphinxlineitem{return}
\sphinxAtStartPar
charge: \sphinxhyphen{}1 e\sphinxhyphen{}; 0 gamma; +1 e+. 

\end{description}\end{quote}


\end{fulllineitems}


\end{sphinxuseclass}
\begin{sphinxuseclass}{breathe-sectiondef}\subsubsection*{Private Members}
\index{PrimaryGenerator::fKinEnergy (C++ member)@\spxentry{PrimaryGenerator::fKinEnergy}\spxextra{C++ member}}

\begin{fulllineitems}
\phantomsection\label{\detokenize{Simulation/SimulationCodeDoc:_CPPv4N16PrimaryGenerator10fKinEnergyE}}
\pysigstartsignatures
\pysigstartmultiline
\pysigline{\phantomsection\label{\detokenize{Simulation/SimulationCodeDoc:class_primary_generator_1a291044306551ba4b1009787414033a02}}\DUrole{kt,kt}{double}\DUrole{w,w}{  }\sphinxbfcode{\sphinxupquote{\DUrole{n,n}{fKinEnergy}}}}
\pysigstopmultiline
\pysigstopsignatures
\sphinxAtStartPar
Kinetic energy of the primary particle in {[}MeV{]} units. 

\end{fulllineitems}

\index{PrimaryGenerator::fPosition (C++ member)@\spxentry{PrimaryGenerator::fPosition}\spxextra{C++ member}}

\begin{fulllineitems}
\phantomsection\label{\detokenize{Simulation/SimulationCodeDoc:_CPPv4N16PrimaryGenerator9fPositionE}}
\pysigstartsignatures
\pysigstartmultiline
\pysigline{\phantomsection\label{\detokenize{Simulation/SimulationCodeDoc:class_primary_generator_1ae101eb40f0432f750c20d4263b7b67b6}}\DUrole{kt,kt}{double}\DUrole{w,w}{  }\sphinxbfcode{\sphinxupquote{\DUrole{n,n}{fPosition}}}\DUrole{p,p}{{[}}\DUrole{m,m}{3}\DUrole{p,p}{{]}}}
\pysigstopmultiline
\pysigstopsignatures
\sphinxAtStartPar
Position of the primary particles in (global) coordinates (legth is in {[}mm{]}). 

\end{fulllineitems}

\index{PrimaryGenerator::fDirection (C++ member)@\spxentry{PrimaryGenerator::fDirection}\spxextra{C++ member}}

\begin{fulllineitems}
\phantomsection\label{\detokenize{Simulation/SimulationCodeDoc:_CPPv4N16PrimaryGenerator10fDirectionE}}
\pysigstartsignatures
\pysigstartmultiline
\pysigline{\phantomsection\label{\detokenize{Simulation/SimulationCodeDoc:class_primary_generator_1a33b23825762b62ff2cbe8be0368c89fb}}\DUrole{kt,kt}{double}\DUrole{w,w}{  }\sphinxbfcode{\sphinxupquote{\DUrole{n,n}{fDirection}}}\DUrole{p,p}{{[}}\DUrole{m,m}{3}\DUrole{p,p}{{]}}}
\pysigstopmultiline
\pysigstopsignatures
\sphinxAtStartPar
Normalised direction of the primary particles. 

\end{fulllineitems}

\index{PrimaryGenerator::fCharge (C++ member)@\spxentry{PrimaryGenerator::fCharge}\spxextra{C++ member}}

\begin{fulllineitems}
\phantomsection\label{\detokenize{Simulation/SimulationCodeDoc:_CPPv4N16PrimaryGenerator7fChargeE}}
\pysigstartsignatures
\pysigstartmultiline
\pysigline{\phantomsection\label{\detokenize{Simulation/SimulationCodeDoc:class_primary_generator_1a42705ccba9e6e220beaccc03d965783f}}\DUrole{kt,kt}{double}\DUrole{w,w}{  }\sphinxbfcode{\sphinxupquote{\DUrole{n,n}{fCharge}}}}
\pysigstopmultiline
\pysigstopsignatures
\sphinxAtStartPar
Charge of the primary particle in units of e+ charge: \sphinxhyphen{}1 e\sphinxhyphen{}; 0 gamma; +1 e+. 

\end{fulllineitems}


\end{sphinxuseclass}
\end{fulllineitems}



\subsection{Auxiliary code documentation}
\label{\detokenize{Simulation/SimulationCodeDoc:auxiliary-code-documentation}}

\subsubsection{Collecting data during the simulation}
\label{\detokenize{Simulation/SimulationCodeDoc:collecting-data-during-the-simulation}}
\sphinxAtStartPar
A collection of data that are recorded during the simulation. 

\sphinxAtStartPar
\begin{description}
\sphinxlineitem{\sphinxstylestrong{Author}}
\sphinxAtStartPar
M. Novak 

\sphinxlineitem{\sphinxstylestrong{Date}}
\sphinxAtStartPar
July 2023

\end{description}


\sphinxAtStartPar
The following data is recorded during the simulation (mean is per event):\begin{itemize}
\item {} 
\sphinxAtStartPar
mean values in the individual layers of the calorimeter for energy deposit, neutral (gamma) and charged (electron/positron) particle simulation steps

\item {} 
\sphinxAtStartPar
mean number of energy deposited in the \sphinxcode{\sphinxupquote{absorber}} and \sphinxcode{\sphinxupquote{gap}}

\item {} 
\sphinxAtStartPar
mean number of secondary gamma, electron and positrons produced

\item {} 
\sphinxAtStartPar
mean number of neutral (gamma) and charged (electron/positron)

\end{itemize}


\sphinxAtStartPar
Quantities, recorded in the individual layers are stored in histograms and written to files at the end of the simulation while the others are reported in the screen. An example looks like 
\begin{sphinxVerbatim}[commandchars=\\\{\}]
\PYG{o}{\PYGZhy{}}\PYG{o}{\PYGZhy{}}\PYG{o}{\PYGZhy{}} \PYG{n}{Results}\PYG{p}{:}\PYG{p}{:}\PYG{n}{WriteResults} \PYG{o}{\PYGZhy{}}\PYG{o}{\PYGZhy{}}\PYG{o}{\PYGZhy{}}\PYG{o}{\PYGZhy{}}\PYG{o}{\PYGZhy{}}\PYG{o}{\PYGZhy{}}\PYG{o}{\PYGZhy{}}\PYG{o}{\PYGZhy{}}\PYG{o}{\PYGZhy{}}\PYG{o}{\PYGZhy{}}\PYG{o}{\PYGZhy{}}\PYG{o}{\PYGZhy{}}\PYG{o}{\PYGZhy{}}\PYG{o}{\PYGZhy{}}\PYG{o}{\PYGZhy{}}\PYG{o}{\PYGZhy{}}\PYG{o}{\PYGZhy{}}\PYG{o}{\PYGZhy{}}\PYG{o}{\PYGZhy{}}\PYG{o}{\PYGZhy{}}\PYG{o}{\PYGZhy{}}\PYG{o}{\PYGZhy{}}\PYG{o}{\PYGZhy{}}\PYG{o}{\PYGZhy{}}\PYG{o}{\PYGZhy{}}\PYG{o}{\PYGZhy{}}\PYG{o}{\PYGZhy{}}\PYG{o}{\PYGZhy{}}\PYG{o}{\PYGZhy{}}\PYG{o}{\PYGZhy{}}\PYG{o}{\PYGZhy{}}\PYG{o}{\PYGZhy{}}\PYG{o}{\PYGZhy{}}\PYG{o}{\PYGZhy{}}

\PYG{n}{Absorber}\PYG{p}{:} \PYG{n}{mean} \PYG{n}{Edep} \PYG{o}{=} \PYG{l+m+mf}{6722.95} \PYG{p}{[}\PYG{n}{MeV}\PYG{p}{]} \PYG{o+ow}{and}  \PYG{n}{Std}\PYG{o}{\PYGZhy{}}\PYG{n}{dev} \PYG{o}{=} \PYG{l+m+mf}{309.636} \PYG{p}{[}\PYG{n}{MeV}\PYG{p}{]}
\PYG{n}{Gap}     \PYG{p}{:} \PYG{n}{mean} \PYG{n}{Edep} \PYG{o}{=} \PYG{l+m+mf}{2571.75} \PYG{p}{[}\PYG{n}{MeV}\PYG{p}{]} \PYG{o+ow}{and}  \PYG{n}{Std}\PYG{o}{\PYGZhy{}}\PYG{n}{dev} \PYG{o}{=} \PYG{l+m+mf}{118.507} \PYG{p}{[}\PYG{n}{MeV}\PYG{p}{]}

\PYG{n}{Mean} \PYG{n}{number} \PYG{n}{of} \PYG{n}{gamma}       \PYG{l+m+mf}{4457.043}
\PYG{n}{Mean} \PYG{n}{number} \PYG{n}{of} \PYG{n}{e}\PYG{o}{\PYGZhy{}}          \PYG{l+m+mf}{7957.899}
\PYG{n}{Mean} \PYG{n}{number} \PYG{n}{of} \PYG{n}{e}\PYG{o}{+}          \PYG{l+m+mf}{428.922}

\PYG{n}{Mean} \PYG{n}{number} \PYG{n}{of} \PYG{n}{e}\PYG{o}{\PYGZhy{}}\PYG{o}{/}\PYG{n}{e}\PYG{o}{+} \PYG{n}{steps} \PYG{l+m+mi}{36097}
\PYG{n}{Mean} \PYG{n}{number} \PYG{n}{of} \PYG{n}{gamma} \PYG{n}{steps} \PYG{l+m+mf}{40436.2}
\PYG{o}{\PYGZhy{}}\PYG{o}{\PYGZhy{}}\PYG{o}{\PYGZhy{}}\PYG{o}{\PYGZhy{}}\PYG{o}{\PYGZhy{}}\PYG{o}{\PYGZhy{}}\PYG{o}{\PYGZhy{}}\PYG{o}{\PYGZhy{}}\PYG{o}{\PYGZhy{}}\PYG{o}{\PYGZhy{}}\PYG{o}{\PYGZhy{}}\PYG{o}{\PYGZhy{}}\PYG{o}{\PYGZhy{}}\PYG{o}{\PYGZhy{}}\PYG{o}{\PYGZhy{}}\PYG{o}{\PYGZhy{}}\PYG{o}{\PYGZhy{}}\PYG{o}{\PYGZhy{}}\PYG{o}{\PYGZhy{}}\PYG{o}{\PYGZhy{}}\PYG{o}{\PYGZhy{}}\PYG{o}{\PYGZhy{}}\PYG{o}{\PYGZhy{}}\PYG{o}{\PYGZhy{}}\PYG{o}{\PYGZhy{}}\PYG{o}{\PYGZhy{}}\PYG{o}{\PYGZhy{}}\PYG{o}{\PYGZhy{}}\PYG{o}{\PYGZhy{}}\PYG{o}{\PYGZhy{}}\PYG{o}{\PYGZhy{}}\PYG{o}{\PYGZhy{}}\PYG{o}{\PYGZhy{}}\PYG{o}{\PYGZhy{}}\PYG{o}{\PYGZhy{}}\PYG{o}{\PYGZhy{}}\PYG{o}{\PYGZhy{}}\PYG{o}{\PYGZhy{}}\PYG{o}{\PYGZhy{}}\PYG{o}{\PYGZhy{}}\PYG{o}{\PYGZhy{}}\PYG{o}{\PYGZhy{}}\PYG{o}{\PYGZhy{}}\PYG{o}{\PYGZhy{}}\PYG{o}{\PYGZhy{}}\PYG{o}{\PYGZhy{}}\PYG{o}{\PYGZhy{}}\PYG{o}{\PYGZhy{}}\PYG{o}{\PYGZhy{}}\PYG{o}{\PYGZhy{}}\PYG{o}{\PYGZhy{}}\PYG{o}{\PYGZhy{}}\PYG{o}{\PYGZhy{}}\PYG{o}{\PYGZhy{}}\PYG{o}{\PYGZhy{}}\PYG{o}{\PYGZhy{}}\PYG{o}{\PYGZhy{}}\PYG{o}{\PYGZhy{}}\PYG{o}{\PYGZhy{}}\PYG{o}{\PYGZhy{}}
\end{sphinxVerbatim}


\begin{sphinxuseclass}{breathe-sectiondef}\subsubsection*{Functions}
\index{WriteResults (C++ function)@\spxentry{WriteResults}\spxextra{C++ function}}

\begin{fulllineitems}
\phantomsection\label{\detokenize{Simulation/SimulationCodeDoc:_CPPv412WriteResultsR7Resultsi}}
\pysigstartsignatures
\pysigstartmultiline
\pysiglinewithargsret{\phantomsection\label{\detokenize{Simulation/SimulationCodeDoc:_results_8hh_1a2e94780b3b797ba10abaf1154db56016}}\DUrole{kt,kt}{void}\DUrole{w,w}{  }\sphinxbfcode{\sphinxupquote{\DUrole{n,n}{WriteResults}}}}{\DUrole{k,k}{struct}\DUrole{w,w}{  }{\hyperref[\detokenize{Simulation/SimulationCodeDoc:_CPPv47Results}]{\sphinxcrossref{\DUrole{n,n,n}{Results}}}}\DUrole{w,w}{  }\DUrole{p,p}{\&}\DUrole{n,sig-param,n}{res}\sphinxparamcomma \DUrole{kt,kt}{int}\DUrole{w,w}{  }\DUrole{n,sig-param,n}{numEvents}\DUrole{w,w}{  }\DUrole{p,p}{=}\DUrole{w,w}{  }\DUrole{m,m}{1}}{}
\pysigstopmultiline
\pysigstopsignatures
\sphinxAtStartPar
Writes the final results of the simulation. 

\sphinxAtStartPar
Writes the 3 histrograms (mean energy deposit, \(\gamma\) and \(e^-/e^+\) steps per\sphinxhyphen{}layer) into files while all the other collected data to the screen. 

\end{fulllineitems}


\end{sphinxuseclass}\index{ResultsPerEvent (C++ struct)@\spxentry{ResultsPerEvent}\spxextra{C++ struct}}

\begin{fulllineitems}
\phantomsection\label{\detokenize{Simulation/SimulationCodeDoc:_CPPv415ResultsPerEvent}}
\pysigstartsignatures
\pysigstartmultiline
\pysigline{\phantomsection\label{\detokenize{Simulation/SimulationCodeDoc:struct_results_per_event}}\DUrole{k,k}{struct}\DUrole{w,w}{  }\sphinxbfcode{\sphinxupquote{\DUrole{n,n}{ResultsPerEvent}}}}
\pysigstopmultiline
\pysigstopsignatures\sphinxstyleemphasis{\#include \textless{}Results.hh\textgreater{}}
\sphinxAtStartPar
Data that needs to be accumulated during one \sphinxcode{\sphinxupquote{event}} (the scope is one event): 

\sphinxAtStartPar
\begin{itemize}
\item {} 
\sphinxAtStartPar
at the beginning of an \sphinxcode{\sphinxupquote{event}}: usually reset (to zero)

\item {} 
\sphinxAtStartPar
at the end of an \sphinxcode{\sphinxupquote{event}}: usually written to the run scope data (see the \sphinxcode{\sphinxupquote{{\hyperref[\detokenize{Simulation/SimulationCodeDoc:struct_results}]{\sphinxcrossref{\DUrole{std,std-ref}{Results}}}}}} below) 

\end{itemize}


\begin{sphinxuseclass}{breathe-sectiondef}\subsubsection*{Public Members}
\index{ResultsPerEvent::fEdepAbs (C++ member)@\spxentry{ResultsPerEvent::fEdepAbs}\spxextra{C++ member}}

\begin{fulllineitems}
\phantomsection\label{\detokenize{Simulation/SimulationCodeDoc:_CPPv4N15ResultsPerEvent8fEdepAbsE}}
\pysigstartsignatures
\pysigstartmultiline
\pysigline{\phantomsection\label{\detokenize{Simulation/SimulationCodeDoc:struct_results_per_event_1a56300e62e8aa6cd9fb4b1cbdc6ced71d}}\DUrole{kt,kt}{double}\DUrole{w,w}{  }\sphinxbfcode{\sphinxupquote{\DUrole{n,n}{fEdepAbs}}}\DUrole{w,w}{  }\DUrole{p,p}{=}\DUrole{w,w}{  }\DUrole{p,p}{\{}\DUrole{m,m}{0.0}\DUrole{p,p}{\}}}
\pysigstopmultiline
\pysigstopsignatures
\sphinxAtStartPar
energy deposit in the absorber during one event 

\end{fulllineitems}

\index{ResultsPerEvent::fEdepGap (C++ member)@\spxentry{ResultsPerEvent::fEdepGap}\spxextra{C++ member}}

\begin{fulllineitems}
\phantomsection\label{\detokenize{Simulation/SimulationCodeDoc:_CPPv4N15ResultsPerEvent8fEdepGapE}}
\pysigstartsignatures
\pysigstartmultiline
\pysigline{\phantomsection\label{\detokenize{Simulation/SimulationCodeDoc:struct_results_per_event_1acb585ff9dcb316804b4ac5999ee914d8}}\DUrole{kt,kt}{double}\DUrole{w,w}{  }\sphinxbfcode{\sphinxupquote{\DUrole{n,n}{fEdepGap}}}\DUrole{w,w}{  }\DUrole{p,p}{=}\DUrole{w,w}{  }\DUrole{p,p}{\{}\DUrole{m,m}{0.0}\DUrole{p,p}{\}}}
\pysigstopmultiline
\pysigstopsignatures
\sphinxAtStartPar
energy deposit in the gap during one event 

\end{fulllineitems}

\index{ResultsPerEvent::fNumSecGamma (C++ member)@\spxentry{ResultsPerEvent::fNumSecGamma}\spxextra{C++ member}}

\begin{fulllineitems}
\phantomsection\label{\detokenize{Simulation/SimulationCodeDoc:_CPPv4N15ResultsPerEvent12fNumSecGammaE}}
\pysigstartsignatures
\pysigstartmultiline
\pysigline{\phantomsection\label{\detokenize{Simulation/SimulationCodeDoc:struct_results_per_event_1a7ec88cb879bc6b6d3457dbe08d1bbf01}}\DUrole{kt,kt}{double}\DUrole{w,w}{  }\sphinxbfcode{\sphinxupquote{\DUrole{n,n}{fNumSecGamma}}}\DUrole{w,w}{  }\DUrole{p,p}{=}\DUrole{w,w}{  }\DUrole{p,p}{\{}\DUrole{m,m}{0.0}\DUrole{p,p}{\}}}
\pysigstopmultiline
\pysigstopsignatures
\sphinxAtStartPar
number of seconday \(\gamma\) particles generated during one event 

\end{fulllineitems}

\index{ResultsPerEvent::fNumSecElectron (C++ member)@\spxentry{ResultsPerEvent::fNumSecElectron}\spxextra{C++ member}}

\begin{fulllineitems}
\phantomsection\label{\detokenize{Simulation/SimulationCodeDoc:_CPPv4N15ResultsPerEvent15fNumSecElectronE}}
\pysigstartsignatures
\pysigstartmultiline
\pysigline{\phantomsection\label{\detokenize{Simulation/SimulationCodeDoc:struct_results_per_event_1a6b4ce615a2259673ce696b747c82812c}}\DUrole{kt,kt}{double}\DUrole{w,w}{  }\sphinxbfcode{\sphinxupquote{\DUrole{n,n}{fNumSecElectron}}}\DUrole{w,w}{  }\DUrole{p,p}{=}\DUrole{w,w}{  }\DUrole{p,p}{\{}\DUrole{m,m}{0.0}\DUrole{p,p}{\}}}
\pysigstopmultiline
\pysigstopsignatures
\sphinxAtStartPar
number of seconday \(e^-\) particles generated during one event 

\end{fulllineitems}

\index{ResultsPerEvent::fNumSecPositron (C++ member)@\spxentry{ResultsPerEvent::fNumSecPositron}\spxextra{C++ member}}

\begin{fulllineitems}
\phantomsection\label{\detokenize{Simulation/SimulationCodeDoc:_CPPv4N15ResultsPerEvent15fNumSecPositronE}}
\pysigstartsignatures
\pysigstartmultiline
\pysigline{\phantomsection\label{\detokenize{Simulation/SimulationCodeDoc:struct_results_per_event_1aaacc361c66159659b16f7016bee3670c}}\DUrole{kt,kt}{double}\DUrole{w,w}{  }\sphinxbfcode{\sphinxupquote{\DUrole{n,n}{fNumSecPositron}}}\DUrole{w,w}{  }\DUrole{p,p}{=}\DUrole{w,w}{  }\DUrole{p,p}{\{}\DUrole{m,m}{0.0}\DUrole{p,p}{\}}}
\pysigstopmultiline
\pysigstopsignatures
\sphinxAtStartPar
number of seconday \(e^+\) particles generated during one event 

\end{fulllineitems}

\index{ResultsPerEvent::fNumStepsGamma (C++ member)@\spxentry{ResultsPerEvent::fNumStepsGamma}\spxextra{C++ member}}

\begin{fulllineitems}
\phantomsection\label{\detokenize{Simulation/SimulationCodeDoc:_CPPv4N15ResultsPerEvent14fNumStepsGammaE}}
\pysigstartsignatures
\pysigstartmultiline
\pysigline{\phantomsection\label{\detokenize{Simulation/SimulationCodeDoc:struct_results_per_event_1ae02070852cace2b3b94741ef81b200a0}}\DUrole{kt,kt}{double}\DUrole{w,w}{  }\sphinxbfcode{\sphinxupquote{\DUrole{n,n}{fNumStepsGamma}}}\DUrole{w,w}{  }\DUrole{p,p}{=}\DUrole{w,w}{  }\DUrole{p,p}{\{}\DUrole{m,m}{0.0}\DUrole{p,p}{\}}}
\pysigstopmultiline
\pysigstopsignatures
\sphinxAtStartPar
number of \(\gamma\) simulation steps during one event 

\end{fulllineitems}

\index{ResultsPerEvent::fNumStepsElPos (C++ member)@\spxentry{ResultsPerEvent::fNumStepsElPos}\spxextra{C++ member}}

\begin{fulllineitems}
\phantomsection\label{\detokenize{Simulation/SimulationCodeDoc:_CPPv4N15ResultsPerEvent14fNumStepsElPosE}}
\pysigstartsignatures
\pysigstartmultiline
\pysigline{\phantomsection\label{\detokenize{Simulation/SimulationCodeDoc:struct_results_per_event_1acf7c36b68976c0d745abf51b964f9335}}\DUrole{kt,kt}{double}\DUrole{w,w}{  }\sphinxbfcode{\sphinxupquote{\DUrole{n,n}{fNumStepsElPos}}}\DUrole{w,w}{  }\DUrole{p,p}{=}\DUrole{w,w}{  }\DUrole{p,p}{\{}\DUrole{m,m}{0.0}\DUrole{p,p}{\}}}
\pysigstopmultiline
\pysigstopsignatures
\sphinxAtStartPar
number of \(e^-/e^+\) simulation steps during one event 

\end{fulllineitems}


\end{sphinxuseclass}
\end{fulllineitems}

\index{Results (C++ struct)@\spxentry{Results}\spxextra{C++ struct}}

\begin{fulllineitems}
\phantomsection\label{\detokenize{Simulation/SimulationCodeDoc:_CPPv47Results}}
\pysigstartsignatures
\pysigstartmultiline
\pysigline{\phantomsection\label{\detokenize{Simulation/SimulationCodeDoc:struct_results}}\DUrole{k,k}{struct}\DUrole{w,w}{  }\sphinxbfcode{\sphinxupquote{\DUrole{n,n}{Results}}}}
\pysigstopmultiline
\pysigstopsignatures\sphinxstyleemphasis{\#include \textless{}Results.hh\textgreater{}}
\sphinxAtStartPar
Data that are collected during the entire \sphinxcode{\sphinxupquote{run}} of the simulation: 

\sphinxAtStartPar
\begin{itemize}
\item {} 
\sphinxAtStartPar
at the beginning of the \sphinxcode{\sphinxupquote{ru}}n: need to be initialised

\item {} 
\sphinxAtStartPar
at the end of an \sphinxcode{\sphinxupquote{run}}: written out (to file or to the std output) Mean quantities are computed over the simulated events. 

\end{itemize}


\begin{sphinxuseclass}{breathe-sectiondef}\subsubsection*{Public Members}
\index{Results::fEdepPerLayer (C++ member)@\spxentry{Results::fEdepPerLayer}\spxextra{C++ member}}

\begin{fulllineitems}
\phantomsection\label{\detokenize{Simulation/SimulationCodeDoc:_CPPv4N7Results13fEdepPerLayerE}}
\pysigstartsignatures
\pysigstartmultiline
\pysigline{\phantomsection\label{\detokenize{Simulation/SimulationCodeDoc:struct_results_1a868f71c54a3068363ff43ec93ce835f0}}{\hyperref[\detokenize{Simulation/SimulationCodeDoc:_CPPv44Hist}]{\sphinxcrossref{\DUrole{n,n,n}{Hist}}}}\DUrole{w,w}{  }\sphinxbfcode{\sphinxupquote{\DUrole{n,n}{fEdepPerLayer}}}}
\pysigstopmultiline
\pysigstopsignatures
\sphinxAtStartPar
mean energy deposit per\sphinxhyphen{}layer histogram 

\end{fulllineitems}

\index{Results::fGammaTrackLenghtPerLayer (C++ member)@\spxentry{Results::fGammaTrackLenghtPerLayer}\spxextra{C++ member}}

\begin{fulllineitems}
\phantomsection\label{\detokenize{Simulation/SimulationCodeDoc:_CPPv4N7Results25fGammaTrackLenghtPerLayerE}}
\pysigstartsignatures
\pysigstartmultiline
\pysigline{\phantomsection\label{\detokenize{Simulation/SimulationCodeDoc:struct_results_1af2f609c28fe15e0892646fae98c8ef16}}{\hyperref[\detokenize{Simulation/SimulationCodeDoc:_CPPv44Hist}]{\sphinxcrossref{\DUrole{n,n,n}{Hist}}}}\DUrole{w,w}{  }\sphinxbfcode{\sphinxupquote{\DUrole{n,n}{fGammaTrackLenghtPerLayer}}}}
\pysigstopmultiline
\pysigstopsignatures
\sphinxAtStartPar
mean number of \(\gamma\) steps per\sphinxhyphen{}layer histogram 

\end{fulllineitems}

\index{Results::fElPosTrackLenghtPerLayer (C++ member)@\spxentry{Results::fElPosTrackLenghtPerLayer}\spxextra{C++ member}}

\begin{fulllineitems}
\phantomsection\label{\detokenize{Simulation/SimulationCodeDoc:_CPPv4N7Results25fElPosTrackLenghtPerLayerE}}
\pysigstartsignatures
\pysigstartmultiline
\pysigline{\phantomsection\label{\detokenize{Simulation/SimulationCodeDoc:struct_results_1a1d825ca3eb96f2604ee4750d6f93bd08}}{\hyperref[\detokenize{Simulation/SimulationCodeDoc:_CPPv44Hist}]{\sphinxcrossref{\DUrole{n,n,n}{Hist}}}}\DUrole{w,w}{  }\sphinxbfcode{\sphinxupquote{\DUrole{n,n}{fElPosTrackLenghtPerLayer}}}}
\pysigstopmultiline
\pysigstopsignatures
\sphinxAtStartPar
mean number of \(e^-/e^+\) steps per\sphinxhyphen{}layer histogram 

\end{fulllineitems}

\index{Results::fEdepAbs (C++ member)@\spxentry{Results::fEdepAbs}\spxextra{C++ member}}

\begin{fulllineitems}
\phantomsection\label{\detokenize{Simulation/SimulationCodeDoc:_CPPv4N7Results8fEdepAbsE}}
\pysigstartsignatures
\pysigstartmultiline
\pysigline{\phantomsection\label{\detokenize{Simulation/SimulationCodeDoc:struct_results_1a7fc8bebd72ddf91416801fe0a2dbc07b}}\DUrole{kt,kt}{double}\DUrole{w,w}{  }\sphinxbfcode{\sphinxupquote{\DUrole{n,n}{fEdepAbs}}}\DUrole{w,w}{  }\DUrole{p,p}{=}\DUrole{w,w}{  }\DUrole{p,p}{\{}\DUrole{m,m}{0.0}\DUrole{p,p}{\}}}
\pysigstopmultiline
\pysigstopsignatures
\sphinxAtStartPar
mean energy deposit in the \sphinxcode{\sphinxupquote{absorber}}

\end{fulllineitems}

\index{Results::fEdepAbs2 (C++ member)@\spxentry{Results::fEdepAbs2}\spxextra{C++ member}}

\begin{fulllineitems}
\phantomsection\label{\detokenize{Simulation/SimulationCodeDoc:_CPPv4N7Results9fEdepAbs2E}}
\pysigstartsignatures
\pysigstartmultiline
\pysigline{\phantomsection\label{\detokenize{Simulation/SimulationCodeDoc:struct_results_1aa8199cd60249fdbee9e78e2423856dd6}}\DUrole{kt,kt}{double}\DUrole{w,w}{  }\sphinxbfcode{\sphinxupquote{\DUrole{n,n}{fEdepAbs2}}}\DUrole{w,w}{  }\DUrole{p,p}{=}\DUrole{w,w}{  }\DUrole{p,p}{\{}\DUrole{m,m}{0.0}\DUrole{p,p}{\}}}
\pysigstopmultiline
\pysigstopsignatures
\sphinxAtStartPar
mean of the squared energy deposit in the \sphinxcode{\sphinxupquote{absorber}}

\end{fulllineitems}

\index{Results::fEdepGap (C++ member)@\spxentry{Results::fEdepGap}\spxextra{C++ member}}

\begin{fulllineitems}
\phantomsection\label{\detokenize{Simulation/SimulationCodeDoc:_CPPv4N7Results8fEdepGapE}}
\pysigstartsignatures
\pysigstartmultiline
\pysigline{\phantomsection\label{\detokenize{Simulation/SimulationCodeDoc:struct_results_1a6de00d97715d97f9a133f059b7ef6ef0}}\DUrole{kt,kt}{double}\DUrole{w,w}{  }\sphinxbfcode{\sphinxupquote{\DUrole{n,n}{fEdepGap}}}\DUrole{w,w}{  }\DUrole{p,p}{=}\DUrole{w,w}{  }\DUrole{p,p}{\{}\DUrole{m,m}{0.0}\DUrole{p,p}{\}}}
\pysigstopmultiline
\pysigstopsignatures
\sphinxAtStartPar
mean energy deposit in the \sphinxcode{\sphinxupquote{gap}}

\end{fulllineitems}

\index{Results::fEdepGap2 (C++ member)@\spxentry{Results::fEdepGap2}\spxextra{C++ member}}

\begin{fulllineitems}
\phantomsection\label{\detokenize{Simulation/SimulationCodeDoc:_CPPv4N7Results9fEdepGap2E}}
\pysigstartsignatures
\pysigstartmultiline
\pysigline{\phantomsection\label{\detokenize{Simulation/SimulationCodeDoc:struct_results_1a32993f72fd83dfb4bf5c86b7e9d8ad45}}\DUrole{kt,kt}{double}\DUrole{w,w}{  }\sphinxbfcode{\sphinxupquote{\DUrole{n,n}{fEdepGap2}}}\DUrole{w,w}{  }\DUrole{p,p}{=}\DUrole{w,w}{  }\DUrole{p,p}{\{}\DUrole{m,m}{0.0}\DUrole{p,p}{\}}}
\pysigstopmultiline
\pysigstopsignatures
\sphinxAtStartPar
mean of the squared energy deposit in the \sphinxcode{\sphinxupquote{gap}}

\end{fulllineitems}

\index{Results::fNumSecGamma (C++ member)@\spxentry{Results::fNumSecGamma}\spxextra{C++ member}}

\begin{fulllineitems}
\phantomsection\label{\detokenize{Simulation/SimulationCodeDoc:_CPPv4N7Results12fNumSecGammaE}}
\pysigstartsignatures
\pysigstartmultiline
\pysigline{\phantomsection\label{\detokenize{Simulation/SimulationCodeDoc:struct_results_1a47beba7267414de3f4f49b24ec415ad0}}\DUrole{kt,kt}{double}\DUrole{w,w}{  }\sphinxbfcode{\sphinxupquote{\DUrole{n,n}{fNumSecGamma}}}\DUrole{w,w}{  }\DUrole{p,p}{=}\DUrole{w,w}{  }\DUrole{p,p}{\{}\DUrole{m,m}{0.0}\DUrole{p,p}{\}}}
\pysigstopmultiline
\pysigstopsignatures
\sphinxAtStartPar
mean number of the produced secondary \(\gamma\) particles 

\end{fulllineitems}

\index{Results::fNumSecGamma2 (C++ member)@\spxentry{Results::fNumSecGamma2}\spxextra{C++ member}}

\begin{fulllineitems}
\phantomsection\label{\detokenize{Simulation/SimulationCodeDoc:_CPPv4N7Results13fNumSecGamma2E}}
\pysigstartsignatures
\pysigstartmultiline
\pysigline{\phantomsection\label{\detokenize{Simulation/SimulationCodeDoc:struct_results_1a3ca7616ea0ea716906d5053182c959c3}}\DUrole{kt,kt}{double}\DUrole{w,w}{  }\sphinxbfcode{\sphinxupquote{\DUrole{n,n}{fNumSecGamma2}}}\DUrole{w,w}{  }\DUrole{p,p}{=}\DUrole{w,w}{  }\DUrole{p,p}{\{}\DUrole{m,m}{0.0}\DUrole{p,p}{\}}}
\pysigstopmultiline
\pysigstopsignatures
\sphinxAtStartPar
mean of the squared number of produced secondary \(\gamma\) particles 

\end{fulllineitems}

\index{Results::fNumSecElectron (C++ member)@\spxentry{Results::fNumSecElectron}\spxextra{C++ member}}

\begin{fulllineitems}
\phantomsection\label{\detokenize{Simulation/SimulationCodeDoc:_CPPv4N7Results15fNumSecElectronE}}
\pysigstartsignatures
\pysigstartmultiline
\pysigline{\phantomsection\label{\detokenize{Simulation/SimulationCodeDoc:struct_results_1a1e2f826da8897f45b17f9a3ab9debfa3}}\DUrole{kt,kt}{double}\DUrole{w,w}{  }\sphinxbfcode{\sphinxupquote{\DUrole{n,n}{fNumSecElectron}}}\DUrole{w,w}{  }\DUrole{p,p}{=}\DUrole{w,w}{  }\DUrole{p,p}{\{}\DUrole{m,m}{0.0}\DUrole{p,p}{\}}}
\pysigstopmultiline
\pysigstopsignatures
\sphinxAtStartPar
mean number of the produced secondary \(e^-\) particles 

\end{fulllineitems}

\index{Results::fNumSecElectron2 (C++ member)@\spxentry{Results::fNumSecElectron2}\spxextra{C++ member}}

\begin{fulllineitems}
\phantomsection\label{\detokenize{Simulation/SimulationCodeDoc:_CPPv4N7Results16fNumSecElectron2E}}
\pysigstartsignatures
\pysigstartmultiline
\pysigline{\phantomsection\label{\detokenize{Simulation/SimulationCodeDoc:struct_results_1a548c3fab61b24d09d868341d2b2a587c}}\DUrole{kt,kt}{double}\DUrole{w,w}{  }\sphinxbfcode{\sphinxupquote{\DUrole{n,n}{fNumSecElectron2}}}\DUrole{w,w}{  }\DUrole{p,p}{=}\DUrole{w,w}{  }\DUrole{p,p}{\{}\DUrole{m,m}{0.0}\DUrole{p,p}{\}}}
\pysigstopmultiline
\pysigstopsignatures
\sphinxAtStartPar
mean of the squared number of produced secondary \(e^-\) particles 

\end{fulllineitems}

\index{Results::fNumSecPositron (C++ member)@\spxentry{Results::fNumSecPositron}\spxextra{C++ member}}

\begin{fulllineitems}
\phantomsection\label{\detokenize{Simulation/SimulationCodeDoc:_CPPv4N7Results15fNumSecPositronE}}
\pysigstartsignatures
\pysigstartmultiline
\pysigline{\phantomsection\label{\detokenize{Simulation/SimulationCodeDoc:struct_results_1a9ff96ee9ba078c770be81e807fce7fb3}}\DUrole{kt,kt}{double}\DUrole{w,w}{  }\sphinxbfcode{\sphinxupquote{\DUrole{n,n}{fNumSecPositron}}}\DUrole{w,w}{  }\DUrole{p,p}{=}\DUrole{w,w}{  }\DUrole{p,p}{\{}\DUrole{m,m}{0.0}\DUrole{p,p}{\}}}
\pysigstopmultiline
\pysigstopsignatures
\sphinxAtStartPar
mean number of the produced secondary \(e^+\) particles 

\end{fulllineitems}

\index{Results::fNumSecPositron2 (C++ member)@\spxentry{Results::fNumSecPositron2}\spxextra{C++ member}}

\begin{fulllineitems}
\phantomsection\label{\detokenize{Simulation/SimulationCodeDoc:_CPPv4N7Results16fNumSecPositron2E}}
\pysigstartsignatures
\pysigstartmultiline
\pysigline{\phantomsection\label{\detokenize{Simulation/SimulationCodeDoc:struct_results_1a54033a6b6e44c31928bf3f47002c2dbf}}\DUrole{kt,kt}{double}\DUrole{w,w}{  }\sphinxbfcode{\sphinxupquote{\DUrole{n,n}{fNumSecPositron2}}}\DUrole{w,w}{  }\DUrole{p,p}{=}\DUrole{w,w}{  }\DUrole{p,p}{\{}\DUrole{m,m}{0.0}\DUrole{p,p}{\}}}
\pysigstopmultiline
\pysigstopsignatures
\sphinxAtStartPar
mean of the squared number of produced secondary \(e^+\) particles 

\end{fulllineitems}

\index{Results::fNumStepsGamma (C++ member)@\spxentry{Results::fNumStepsGamma}\spxextra{C++ member}}

\begin{fulllineitems}
\phantomsection\label{\detokenize{Simulation/SimulationCodeDoc:_CPPv4N7Results14fNumStepsGammaE}}
\pysigstartsignatures
\pysigstartmultiline
\pysigline{\phantomsection\label{\detokenize{Simulation/SimulationCodeDoc:struct_results_1a40a1269864fb39cb90b41eb20543b812}}\DUrole{kt,kt}{double}\DUrole{w,w}{  }\sphinxbfcode{\sphinxupquote{\DUrole{n,n}{fNumStepsGamma}}}\DUrole{w,w}{  }\DUrole{p,p}{=}\DUrole{w,w}{  }\DUrole{p,p}{\{}\DUrole{m,m}{0.0}\DUrole{p,p}{\}}}
\pysigstopmultiline
\pysigstopsignatures
\sphinxAtStartPar
mean number of \(\gamma\) steps in the entire calorimeter 

\end{fulllineitems}

\index{Results::fNumStepsGamma2 (C++ member)@\spxentry{Results::fNumStepsGamma2}\spxextra{C++ member}}

\begin{fulllineitems}
\phantomsection\label{\detokenize{Simulation/SimulationCodeDoc:_CPPv4N7Results15fNumStepsGamma2E}}
\pysigstartsignatures
\pysigstartmultiline
\pysigline{\phantomsection\label{\detokenize{Simulation/SimulationCodeDoc:struct_results_1aa7e6a1539475aaf2c03219ae4e592948}}\DUrole{kt,kt}{double}\DUrole{w,w}{  }\sphinxbfcode{\sphinxupquote{\DUrole{n,n}{fNumStepsGamma2}}}\DUrole{w,w}{  }\DUrole{p,p}{=}\DUrole{w,w}{  }\DUrole{p,p}{\{}\DUrole{m,m}{0.0}\DUrole{p,p}{\}}}
\pysigstopmultiline
\pysigstopsignatures
\sphinxAtStartPar
mean of the squared number of \(\gamma\) steps in the entire calorimeter 

\end{fulllineitems}

\index{Results::fNumStepsElPos (C++ member)@\spxentry{Results::fNumStepsElPos}\spxextra{C++ member}}

\begin{fulllineitems}
\phantomsection\label{\detokenize{Simulation/SimulationCodeDoc:_CPPv4N7Results14fNumStepsElPosE}}
\pysigstartsignatures
\pysigstartmultiline
\pysigline{\phantomsection\label{\detokenize{Simulation/SimulationCodeDoc:struct_results_1a155d60e512b400067017049605b6bf92}}\DUrole{kt,kt}{double}\DUrole{w,w}{  }\sphinxbfcode{\sphinxupquote{\DUrole{n,n}{fNumStepsElPos}}}\DUrole{w,w}{  }\DUrole{p,p}{=}\DUrole{w,w}{  }\DUrole{p,p}{\{}\DUrole{m,m}{0.0}\DUrole{p,p}{\}}}
\pysigstopmultiline
\pysigstopsignatures
\sphinxAtStartPar
mean number of \(e^-/e^+\) steps in the entire calorimeter 

\end{fulllineitems}

\index{Results::fNumStepsElPos2 (C++ member)@\spxentry{Results::fNumStepsElPos2}\spxextra{C++ member}}

\begin{fulllineitems}
\phantomsection\label{\detokenize{Simulation/SimulationCodeDoc:_CPPv4N7Results15fNumStepsElPos2E}}
\pysigstartsignatures
\pysigstartmultiline
\pysigline{\phantomsection\label{\detokenize{Simulation/SimulationCodeDoc:struct_results_1a931fd090154f4d3298a74fe488034456}}\DUrole{kt,kt}{double}\DUrole{w,w}{  }\sphinxbfcode{\sphinxupquote{\DUrole{n,n}{fNumStepsElPos2}}}\DUrole{w,w}{  }\DUrole{p,p}{=}\DUrole{w,w}{  }\DUrole{p,p}{\{}\DUrole{m,m}{0.0}\DUrole{p,p}{\}}}
\pysigstopmultiline
\pysigstopsignatures
\sphinxAtStartPar
mean of the squared number of \(e^-/e^+\) steps in the entire calorimeter 

\end{fulllineitems}

\index{Results::fPerEventRes (C++ member)@\spxentry{Results::fPerEventRes}\spxextra{C++ member}}

\begin{fulllineitems}
\phantomsection\label{\detokenize{Simulation/SimulationCodeDoc:_CPPv4N7Results12fPerEventResE}}
\pysigstartsignatures
\pysigstartmultiline
\pysigline{\phantomsection\label{\detokenize{Simulation/SimulationCodeDoc:struct_results_1ab17af8cb8b09a2c55f3b2d7d4c8f07a9}}{\hyperref[\detokenize{Simulation/SimulationCodeDoc:_CPPv415ResultsPerEvent}]{\sphinxcrossref{\DUrole{n,n,n}{ResultsPerEvent}}}}\DUrole{w,w}{  }\sphinxbfcode{\sphinxupquote{\DUrole{n,n}{fPerEventRes}}}}
\pysigstopmultiline
\pysigstopsignatures
\sphinxAtStartPar
data structure to accumulate results during a single event 

\end{fulllineitems}


\end{sphinxuseclass}
\end{fulllineitems}

\index{Hist (C++ class)@\spxentry{Hist}\spxextra{C++ class}}

\begin{fulllineitems}
\phantomsection\label{\detokenize{Simulation/SimulationCodeDoc:_CPPv44Hist}}
\pysigstartsignatures
\pysigstartmultiline
\pysigline{\phantomsection\label{\detokenize{Simulation/SimulationCodeDoc:class_hist}}\DUrole{k,k}{class}\DUrole{w,w}{  }\sphinxbfcode{\sphinxupquote{\DUrole{n,n}{Hist}}}}
\pysigstopmultiline
\pysigstopsignatures
\sphinxAtStartPar
A simple histogram only to collect some data during the simulation. 

\sphinxAtStartPar
\begin{description}
\sphinxlineitem{\sphinxstylestrong{Author}}
\sphinxAtStartPar
M. Novak 

\sphinxlineitem{\sphinxstylestrong{Date}}
\sphinxAtStartPar
July 2023 

\end{description}


\begin{sphinxuseclass}{breathe-sectiondef}\subsubsection*{Public Functions}
\index{Hist::Hist (C++ function)@\spxentry{Hist::Hist}\spxextra{C++ function}}

\begin{fulllineitems}
\phantomsection\label{\detokenize{Simulation/SimulationCodeDoc:_CPPv4N4Hist4HistERKNSt6stringEddi}}
\pysigstartsignatures
\pysigstartmultiline
\pysiglinewithargsret{\phantomsection\label{\detokenize{Simulation/SimulationCodeDoc:class_hist_1a126aa63a703e1143ead88cbb18432759}}\sphinxbfcode{\sphinxupquote{\DUrole{n,n}{Hist}}}}{\DUrole{k,k}{const}\DUrole{w,w}{  }\DUrole{n,n,n}{std}\DUrole{p,p}{::}\DUrole{n,n,n}{string}\DUrole{w,w}{  }\DUrole{p,p}{\&}\DUrole{n,sig-param,n}{filename}\sphinxparamcomma \DUrole{kt,kt}{double}\DUrole{w,w}{  }\DUrole{n,sig-param,n}{min}\sphinxparamcomma \DUrole{kt,kt}{double}\DUrole{w,w}{  }\DUrole{n,sig-param,n}{max}\sphinxparamcomma \DUrole{kt,kt}{int}\DUrole{w,w}{  }\DUrole{n,sig-param,n}{numbin}}{}
\pysigstopmultiline
\pysigstopsignatures
\sphinxAtStartPar
Constructor. 

\sphinxAtStartPar
\begin{quote}\begin{description}
\sphinxlineitem{param filename}
\sphinxAtStartPar
String to be used as file name when writing into file. 

\sphinxlineitem{param min}
\sphinxAtStartPar
Minimum bin value. 

\sphinxlineitem{param max}
\sphinxAtStartPar
Maximum bin value. 

\sphinxlineitem{param numbin}
\sphinxAtStartPar
Number of bins required between \sphinxcode{\sphinxupquote{min}} and \sphinxcode{\sphinxupquote{max}}. 

\end{description}\end{quote}


\end{fulllineitems}

\index{Hist::Hist (C++ function)@\spxentry{Hist::Hist}\spxextra{C++ function}}

\begin{fulllineitems}
\phantomsection\label{\detokenize{Simulation/SimulationCodeDoc:_CPPv4N4Hist4HistERKNSt6stringEddd}}
\pysigstartsignatures
\pysigstartmultiline
\pysiglinewithargsret{\phantomsection\label{\detokenize{Simulation/SimulationCodeDoc:class_hist_1a5aeb6c2d0e4967d732427cfde67f2dce}}\sphinxbfcode{\sphinxupquote{\DUrole{n,n}{Hist}}}}{\DUrole{k,k}{const}\DUrole{w,w}{  }\DUrole{n,n,n}{std}\DUrole{p,p}{::}\DUrole{n,n,n}{string}\DUrole{w,w}{  }\DUrole{p,p}{\&}\DUrole{n,sig-param,n}{filename}\sphinxparamcomma \DUrole{kt,kt}{double}\DUrole{w,w}{  }\DUrole{n,sig-param,n}{min}\sphinxparamcomma \DUrole{kt,kt}{double}\DUrole{w,w}{  }\DUrole{n,sig-param,n}{max}\sphinxparamcomma \DUrole{kt,kt}{double}\DUrole{w,w}{  }\DUrole{n,sig-param,n}{delta}}{}
\pysigstopmultiline
\pysigstopsignatures
\sphinxAtStartPar
Constructor. 

\sphinxAtStartPar
\begin{quote}\begin{description}
\sphinxlineitem{param filename}
\sphinxAtStartPar
String to be used as file name when writing into file. 

\sphinxlineitem{param min}
\sphinxAtStartPar
Minimum bin value. 

\sphinxlineitem{param max}
\sphinxAtStartPar
Maximum bin value. 

\sphinxlineitem{param delta}
\sphinxAtStartPar
Required width of a bin. 

\end{description}\end{quote}


\end{fulllineitems}

\index{Hist::Hist (C++ function)@\spxentry{Hist::Hist}\spxextra{C++ function}}

\begin{fulllineitems}
\phantomsection\label{\detokenize{Simulation/SimulationCodeDoc:_CPPv4N4Hist4HistEv}}
\pysigstartsignatures
\pysigstartmultiline
\pysiglinewithargsret{\phantomsection\label{\detokenize{Simulation/SimulationCodeDoc:class_hist_1a1d872656e56cbe3d44fa69ca110bc68a}}\sphinxbfcode{\sphinxupquote{\DUrole{n,n}{Hist}}}}{}{}
\pysigstopmultiline
\pysigstopsignatures
\sphinxAtStartPar
Default constructor. 

\end{fulllineitems}

\index{Hist::\textasciitilde{}Hist (C++ function)@\spxentry{Hist::\textasciitilde{}Hist}\spxextra{C++ function}}

\begin{fulllineitems}
\phantomsection\label{\detokenize{Simulation/SimulationCodeDoc:_CPPv4N4HistD0Ev}}
\pysigstartsignatures
\pysigstartmultiline
\pysiglinewithargsret{\phantomsection\label{\detokenize{Simulation/SimulationCodeDoc:class_hist_1a33efa5f5905fa72c4ea41d416c7d5456}}\DUrole{k,k}{inline}\DUrole{w,w}{  }\sphinxbfcode{\sphinxupquote{\DUrole{n,n}{\textasciitilde{}Hist}}}}{}{}
\pysigstopmultiline
\pysigstopsignatures
\sphinxAtStartPar
Destructor. 

\end{fulllineitems}

\index{Hist::Initialize (C++ function)@\spxentry{Hist::Initialize}\spxextra{C++ function}}

\begin{fulllineitems}
\phantomsection\label{\detokenize{Simulation/SimulationCodeDoc:_CPPv4N4Hist10InitializeEv}}
\pysigstartsignatures
\pysigstartmultiline
\pysiglinewithargsret{\phantomsection\label{\detokenize{Simulation/SimulationCodeDoc:class_hist_1a79f2da3286205ec4cb025cddfd307f8a}}\DUrole{kt,kt}{void}\DUrole{w,w}{  }\sphinxbfcode{\sphinxupquote{\DUrole{n,n}{Initialize}}}}{}{}
\pysigstopmultiline
\pysigstopsignatures
\sphinxAtStartPar
Auxiliary method to setup the initial state of the histogram. 

\end{fulllineitems}

\index{Hist::ReSet (C++ function)@\spxentry{Hist::ReSet}\spxextra{C++ function}}

\begin{fulllineitems}
\phantomsection\label{\detokenize{Simulation/SimulationCodeDoc:_CPPv4N4Hist5ReSetERKNSt6stringEddi}}
\pysigstartsignatures
\pysigstartmultiline
\pysiglinewithargsret{\phantomsection\label{\detokenize{Simulation/SimulationCodeDoc:class_hist_1ac6959a1bab2f319671b4e44398ebce2a}}\DUrole{kt,kt}{void}\DUrole{w,w}{  }\sphinxbfcode{\sphinxupquote{\DUrole{n,n}{ReSet}}}}{\DUrole{k,k}{const}\DUrole{w,w}{  }\DUrole{n,n,n}{std}\DUrole{p,p}{::}\DUrole{n,n,n}{string}\DUrole{w,w}{  }\DUrole{p,p}{\&}\DUrole{n,sig-param,n}{filename}\sphinxparamcomma \DUrole{kt,kt}{double}\DUrole{w,w}{  }\DUrole{n,sig-param,n}{min}\sphinxparamcomma \DUrole{kt,kt}{double}\DUrole{w,w}{  }\DUrole{n,sig-param,n}{max}\sphinxparamcomma \DUrole{kt,kt}{int}\DUrole{w,w}{  }\DUrole{n,sig-param,n}{numbins}}{}
\pysigstopmultiline
\pysigstopsignatures
\sphinxAtStartPar
Method to modify the properties of the histogram. 

\sphinxAtStartPar
\begin{quote}\begin{description}
\sphinxlineitem{param filename}
\sphinxAtStartPar
The new name of the filename. 

\sphinxlineitem{param min}
\sphinxAtStartPar
The new minimum bin value. 

\sphinxlineitem{param max}
\sphinxAtStartPar
The new maximum bin value. 

\sphinxlineitem{param numbins}
\sphinxAtStartPar
The new number of bins required between \sphinxcode{\sphinxupquote{min}} and \sphinxcode{\sphinxupquote{max}}. 

\end{description}\end{quote}


\end{fulllineitems}

\index{Hist::Fill (C++ function)@\spxentry{Hist::Fill}\spxextra{C++ function}}

\begin{fulllineitems}
\phantomsection\label{\detokenize{Simulation/SimulationCodeDoc:_CPPv4N4Hist4FillEd}}
\pysigstartsignatures
\pysigstartmultiline
\pysiglinewithargsret{\phantomsection\label{\detokenize{Simulation/SimulationCodeDoc:class_hist_1a83904f1855afc769d8cfa1be636f8172}}\DUrole{kt,kt}{void}\DUrole{w,w}{  }\sphinxbfcode{\sphinxupquote{\DUrole{n,n}{Fill}}}}{\DUrole{kt,kt}{double}\DUrole{w,w}{  }\DUrole{n,sig-param,n}{x}}{}
\pysigstopmultiline
\pysigstopsignatures
\sphinxAtStartPar
Method to populate the histogram with data: the corresponding bin content is increased by 1. 

\sphinxAtStartPar
\begin{quote}\begin{description}
\sphinxlineitem{param x}
\sphinxAtStartPar
Value to add. 

\end{description}\end{quote}


\end{fulllineitems}

\index{Hist::Fill (C++ function)@\spxentry{Hist::Fill}\spxextra{C++ function}}

\begin{fulllineitems}
\phantomsection\label{\detokenize{Simulation/SimulationCodeDoc:_CPPv4N4Hist4FillEdd}}
\pysigstartsignatures
\pysigstartmultiline
\pysiglinewithargsret{\phantomsection\label{\detokenize{Simulation/SimulationCodeDoc:class_hist_1a1ff2822e4272fae23a83b7f22d507a8c}}\DUrole{kt,kt}{void}\DUrole{w,w}{  }\sphinxbfcode{\sphinxupquote{\DUrole{n,n}{Fill}}}}{\DUrole{kt,kt}{double}\DUrole{w,w}{  }\DUrole{n,sig-param,n}{x}\sphinxparamcomma \DUrole{kt,kt}{double}\DUrole{w,w}{  }\DUrole{n,sig-param,n}{w}}{}
\pysigstopmultiline
\pysigstopsignatures
\sphinxAtStartPar
Method to populate the histogram with data and a weight: the corresponding bin content is increased by the weight. 

\sphinxAtStartPar
\begin{quote}\begin{description}
\sphinxlineitem{param x}
\sphinxAtStartPar
Value to add. 

\sphinxlineitem{param w}
\sphinxAtStartPar
The corresponding weight. 

\end{description}\end{quote}


\end{fulllineitems}

\index{Hist::Scale (C++ function)@\spxentry{Hist::Scale}\spxextra{C++ function}}

\begin{fulllineitems}
\phantomsection\label{\detokenize{Simulation/SimulationCodeDoc:_CPPv4N4Hist5ScaleEd}}
\pysigstartsignatures
\pysigstartmultiline
\pysiglinewithargsret{\phantomsection\label{\detokenize{Simulation/SimulationCodeDoc:class_hist_1ac6196bc73e90d6d3fa21701520f7be5a}}\DUrole{kt,kt}{void}\DUrole{w,w}{  }\sphinxbfcode{\sphinxupquote{\DUrole{n,n}{Scale}}}}{\DUrole{kt,kt}{double}\DUrole{w,w}{  }\DUrole{n,sig-param,n}{sc}}{}
\pysigstopmultiline
\pysigstopsignatures
\sphinxAtStartPar
Method to scale all bin content by a constant. 

\sphinxAtStartPar
\begin{quote}\begin{description}
\sphinxlineitem{param sc}
\sphinxAtStartPar
Scaling factor. 

\end{description}\end{quote}


\end{fulllineitems}

\index{Hist::GetNumBins (C++ function)@\spxentry{Hist::GetNumBins}\spxextra{C++ function}}

\begin{fulllineitems}
\phantomsection\label{\detokenize{Simulation/SimulationCodeDoc:_CPPv4NK4Hist10GetNumBinsEv}}
\pysigstartsignatures
\pysigstartmultiline
\pysiglinewithargsret{\phantomsection\label{\detokenize{Simulation/SimulationCodeDoc:class_hist_1a0195888d2c9f0fb12cd5eb713795b399}}\DUrole{k,k}{inline}\DUrole{w,w}{  }\DUrole{kt,kt}{int}\DUrole{w,w}{  }\sphinxbfcode{\sphinxupquote{\DUrole{n,n}{GetNumBins}}}}{}{\DUrole{w,w}{  }\DUrole{k,k}{const}}
\pysigstopmultiline
\pysigstopsignatures
\sphinxAtStartPar
Method to provide the number of bins. 

\sphinxAtStartPar
\begin{quote}\begin{description}
\sphinxlineitem{return}
\sphinxAtStartPar
Number of bins. 

\end{description}\end{quote}


\end{fulllineitems}


\end{sphinxuseclass}
\end{fulllineitems}



\subsubsection{Providing input arguments to the \sphinxstyleliteralintitle{\sphinxupquote{HepEmShow}} application}
\label{\detokenize{Simulation/SimulationCodeDoc:providing-input-arguments-to-the-hepemshow-application}}
\sphinxAtStartPar
To see all configuration option, run the application as:

\begin{sphinxVerbatim}[commandchars=\\\{\}]
\PYG{o}{.}\PYG{o}{/}\PYG{n}{HepEmShow} \PYG{o}{\PYGZhy{}}\PYG{o}{\PYGZhy{}}\PYG{n}{help}

\PYG{o}{==}\PYG{o}{=} \PYG{n}{Usage}\PYG{p}{:} \PYG{n}{HepEmShow} \PYG{p}{[}\PYG{n}{OPTIONS}\PYG{p}{]}

    \PYG{o}{\PYGZhy{}}\PYG{n}{l}  \PYG{o}{\PYGZhy{}}\PYG{o}{\PYGZhy{}}\PYG{n}{number}\PYG{o}{\PYGZhy{}}\PYG{n}{of}\PYG{o}{\PYGZhy{}}\PYG{n}{layers}      \PYG{p}{(}\PYG{n}{number} \PYG{n}{of} \PYG{n}{layers} \PYG{o+ow}{in} \PYG{n}{the} \PYG{n}{calorimeter}\PYG{p}{)}           \PYG{o}{\PYGZhy{}} \PYG{n}{default}\PYG{p}{:} \PYG{l+m+mi}{50}
    \PYG{o}{\PYGZhy{}}\PYG{n}{a}  \PYG{o}{\PYGZhy{}}\PYG{o}{\PYGZhy{}}\PYG{n}{absorber}\PYG{o}{\PYGZhy{}}\PYG{n}{thickness}    \PYG{p}{(}\PYG{o+ow}{in} \PYG{p}{[}\PYG{n}{mm}\PYG{p}{]} \PYG{n}{units}\PYG{p}{)}                                 \PYG{o}{\PYGZhy{}} \PYG{n}{default}\PYG{p}{:} \PYG{l+m+mf}{2.3}
    \PYG{o}{\PYGZhy{}}\PYG{n}{g}  \PYG{o}{\PYGZhy{}}\PYG{o}{\PYGZhy{}}\PYG{n}{gap}\PYG{o}{\PYGZhy{}}\PYG{n}{thickness}         \PYG{p}{(}\PYG{o+ow}{in} \PYG{p}{[}\PYG{n}{mm}\PYG{p}{]} \PYG{n}{units}\PYG{p}{)}                                 \PYG{o}{\PYGZhy{}} \PYG{n}{default}\PYG{p}{:} \PYG{l+m+mf}{5.7}
    \PYG{o}{\PYGZhy{}}\PYG{n}{t}  \PYG{o}{\PYGZhy{}}\PYG{o}{\PYGZhy{}}\PYG{n}{transverse}\PYG{o}{\PYGZhy{}}\PYG{n}{size}       \PYG{p}{(}\PYG{n}{of} \PYG{n}{the} \PYG{n}{calorimeter} \PYG{o+ow}{in} \PYG{p}{[}\PYG{n}{mm}\PYG{p}{]} \PYG{n}{units}\PYG{p}{)}              \PYG{o}{\PYGZhy{}} \PYG{n}{default}\PYG{p}{:} \PYG{l+m+mi}{400}
    \PYG{o}{\PYGZhy{}}\PYG{n}{p}  \PYG{o}{\PYGZhy{}}\PYG{o}{\PYGZhy{}}\PYG{n}{primary}\PYG{o}{\PYGZhy{}}\PYG{n}{particle}      \PYG{p}{(}\PYG{n}{possible} \PYG{n}{particle} \PYG{n}{names}\PYG{p}{:} \PYG{n}{e}\PYG{o}{\PYGZhy{}}\PYG{p}{,} \PYG{n}{e}\PYG{o}{+} \PYG{o+ow}{and} \PYG{n}{gamma}\PYG{p}{)}     \PYG{o}{\PYGZhy{}} \PYG{n}{default}\PYG{p}{:} \PYG{n}{e}\PYG{o}{\PYGZhy{}}
    \PYG{o}{\PYGZhy{}}\PYG{n}{e}  \PYG{o}{\PYGZhy{}}\PYG{o}{\PYGZhy{}}\PYG{n}{primary}\PYG{o}{\PYGZhy{}}\PYG{n}{energy}        \PYG{p}{(}\PYG{o+ow}{in} \PYG{p}{[}\PYG{n}{MeV}\PYG{p}{]} \PYG{n}{units}\PYG{p}{)}                                \PYG{o}{\PYGZhy{}} \PYG{n}{default}\PYG{p}{:} \PYG{l+m+mi}{10} \PYG{l+m+mi}{000}
    \PYG{o}{\PYGZhy{}}\PYG{n}{n}  \PYG{o}{\PYGZhy{}}\PYG{o}{\PYGZhy{}}\PYG{n}{number}\PYG{o}{\PYGZhy{}}\PYG{n}{of}\PYG{o}{\PYGZhy{}}\PYG{n}{events}      \PYG{p}{(}\PYG{n}{number} \PYG{n}{of} \PYG{n}{primary} \PYG{n}{events} \PYG{n}{to} \PYG{n}{simulate}\PYG{p}{)}          \PYG{o}{\PYGZhy{}} \PYG{n}{default}\PYG{p}{:} \PYG{l+m+mi}{1000}
    \PYG{o}{\PYGZhy{}}\PYG{n}{s}  \PYG{o}{\PYGZhy{}}\PYG{o}{\PYGZhy{}}\PYG{n}{random}\PYG{o}{\PYGZhy{}}\PYG{n}{seed}                                                           \PYG{o}{\PYGZhy{}} \PYG{n}{default}\PYG{p}{:} \PYG{l+m+mi}{1234}
    \PYG{o}{\PYGZhy{}}\PYG{n}{d}  \PYG{o}{\PYGZhy{}}\PYG{o}{\PYGZhy{}}\PYG{n}{g4hepem}\PYG{o}{\PYGZhy{}}\PYG{n}{data}\PYG{o}{\PYGZhy{}}\PYG{n}{file}     \PYG{p}{(}\PYG{n}{the} \PYG{n}{pre}\PYG{o}{\PYGZhy{}}\PYG{n}{generated} \PYG{n}{data} \PYG{n}{file} \PYG{k}{with} \PYG{n}{its} \PYG{n}{path}\PYG{p}{)}     \PYG{o}{\PYGZhy{}} \PYG{n}{default}\PYG{p}{:} \PYG{o}{.}\PYG{o}{.}\PYG{o}{/}\PYG{n}{data}\PYG{o}{/}\PYG{n}{hepem\PYGZus{}data}
    \PYG{o}{\PYGZhy{}}\PYG{n}{v}  \PYG{o}{\PYGZhy{}}\PYG{o}{\PYGZhy{}}\PYG{n}{run}\PYG{o}{\PYGZhy{}}\PYG{n}{verbosity}         \PYG{p}{(}\PYG{n}{verbosity} \PYG{n}{of} \PYG{n}{run} \PYG{n}{infomation}\PYG{p}{:} \PYG{n}{nothing} \PYG{n}{when} \PYG{l+m+mi}{0}\PYG{p}{)}   \PYG{o}{\PYGZhy{}} \PYG{n}{default}\PYG{p}{:} \PYG{l+m+mi}{1}
    \PYG{o}{\PYGZhy{}}\PYG{n}{h}  \PYG{o}{\PYGZhy{}}\PYG{o}{\PYGZhy{}}\PYG{n}{help}
\end{sphinxVerbatim}
\index{InputParameters (C++ struct)@\spxentry{InputParameters}\spxextra{C++ struct}}

\begin{fulllineitems}
\phantomsection\label{\detokenize{Simulation/SimulationCodeDoc:_CPPv415InputParameters}}
\pysigstartsignatures
\pysigstartmultiline
\pysigline{\phantomsection\label{\detokenize{Simulation/SimulationCodeDoc:struct_input_parameters}}\DUrole{k,k}{struct}\DUrole{w,w}{  }\sphinxbfcode{\sphinxupquote{\DUrole{n,n}{InputParameters}}}}
\pysigstopmultiline
\pysigstopsignatures
\sphinxAtStartPar
A data structure that encapsulates all the possible input arguments of the \sphinxcode{\sphinxupquote{HepEmShow}} application. 

\sphinxAtStartPar
\begin{description}
\sphinxlineitem{\sphinxstylestrong{Author}}
\sphinxAtStartPar
M. Novak 

\sphinxlineitem{\sphinxstylestrong{Date}}
\sphinxAtStartPar
Aug 2023 

\end{description}


\begin{sphinxuseclass}{breathe-sectiondef}\subsubsection*{Public Functions}
\index{InputParameters::InputParameters (C++ function)@\spxentry{InputParameters::InputParameters}\spxextra{C++ function}}

\begin{fulllineitems}
\phantomsection\label{\detokenize{Simulation/SimulationCodeDoc:_CPPv4N15InputParameters15InputParametersEv}}
\pysigstartsignatures
\pysigstartmultiline
\pysiglinewithargsret{\phantomsection\label{\detokenize{Simulation/SimulationCodeDoc:struct_input_parameters_1a2ede6d73636729561755f9ca986475f8}}\DUrole{k,k}{inline}\DUrole{w,w}{  }\sphinxbfcode{\sphinxupquote{\DUrole{n,n}{InputParameters}}}}{}{}
\pysigstopmultiline
\pysigstopsignatures
\sphinxAtStartPar
CTR with default values: default geometry, primary and event configuirations (see below) with pre\sphinxhyphen{}generated data files expected at \sphinxcode{\sphinxupquote{../data/hepem\_data}} relative to the \sphinxcode{\sphinxupquote{HepEmShow}} executable. 

\end{fulllineitems}


\end{sphinxuseclass}
\begin{sphinxuseclass}{breathe-sectiondef}\subsubsection*{Public Members}
\index{InputParameters::fGeometry (C++ member)@\spxentry{InputParameters::fGeometry}\spxextra{C++ member}}

\begin{fulllineitems}
\phantomsection\label{\detokenize{Simulation/SimulationCodeDoc:_CPPv4N15InputParameters9fGeometryE}}
\pysigstartsignatures
\pysigstartmultiline
\pysigline{\phantomsection\label{\detokenize{Simulation/SimulationCodeDoc:struct_input_parameters_1a921dbfbac71b688dfde11060591e69ed}}{\hyperref[\detokenize{Simulation/SimulationCodeDoc:_CPPv4N15InputParameters8GeometryE}]{\sphinxcrossref{\DUrole{n,n,n}{Geometry}}}}\DUrole{w,w}{  }\sphinxbfcode{\sphinxupquote{\DUrole{n,n}{fGeometry}}}}
\pysigstopmultiline
\pysigstopsignatures
\sphinxAtStartPar
the geometry related configuration 

\end{fulllineitems}

\index{InputParameters::fPrimaryAndEvents (C++ member)@\spxentry{InputParameters::fPrimaryAndEvents}\spxextra{C++ member}}

\begin{fulllineitems}
\phantomsection\label{\detokenize{Simulation/SimulationCodeDoc:_CPPv4N15InputParameters17fPrimaryAndEventsE}}
\pysigstartsignatures
\pysigstartmultiline
\pysigline{\phantomsection\label{\detokenize{Simulation/SimulationCodeDoc:struct_input_parameters_1a34ce91465085f18f7e76bdb5a1f2a1b3}}{\hyperref[\detokenize{Simulation/SimulationCodeDoc:_CPPv4N15InputParameters16PrimaryAndEventsE}]{\sphinxcrossref{\DUrole{n,n,n}{PrimaryAndEvents}}}}\DUrole{w,w}{  }\sphinxbfcode{\sphinxupquote{\DUrole{n,n}{fPrimaryAndEvents}}}}
\pysigstopmultiline
\pysigstopsignatures
\sphinxAtStartPar
the primary partcile and events related configuration 

\end{fulllineitems}

\index{InputParameters::fG4HepEmDataFile (C++ member)@\spxentry{InputParameters::fG4HepEmDataFile}\spxextra{C++ member}}

\begin{fulllineitems}
\phantomsection\label{\detokenize{Simulation/SimulationCodeDoc:_CPPv4N15InputParameters16fG4HepEmDataFileE}}
\pysigstartsignatures
\pysigstartmultiline
\pysigline{\phantomsection\label{\detokenize{Simulation/SimulationCodeDoc:struct_input_parameters_1a3a00664f729cff9c2ede9a82be35f9f1}}\DUrole{n,n,n}{std}\DUrole{p,p}{::}\DUrole{n,n,n}{string}\DUrole{w,w}{  }\sphinxbfcode{\sphinxupquote{\DUrole{n,n}{fG4HepEmDataFile}}}}
\pysigstopmultiline
\pysigstopsignatures
\sphinxAtStartPar
the pre\sphinxhyphen{}generated data file (with path) 

\end{fulllineitems}

\index{InputParameters::fRunVerbosity (C++ member)@\spxentry{InputParameters::fRunVerbosity}\spxextra{C++ member}}

\begin{fulllineitems}
\phantomsection\label{\detokenize{Simulation/SimulationCodeDoc:_CPPv4N15InputParameters13fRunVerbosityE}}
\pysigstartsignatures
\pysigstartmultiline
\pysigline{\phantomsection\label{\detokenize{Simulation/SimulationCodeDoc:struct_input_parameters_1ae5022264c5b3d91446c18fd975a6f742}}\DUrole{kt,kt}{int}\DUrole{w,w}{  }\sphinxbfcode{\sphinxupquote{\DUrole{n,n}{fRunVerbosity}}}}
\pysigstopmultiline
\pysigstopsignatures
\sphinxAtStartPar
level of printout verbosity duing setting up: nothing when \textless{} 1. 

\end{fulllineitems}


\end{sphinxuseclass}\index{InputParameters::Geometry (C++ struct)@\spxentry{InputParameters::Geometry}\spxextra{C++ struct}}

\begin{fulllineitems}
\phantomsection\label{\detokenize{Simulation/SimulationCodeDoc:_CPPv4N15InputParameters8GeometryE}}
\pysigstartsignatures
\pysigstartmultiline
\pysigline{\phantomsection\label{\detokenize{Simulation/SimulationCodeDoc:struct_input_parameters_1_1_geometry}}\DUrole{k,k}{struct}\DUrole{w,w}{  }\sphinxbfcode{\sphinxupquote{\DUrole{n,n}{Geometry}}}}
\pysigstopmultiline
\pysigstopsignatures
\sphinxAtStartPar
The geometry related input arguments. 

\begin{sphinxuseclass}{breathe-sectiondef}\subsubsection*{Public Functions}
\index{InputParameters::Geometry::Geometry (C++ function)@\spxentry{InputParameters::Geometry::Geometry}\spxextra{C++ function}}

\begin{fulllineitems}
\phantomsection\label{\detokenize{Simulation/SimulationCodeDoc:_CPPv4N15InputParameters8Geometry8GeometryEv}}
\pysigstartsignatures
\pysigstartmultiline
\pysiglinewithargsret{\phantomsection\label{\detokenize{Simulation/SimulationCodeDoc:struct_input_parameters_1_1_geometry_1a8ff86e44e708b2f283144e44f546e025}}\DUrole{k,k}{inline}\DUrole{w,w}{  }\sphinxbfcode{\sphinxupquote{\DUrole{n,n}{Geometry}}}}{}{}
\pysigstopmultiline
\pysigstopsignatures
\sphinxAtStartPar
CTR with default values: 50 layers of 2.3 mm absorber and 5.7 mm gap with 400 mm transvers size. 

\end{fulllineitems}


\end{sphinxuseclass}
\begin{sphinxuseclass}{breathe-sectiondef}\subsubsection*{Public Members}
\index{InputParameters::Geometry::fNumLayers (C++ member)@\spxentry{InputParameters::Geometry::fNumLayers}\spxextra{C++ member}}

\begin{fulllineitems}
\phantomsection\label{\detokenize{Simulation/SimulationCodeDoc:_CPPv4N15InputParameters8Geometry10fNumLayersE}}
\pysigstartsignatures
\pysigstartmultiline
\pysigline{\phantomsection\label{\detokenize{Simulation/SimulationCodeDoc:struct_input_parameters_1_1_geometry_1a1da1662e45b6034aa1e3e04f97bc2d03}}\DUrole{kt,kt}{int}\DUrole{w,w}{  }\sphinxbfcode{\sphinxupquote{\DUrole{n,n}{fNumLayers}}}}
\pysigstopmultiline
\pysigstopsignatures
\sphinxAtStartPar
number of layers in the calorimeter 

\end{fulllineitems}

\index{InputParameters::Geometry::fThicknessAbsorber (C++ member)@\spxentry{InputParameters::Geometry::fThicknessAbsorber}\spxextra{C++ member}}

\begin{fulllineitems}
\phantomsection\label{\detokenize{Simulation/SimulationCodeDoc:_CPPv4N15InputParameters8Geometry18fThicknessAbsorberE}}
\pysigstartsignatures
\pysigstartmultiline
\pysigline{\phantomsection\label{\detokenize{Simulation/SimulationCodeDoc:struct_input_parameters_1_1_geometry_1aaec005d9ef9029fe121d6323eb9faf76}}\DUrole{kt,kt}{double}\DUrole{w,w}{  }\sphinxbfcode{\sphinxupquote{\DUrole{n,n}{fThicknessAbsorber}}}}
\pysigstopmultiline
\pysigstopsignatures
\sphinxAtStartPar
absorber thickness along X in {[}mm{]} 

\end{fulllineitems}

\index{InputParameters::Geometry::fThicknessGap (C++ member)@\spxentry{InputParameters::Geometry::fThicknessGap}\spxextra{C++ member}}

\begin{fulllineitems}
\phantomsection\label{\detokenize{Simulation/SimulationCodeDoc:_CPPv4N15InputParameters8Geometry13fThicknessGapE}}
\pysigstartsignatures
\pysigstartmultiline
\pysigline{\phantomsection\label{\detokenize{Simulation/SimulationCodeDoc:struct_input_parameters_1_1_geometry_1a7d1ac576d4ec41f45b470fe468d097a9}}\DUrole{kt,kt}{double}\DUrole{w,w}{  }\sphinxbfcode{\sphinxupquote{\DUrole{n,n}{fThicknessGap}}}}
\pysigstopmultiline
\pysigstopsignatures
\sphinxAtStartPar
gap thickness along X in {[}mm{]} 

\end{fulllineitems}

\index{InputParameters::Geometry::fThicknessCalo (C++ member)@\spxentry{InputParameters::Geometry::fThicknessCalo}\spxextra{C++ member}}

\begin{fulllineitems}
\phantomsection\label{\detokenize{Simulation/SimulationCodeDoc:_CPPv4N15InputParameters8Geometry14fThicknessCaloE}}
\pysigstartsignatures
\pysigstartmultiline
\pysigline{\phantomsection\label{\detokenize{Simulation/SimulationCodeDoc:struct_input_parameters_1_1_geometry_1a50bbe4e295d36cb85cd32312863c88f1}}\DUrole{kt,kt}{double}\DUrole{w,w}{  }\sphinxbfcode{\sphinxupquote{\DUrole{n,n}{fThicknessCalo}}}}
\pysigstopmultiline
\pysigstopsignatures
\sphinxAtStartPar
calorimeter thickness along X {[}mm{]} ONLY if number of layers is zero 

\end{fulllineitems}

\index{InputParameters::Geometry::fSizeTransverse (C++ member)@\spxentry{InputParameters::Geometry::fSizeTransverse}\spxextra{C++ member}}

\begin{fulllineitems}
\phantomsection\label{\detokenize{Simulation/SimulationCodeDoc:_CPPv4N15InputParameters8Geometry15fSizeTransverseE}}
\pysigstartsignatures
\pysigstartmultiline
\pysigline{\phantomsection\label{\detokenize{Simulation/SimulationCodeDoc:struct_input_parameters_1_1_geometry_1afc52a39c40d78b6939cc676088fa42d4}}\DUrole{kt,kt}{double}\DUrole{w,w}{  }\sphinxbfcode{\sphinxupquote{\DUrole{n,n}{fSizeTransverse}}}}
\pysigstopmultiline
\pysigstopsignatures
\sphinxAtStartPar
calorimeter full size along YZ in {[}mm{]} 

\end{fulllineitems}


\end{sphinxuseclass}
\end{fulllineitems}

\index{InputParameters::PrimaryAndEvents (C++ struct)@\spxentry{InputParameters::PrimaryAndEvents}\spxextra{C++ struct}}

\begin{fulllineitems}
\phantomsection\label{\detokenize{Simulation/SimulationCodeDoc:_CPPv4N15InputParameters16PrimaryAndEventsE}}
\pysigstartsignatures
\pysigstartmultiline
\pysigline{\phantomsection\label{\detokenize{Simulation/SimulationCodeDoc:struct_input_parameters_1_1_primary_and_events}}\DUrole{k,k}{struct}\DUrole{w,w}{  }\sphinxbfcode{\sphinxupquote{\DUrole{n,n}{PrimaryAndEvents}}}}
\pysigstopmultiline
\pysigstopsignatures
\sphinxAtStartPar
The primary partcile and events related input arguments. 

\begin{sphinxuseclass}{breathe-sectiondef}\subsubsection*{Public Functions}
\index{InputParameters::PrimaryAndEvents::PrimaryAndEvents (C++ function)@\spxentry{InputParameters::PrimaryAndEvents::PrimaryAndEvents}\spxextra{C++ function}}

\begin{fulllineitems}
\phantomsection\label{\detokenize{Simulation/SimulationCodeDoc:_CPPv4N15InputParameters16PrimaryAndEvents16PrimaryAndEventsEv}}
\pysigstartsignatures
\pysigstartmultiline
\pysiglinewithargsret{\phantomsection\label{\detokenize{Simulation/SimulationCodeDoc:struct_input_parameters_1_1_primary_and_events_1a266b39bff4d4717797753c9bce2d6ba1}}\DUrole{k,k}{inline}\DUrole{w,w}{  }\sphinxbfcode{\sphinxupquote{\DUrole{n,n}{PrimaryAndEvents}}}}{}{}
\pysigstopmultiline
\pysigstopsignatures
\sphinxAtStartPar
CTR with default values: simulate 1000 events, starting with an electron of 10 GeV each (do not report progress). 

\end{fulllineitems}


\end{sphinxuseclass}
\begin{sphinxuseclass}{breathe-sectiondef}\subsubsection*{Public Members}
\index{InputParameters::PrimaryAndEvents::fParticleName (C++ member)@\spxentry{InputParameters::PrimaryAndEvents::fParticleName}\spxextra{C++ member}}

\begin{fulllineitems}
\phantomsection\label{\detokenize{Simulation/SimulationCodeDoc:_CPPv4N15InputParameters16PrimaryAndEvents13fParticleNameE}}
\pysigstartsignatures
\pysigstartmultiline
\pysigline{\phantomsection\label{\detokenize{Simulation/SimulationCodeDoc:struct_input_parameters_1_1_primary_and_events_1abe7bda9ce7c54ed3d23dbf9dced99fb4}}\DUrole{n,n,n}{std}\DUrole{p,p}{::}\DUrole{n,n,n}{string}\DUrole{w,w}{  }\sphinxbfcode{\sphinxupquote{\DUrole{n,n}{fParticleName}}}}
\pysigstopmultiline
\pysigstopsignatures
\sphinxAtStartPar
primary particle name: \{“e\sphinxhyphen{}”, “e+” or “gamma”\} 

\end{fulllineitems}

\index{InputParameters::PrimaryAndEvents::fParticleEnergy (C++ member)@\spxentry{InputParameters::PrimaryAndEvents::fParticleEnergy}\spxextra{C++ member}}

\begin{fulllineitems}
\phantomsection\label{\detokenize{Simulation/SimulationCodeDoc:_CPPv4N15InputParameters16PrimaryAndEvents15fParticleEnergyE}}
\pysigstartsignatures
\pysigstartmultiline
\pysigline{\phantomsection\label{\detokenize{Simulation/SimulationCodeDoc:struct_input_parameters_1_1_primary_and_events_1a9f8bb9a2b488ca127faf32c26ebbf019}}\DUrole{kt,kt}{double}\DUrole{w,w}{  }\sphinxbfcode{\sphinxupquote{\DUrole{n,n}{fParticleEnergy}}}}
\pysigstopmultiline
\pysigstopsignatures
\sphinxAtStartPar
primary particle energy in {[}MeV{]} 

\end{fulllineitems}

\index{InputParameters::PrimaryAndEvents::fNumEvents (C++ member)@\spxentry{InputParameters::PrimaryAndEvents::fNumEvents}\spxextra{C++ member}}

\begin{fulllineitems}
\phantomsection\label{\detokenize{Simulation/SimulationCodeDoc:_CPPv4N15InputParameters16PrimaryAndEvents10fNumEventsE}}
\pysigstartsignatures
\pysigstartmultiline
\pysigline{\phantomsection\label{\detokenize{Simulation/SimulationCodeDoc:struct_input_parameters_1_1_primary_and_events_1a55e5297ef8e8203d4942afff69a61ba8}}\DUrole{kt,kt}{int}\DUrole{w,w}{  }\sphinxbfcode{\sphinxupquote{\DUrole{n,n}{fNumEvents}}}}
\pysigstopmultiline
\pysigstopsignatures
\sphinxAtStartPar
number of events to simulate (each will start with a single primary) 

\end{fulllineitems}

\index{InputParameters::PrimaryAndEvents::fRandomSeed (C++ member)@\spxentry{InputParameters::PrimaryAndEvents::fRandomSeed}\spxextra{C++ member}}

\begin{fulllineitems}
\phantomsection\label{\detokenize{Simulation/SimulationCodeDoc:_CPPv4N15InputParameters16PrimaryAndEvents11fRandomSeedE}}
\pysigstartsignatures
\pysigstartmultiline
\pysigline{\phantomsection\label{\detokenize{Simulation/SimulationCodeDoc:struct_input_parameters_1_1_primary_and_events_1a9e856f4c65b7c54b585712a80e7e27d2}}\DUrole{kt,kt}{double}\DUrole{w,w}{  }\sphinxbfcode{\sphinxupquote{\DUrole{n,n}{fRandomSeed}}}}
\pysigstopmultiline
\pysigstopsignatures
\sphinxAtStartPar
seed for the random number generator 

\end{fulllineitems}


\end{sphinxuseclass}
\end{fulllineitems}


\end{fulllineitems}



\subsubsection{Event loop, stepping loops and the track stack}
\label{\detokenize{Simulation/SimulationCodeDoc:event-loop-stepping-loops-and-the-track-stack}}\index{EventLoop (C++ class)@\spxentry{EventLoop}\spxextra{C++ class}}

\begin{fulllineitems}
\phantomsection\label{\detokenize{Simulation/SimulationCodeDoc:_CPPv49EventLoop}}
\pysigstartsignatures
\pysigstartmultiline
\pysigline{\phantomsection\label{\detokenize{Simulation/SimulationCodeDoc:class_event_loop}}\DUrole{k,k}{class}\DUrole{w,w}{  }\sphinxbfcode{\sphinxupquote{\DUrole{n,n}{EventLoop}}}}
\pysigstopmultiline
\pysigstopsignatures
\sphinxAtStartPar
Event loop for simulating the required number of primary tracks/events. 

\sphinxAtStartPar
\begin{description}
\sphinxlineitem{\sphinxstylestrong{Author}}
\sphinxAtStartPar
M. Novak 

\sphinxlineitem{\sphinxstylestrong{Date}}
\sphinxAtStartPar
July 2023

\end{description}


\sphinxAtStartPar
The \sphinxcode{\sphinxupquote{{\hyperref[\detokenize{Simulation/SimulationCodeDoc:class_event_loop_1a7b1d4b512c5fa676bd990cfaa6d561c9}]{\sphinxcrossref{\DUrole{std,std-ref}{EventLoop::ProcessEvents()}}}}}} method is responsible to generate track(s) for the required number of events and simulate the histories of all primary and their secondary tracks. 

\begin{sphinxuseclass}{breathe-sectiondef}\subsubsection*{Public Static Functions}
\index{EventLoop::ProcessEvents (C++ function)@\spxentry{EventLoop::ProcessEvents}\spxextra{C++ function}}

\begin{fulllineitems}
\phantomsection\label{\detokenize{Simulation/SimulationCodeDoc:_CPPv4N9EventLoop13ProcessEventsER13G4HepEmTLDataR12G4HepEmStateR16PrimaryGeneratorR8GeometryR7Resultsii}}
\pysigstartsignatures
\pysigstartmultiline
\pysiglinewithargsret{\phantomsection\label{\detokenize{Simulation/SimulationCodeDoc:class_event_loop_1a7b1d4b512c5fa676bd990cfaa6d561c9}}\DUrole{k,k}{static}\DUrole{w,w}{  }\DUrole{kt,kt}{void}\DUrole{w,w}{  }\sphinxbfcode{\sphinxupquote{\DUrole{n,n}{ProcessEvents}}}}{\DUrole{n,n,n}{G4HepEmTLData}\DUrole{w,w}{  }\DUrole{p,p}{\&}\DUrole{n,sig-param,n}{theTLData}\sphinxparamcomma \DUrole{n,n,n}{G4HepEmState}\DUrole{w,w}{  }\DUrole{p,p}{\&}\DUrole{n,sig-param,n}{theState}\sphinxparamcomma {\hyperref[\detokenize{Simulation/SimulationCodeDoc:_CPPv416PrimaryGenerator}]{\sphinxcrossref{\DUrole{n,n,n}{PrimaryGenerator}}}}\DUrole{w,w}{  }\DUrole{p,p}{\&}\DUrole{n,sig-param,n}{thePrimaryGenerator}\sphinxparamcomma {\hyperref[\detokenize{Simulation/SimulationCodeDoc:_CPPv48Geometry}]{\sphinxcrossref{\DUrole{n,n,n}{Geometry}}}}\DUrole{w,w}{  }\DUrole{p,p}{\&}\DUrole{n,sig-param,n}{theGeometry}\sphinxparamcomma {\hyperref[\detokenize{Simulation/SimulationCodeDoc:_CPPv47Results}]{\sphinxcrossref{\DUrole{n,n,n}{Results}}}}\DUrole{w,w}{  }\DUrole{p,p}{\&}\DUrole{n,sig-param,n}{theResult}\sphinxparamcomma \DUrole{kt,kt}{int}\DUrole{w,w}{  }\DUrole{n,sig-param,n}{numEventToSimulate}\sphinxparamcomma \DUrole{kt,kt}{int}\DUrole{w,w}{  }\DUrole{n,sig-param,n}{verbosity}}{}
\pysigstopmultiline
\pysigstopsignatures
\sphinxAtStartPar
Generates and simulates the required number of events. 

\sphinxAtStartPar
Events, i.e. primary track(s) are generated by using the input \sphinxcode{\sphinxupquote{{\hyperref[\detokenize{Simulation/SimulationCodeDoc:class_primary_generator}]{\sphinxcrossref{\DUrole{std,std-ref}{PrimaryGenerator}}}}}}. At the beginning of each event, the \sphinxcode{\sphinxupquote{{\hyperref[\detokenize{Simulation/SimulationCodeDoc:class_primary_generator}]{\sphinxcrossref{\DUrole{std,std-ref}{PrimaryGenerator}}}}}} is used to generate the primary track(s) that belong to the actual event. Note, that we have only one primary track per\sphinxhyphen{}event at the moment. The generated primary track(s) is inserted/pushed into the \sphinxcode{\sphinxupquote{{\hyperref[\detokenize{Simulation/SimulationCodeDoc:class_track_stack}]{\sphinxcrossref{\DUrole{std,std-ref}{TrackStack}}}}}} as the very first track and the simulation of the event starts. During the simulation of the event:\begin{itemize}
\item {} 
\sphinxAtStartPar
one track is popped from the stack and the appropriate \sphinxcode{\sphinxupquote{{\hyperref[\detokenize{Simulation/SimulationCodeDoc:class_stepping_loop}]{\sphinxcrossref{\DUrole{std,std-ref}{SteppingLoop}}}}}} is called to simulate its entire history ina step\sphinxhyphen{}by\sphinxhyphen{}step way

\item {} 
\sphinxAtStartPar
at the end of each simulation step, secondary tracks that are created in that step in the related physics interaction (if any), are inserted/pushed into the \sphinxcode{\sphinxupquote{{\hyperref[\detokenize{Simulation/SimulationCodeDoc:class_track_stack}]{\sphinxcrossref{\DUrole{std,std-ref}{TrackStack}}}}}} Simulation of the event is completed when the \sphinxcode{\sphinxupquote{{\hyperref[\detokenize{Simulation/SimulationCodeDoc:class_track_stack}]{\sphinxcrossref{\DUrole{std,std-ref}{TrackStack}}}}}} becomes empty. See the implementation for more details.

\end{itemize}


\sphinxAtStartPar
In order to be able to collect some infomation during the event processing, the \sphinxcode{\sphinxupquote{{\hyperref[\detokenize{Simulation/SimulationCodeDoc:class_event_loop_1af34a47f54be63c327216ab77bb0505d4}]{\sphinxcrossref{\DUrole{std,std-ref}{BeginOfEventAction()}}}}}}/\sphinxcode{\sphinxupquote{{\hyperref[\detokenize{Simulation/SimulationCodeDoc:class_event_loop_1ac63175ca75f98869d887e0b4dd72e38e}]{\sphinxcrossref{\DUrole{std,std-ref}{EndOfEventAction()}}}}}} methods are invoked before/after each event processing while the \sphinxcode{\sphinxupquote{{\hyperref[\detokenize{Simulation/SimulationCodeDoc:class_event_loop_1a4fd22ea39b677d0802d70f67e571b616}]{\sphinxcrossref{\DUrole{std,std-ref}{BeginOfTrackingAction()}}}}}}/\sphinxcode{\sphinxupquote{{\hyperref[\detokenize{Simulation/SimulationCodeDoc:class_event_loop_1ad4213d30cae150022509656095dd8f88}]{\sphinxcrossref{\DUrole{std,std-ref}{EndOfTrackingAction()}}}}}} methods are invoked before/after tracking each new track.

\sphinxAtStartPar
\begin{quote}\begin{description}
\sphinxlineitem{param theTLData}
\sphinxAtStartPar
a \sphinxcode{\sphinxupquote{G4HepEm}} specific (thread local) object primarily used to obtain all physics related information from \sphinxcode{\sphinxupquote{G4HepEm}} needed to compute a simulation step 

\sphinxlineitem{param theState}
\sphinxAtStartPar
a \sphinxcode{\sphinxupquote{G4HepEm}} specific object that stores pointers to the top level \sphinxcode{\sphinxupquote{G4HepEm}} data structure and parameters that are used by \sphinxcode{\sphinxupquote{G4HepEm}} to provide all physics related infomation needed to compute a simulation step 

\sphinxlineitem{param thePrimaryGenerator}
\sphinxAtStartPar
the primary generator that is used to generate primary track(s) at the beginning of each event (only one primary track per event in our case now) 

\sphinxlineitem{param theGeometry}
\sphinxAtStartPar
the geometry of the application in which the input track history is simulated 

\sphinxlineitem{param theResult}
\sphinxAtStartPar
the data structure that holds all the infomation needs to be collected during the simulation. 

\sphinxlineitem{param numEventToSimulate}
\sphinxAtStartPar
number of events required to be simulated 

\sphinxlineitem{param verbosity}
\sphinxAtStartPar
to control the verbosity of printouts reporting progress and state of the event processing 

\end{description}\end{quote}


\end{fulllineitems}


\end{sphinxuseclass}
\begin{sphinxuseclass}{breathe-sectiondef}\subsubsection*{Private Static Functions}
\index{EventLoop::BeginOfEventAction (C++ function)@\spxentry{EventLoop::BeginOfEventAction}\spxextra{C++ function}}

\begin{fulllineitems}
\phantomsection\label{\detokenize{Simulation/SimulationCodeDoc:_CPPv4N9EventLoop18BeginOfEventActionER7ResultsiRK12G4HepEmTrack}}
\pysigstartsignatures
\pysigstartmultiline
\pysiglinewithargsret{\phantomsection\label{\detokenize{Simulation/SimulationCodeDoc:class_event_loop_1af34a47f54be63c327216ab77bb0505d4}}\DUrole{k,k}{static}\DUrole{w,w}{  }\DUrole{kt,kt}{void}\DUrole{w,w}{  }\sphinxbfcode{\sphinxupquote{\DUrole{n,n}{BeginOfEventAction}}}}{{\hyperref[\detokenize{Simulation/SimulationCodeDoc:_CPPv47Results}]{\sphinxcrossref{\DUrole{n,n,n}{Results}}}}\DUrole{w,w}{  }\DUrole{p,p}{\&}\DUrole{n,sig-param,n}{theResult}\sphinxparamcomma \DUrole{kt,kt}{int}\DUrole{w,w}{  }\DUrole{n,sig-param,n}{eventID}\sphinxparamcomma \DUrole{k,k}{const}\DUrole{w,w}{  }\DUrole{n,n,n}{G4HepEmTrack}\DUrole{w,w}{  }\DUrole{p,p}{\&}\DUrole{n,sig-param,n}{thePrimaryTrack}}{}
\pysigstopmultiline
\pysigstopsignatures
\sphinxAtStartPar
Method invoked at the beginning of each event by passing the (single) primary track of the event. 

\end{fulllineitems}

\index{EventLoop::EndOfEventAction (C++ function)@\spxentry{EventLoop::EndOfEventAction}\spxextra{C++ function}}

\begin{fulllineitems}
\phantomsection\label{\detokenize{Simulation/SimulationCodeDoc:_CPPv4N9EventLoop16EndOfEventActionER7Resultsi}}
\pysigstartsignatures
\pysigstartmultiline
\pysiglinewithargsret{\phantomsection\label{\detokenize{Simulation/SimulationCodeDoc:class_event_loop_1ac63175ca75f98869d887e0b4dd72e38e}}\DUrole{k,k}{static}\DUrole{w,w}{  }\DUrole{kt,kt}{void}\DUrole{w,w}{  }\sphinxbfcode{\sphinxupquote{\DUrole{n,n}{EndOfEventAction}}}}{{\hyperref[\detokenize{Simulation/SimulationCodeDoc:_CPPv47Results}]{\sphinxcrossref{\DUrole{n,n,n}{Results}}}}\DUrole{w,w}{  }\DUrole{p,p}{\&}\DUrole{n,sig-param,n}{theResult}\sphinxparamcomma \DUrole{kt,kt}{int}\DUrole{w,w}{  }\DUrole{n,sig-param,n}{eventID}}{}
\pysigstopmultiline
\pysigstopsignatures
\sphinxAtStartPar
Method invoked at the end of each event. 

\end{fulllineitems}

\index{EventLoop::BeginOfTrackingAction (C++ function)@\spxentry{EventLoop::BeginOfTrackingAction}\spxextra{C++ function}}

\begin{fulllineitems}
\phantomsection\label{\detokenize{Simulation/SimulationCodeDoc:_CPPv4N9EventLoop21BeginOfTrackingActionER7ResultsR12G4HepEmTrack}}
\pysigstartsignatures
\pysigstartmultiline
\pysiglinewithargsret{\phantomsection\label{\detokenize{Simulation/SimulationCodeDoc:class_event_loop_1a4fd22ea39b677d0802d70f67e571b616}}\DUrole{k,k}{static}\DUrole{w,w}{  }\DUrole{kt,kt}{void}\DUrole{w,w}{  }\sphinxbfcode{\sphinxupquote{\DUrole{n,n}{BeginOfTrackingAction}}}}{{\hyperref[\detokenize{Simulation/SimulationCodeDoc:_CPPv47Results}]{\sphinxcrossref{\DUrole{n,n,n}{Results}}}}\DUrole{w,w}{  }\DUrole{p,p}{\&}\DUrole{n,sig-param,n}{theResult}\sphinxparamcomma \DUrole{n,n,n}{G4HepEmTrack}\DUrole{w,w}{  }\DUrole{p,p}{\&}\DUrole{n,sig-param,n}{theTrack}}{}
\pysigstopmultiline
\pysigstopsignatures
\sphinxAtStartPar
Method invoked before start tracking of a new track (provided as input argument). 

\end{fulllineitems}

\index{EventLoop::EndOfTrackingAction (C++ function)@\spxentry{EventLoop::EndOfTrackingAction}\spxextra{C++ function}}

\begin{fulllineitems}
\phantomsection\label{\detokenize{Simulation/SimulationCodeDoc:_CPPv4N9EventLoop19EndOfTrackingActionER7ResultsR12G4HepEmTrack}}
\pysigstartsignatures
\pysigstartmultiline
\pysiglinewithargsret{\phantomsection\label{\detokenize{Simulation/SimulationCodeDoc:class_event_loop_1ad4213d30cae150022509656095dd8f88}}\DUrole{k,k}{static}\DUrole{w,w}{  }\DUrole{kt,kt}{void}\DUrole{w,w}{  }\sphinxbfcode{\sphinxupquote{\DUrole{n,n}{EndOfTrackingAction}}}}{{\hyperref[\detokenize{Simulation/SimulationCodeDoc:_CPPv47Results}]{\sphinxcrossref{\DUrole{n,n,n}{Results}}}}\DUrole{w,w}{  }\DUrole{p,p}{\&}\DUrole{n,sig-param,n}{theResult}\sphinxparamcomma \DUrole{n,n,n}{G4HepEmTrack}\DUrole{w,w}{  }\DUrole{p,p}{\&}\DUrole{n,sig-param,n}{theTrack}}{}
\pysigstopmultiline
\pysigstopsignatures
\sphinxAtStartPar
Method invoked after terminating tracking of a track (provided as input argument). 

\end{fulllineitems}


\end{sphinxuseclass}
\end{fulllineitems}

\index{SteppingLoop (C++ class)@\spxentry{SteppingLoop}\spxextra{C++ class}}

\begin{fulllineitems}
\phantomsection\label{\detokenize{Simulation/SimulationCodeDoc:_CPPv412SteppingLoop}}
\pysigstartsignatures
\pysigstartmultiline
\pysigline{\phantomsection\label{\detokenize{Simulation/SimulationCodeDoc:class_stepping_loop}}\DUrole{k,k}{class}\DUrole{w,w}{  }\sphinxbfcode{\sphinxupquote{\DUrole{n,n}{SteppingLoop}}}}
\pysigstopmultiline
\pysigstopsignatures
\sphinxAtStartPar
Stepping loops for simulating \(e^-\), \(e^+\) and \(\gamma\) particle histories. 

\sphinxAtStartPar
\begin{description}
\sphinxlineitem{\sphinxstylestrong{Author}}
\sphinxAtStartPar
M. Novak 

\sphinxlineitem{\sphinxstylestrong{Date}}
\sphinxAtStartPar
July 2023

\end{description}


\sphinxAtStartPar
The stepping loops can calculate a given \(\gamma\) or \(e^-/e^+\) particle simulation history from their initial state till the end in a step\sphinxhyphen{}by\sphinxhyphen{}step way (by the \sphinxcode{\sphinxupquote{{\hyperref[\detokenize{Simulation/SimulationCodeDoc:class_stepping_loop_1a1d7aa9c14da7b327829dc318d13fc2ed}]{\sphinxcrossref{\DUrole{std,std-ref}{SteppingLoop::GammaStepper}}}}(G4HepEmTLData\&, G4HepEmState\&, {\hyperref[\detokenize{Simulation/SimulationCodeDoc:class_track_stack}]{\sphinxcrossref{\DUrole{std,std-ref}{TrackStack}}}}\&, {\hyperref[\detokenize{Simulation/SimulationCodeDoc:class_geometry}]{\sphinxcrossref{\DUrole{std,std-ref}{Geometry}}}}\&, {\hyperref[\detokenize{Simulation/SimulationCodeDoc:struct_results}]{\sphinxcrossref{\DUrole{std,std-ref}{Results}}}}\&, int)}} and \sphinxcode{\sphinxupquote{{\hyperref[\detokenize{Simulation/SimulationCodeDoc:class_stepping_loop_1ad0de81b62ac3ba13532464c2c01f1b39}]{\sphinxcrossref{\DUrole{std,std-ref}{SteppingLoop::ElectronStepper}}}}(G4HepEmTLData\&, G4HepEmState\&, {\hyperref[\detokenize{Simulation/SimulationCodeDoc:class_track_stack}]{\sphinxcrossref{\DUrole{std,std-ref}{TrackStack}}}}\&, {\hyperref[\detokenize{Simulation/SimulationCodeDoc:class_geometry}]{\sphinxcrossref{\DUrole{std,std-ref}{Geometry}}}}\&, {\hyperref[\detokenize{Simulation/SimulationCodeDoc:struct_results}]{\sphinxcrossref{\DUrole{std,std-ref}{Results}}}}\&, int)}} respectively). At each step:\begin{itemize}
\item {} 
\sphinxAtStartPar
the actual step length is calculated (accounting both the geometrical and the physics related constraints)

\item {} 
\sphinxAtStartPar
the track is moved to its post\sphinxhyphen{}step position

\item {} 
\sphinxAtStartPar
all physics related actions, happening along and/or at the post\sphinxhyphen{}step point, are performed on the track

\item {} 
\sphinxAtStartPar
secondary tracks, generated in the given step by a physics interaction (if any), are insterted into the track stack (by calling the \sphinxcode{\sphinxupquote{{\hyperref[\detokenize{Simulation/SimulationCodeDoc:class_stepping_loop_1af0fc2b1e935c4f75824521c1976e5644}]{\sphinxcrossref{\DUrole{std,std-ref}{SteppingLoop::StackSecondaries}}}}(G4HepEmTLData\&, {\hyperref[\detokenize{Simulation/SimulationCodeDoc:class_track_stack}]{\sphinxcrossref{\DUrole{std,std-ref}{TrackStack}}}}\&, G4HepEmTrack\&)}} method)

\item {} 
\sphinxAtStartPar
information (e.g. energy deposit) might be collected at the end of each simulation step (by calling the \sphinxcode{\sphinxupquote{{\hyperref[\detokenize{Simulation/SimulationCodeDoc:class_stepping_loop_1a473d26c5090a12b239c3e2a1ac682243}]{\sphinxcrossref{\DUrole{std,std-ref}{SteppingLoop::SteppingAction}}}}({\hyperref[\detokenize{Simulation/SimulationCodeDoc:struct_results}]{\sphinxcrossref{\DUrole{std,std-ref}{Results}}}}\&, const G4HepEmTrack\&, const Box*, double, int, int, int, int)}} method )

\end{itemize}


\sphinxAtStartPar
\sphinxstylestrong{A bit more details}:

\sphinxAtStartPar
A simulation history is terminated when:\begin{itemize}
\item {} 
\sphinxAtStartPar
the particle kinetic energy becomes zero (e.g. an \(e^-\) lost all its kinetic energy along its last step)

\item {} 
\sphinxAtStartPar
the particle participated in a destructive interaction (e.g. photoelectric absorption of a \(\gamma\) photon or conversion to \(e^-/e^+\) pairs)

\item {} 
\sphinxAtStartPar
the particle leaves the calorimeter (in a normal \sphinxcode{\sphinxupquote{Geant4}} simulation the the history is terminated when the particle leaves the world)

\end{itemize}


\sphinxAtStartPar
The physics related step length constraints as well as the actions (including the secondary track production) are provided by the \sphinxcode{\sphinxupquote{G4HepEm}} implementation of the EM physics simulation.

\sphinxAtStartPar
\sphinxcode{\sphinxupquote{G4HepEm}} implements two top level methods, in its Gamma and Electron managers:\begin{itemize}
\item {} 
\sphinxAtStartPar
to provide the information on \sphinxcode{\sphinxupquote{HowFar}} a given input \(\gamma\) or \(e^-/e^+\) track goes according to their physics related constraints (e.g. till their next physics interaction takes place or due to other physics related constraints).

\item {} 
\sphinxAtStartPar
to \sphinxcode{\sphinxupquote{Perform}} all necessary physics related updates on the given input \(\gamma\) or \(e^-/e^+\) track and produce all secondary tracks in the given physics interaction (if any).

\end{itemize}


\sphinxAtStartPar
The first (\sphinxcode{\sphinxupquote{HowFar}}) is invoked at the pre\sphinxhyphen{}step point while the second (\sphinxcode{\sphinxupquote{Perform}}) is at the post\sphinxhyphen{}step point of each individual simulation step computation inside the \sphinxcode{\sphinxupquote{{\hyperref[\detokenize{Simulation/SimulationCodeDoc:class_stepping_loop_1a1d7aa9c14da7b327829dc318d13fc2ed}]{\sphinxcrossref{\DUrole{std,std-ref}{SteppingLoop::GammaStepper()}}}}}} and \sphinxcode{\sphinxupquote{{\hyperref[\detokenize{Simulation/SimulationCodeDoc:class_stepping_loop_1ad0de81b62ac3ba13532464c2c01f1b39}]{\sphinxcrossref{\DUrole{std,std-ref}{SteppingLoop::ElectronStepper()}}}}}}.

\sphinxAtStartPar
In \sphinxcode{\sphinxupquote{G4HepEm}} it’s the \sphinxcode{\sphinxupquote{G4HepEmTLData}} (thread local data) that is used in the top level, two sided communication between the consumer and \sphinxcode{\sphinxupquote{G4HepEm}}. It encapsulates the (primary and secondary) tracks and the random number generator dedicated for one particular thread. Its \sphinxstylestrong{primary} Gamma/Electron track field is used to store the actual state of the \(\gamma\) or \(e^-/e^+\) track that is under tracking. The step limit, imposed by all physics related constraints on the actual track, is calculated at each pre\sphinxhyphen{}step point by calling the bove\sphinxhyphen{}mentioned \sphinxcode{\sphinxupquote{HowFar}} top level method provided by \sphinxcode{\sphinxupquote{G4HepEm}}. Then the \sphinxcode{\sphinxupquote{Perform}} method needs to be invoked at the post\sphinxhyphen{}step point that performs all necessary physics related updates on the input \sphinxstylestrong{primary} track while produces all \sphinxstylestrong{scondary} tracks related to the given physics interaction (if any). The secondary tracks are delivered back to the caller in the appropriate \sphinxstylestrong{seondary} track fields of the \sphinxcode{\sphinxupquote{G4HepEmTLData}} object.

\sphinxAtStartPar
We might provide more details on how a \(\gamma\) and \(e^-/e^+\) simulation step is computed but you might find some information by inspecting the implementations of the top level \sphinxcode{\sphinxupquote{HowFar}} and \sphinxcode{\sphinxupquote{Perform}} \sphinxcode{\sphinxupquote{G4HepEm}} methods in the corresponding \sphinxcode{\sphinxupquote{G4HepEmGammaManager/G4HepEmElectronManager}}. 

\begin{sphinxuseclass}{breathe-sectiondef}\subsubsection*{Public Static Functions}
\index{SteppingLoop::GammaStepper (C++ function)@\spxentry{SteppingLoop::GammaStepper}\spxextra{C++ function}}

\begin{fulllineitems}
\phantomsection\label{\detokenize{Simulation/SimulationCodeDoc:_CPPv4N12SteppingLoop12GammaStepperER13G4HepEmTLDataR12G4HepEmStateR10TrackStackR8GeometryR7Resultsi}}
\pysigstartsignatures
\pysigstartmultiline
\pysiglinewithargsret{\phantomsection\label{\detokenize{Simulation/SimulationCodeDoc:class_stepping_loop_1a1d7aa9c14da7b327829dc318d13fc2ed}}\DUrole{k,k}{static}\DUrole{w,w}{  }\DUrole{kt,kt}{void}\DUrole{w,w}{  }\sphinxbfcode{\sphinxupquote{\DUrole{n,n}{GammaStepper}}}}{\DUrole{n,n,n}{G4HepEmTLData}\DUrole{w,w}{  }\DUrole{p,p}{\&}\DUrole{n,sig-param,n}{theTLData}\sphinxparamcomma \DUrole{n,n,n}{G4HepEmState}\DUrole{w,w}{  }\DUrole{p,p}{\&}\DUrole{n,sig-param,n}{theState}\sphinxparamcomma {\hyperref[\detokenize{Simulation/SimulationCodeDoc:_CPPv410TrackStack}]{\sphinxcrossref{\DUrole{n,n,n}{TrackStack}}}}\DUrole{w,w}{  }\DUrole{p,p}{\&}\DUrole{n,sig-param,n}{theTrackStack}\sphinxparamcomma {\hyperref[\detokenize{Simulation/SimulationCodeDoc:_CPPv48Geometry}]{\sphinxcrossref{\DUrole{n,n,n}{Geometry}}}}\DUrole{w,w}{  }\DUrole{p,p}{\&}\DUrole{n,sig-param,n}{theGeometry}\sphinxparamcomma {\hyperref[\detokenize{Simulation/SimulationCodeDoc:_CPPv47Results}]{\sphinxcrossref{\DUrole{n,n,n}{Results}}}}\DUrole{w,w}{  }\DUrole{p,p}{\&}\DUrole{n,sig-param,n}{theResult}\sphinxparamcomma \DUrole{kt,kt}{int}\DUrole{w,w}{  }\DUrole{n,sig-param,n}{eventID}}{}
\pysigstopmultiline
\pysigstopsignatures
\sphinxAtStartPar
Stepping loop for simulating the entire history of a \(\gamma\) track. 

\sphinxAtStartPar
The initial state of the \(\gamma\) track is provided in the \sphinxcode{\sphinxupquote{G4HepEmGammaTrack}} field of \sphinxcode{\sphinxupquote{theTLData}} input argument by the caller. The history is simulated then till the end, the state of the \(\gamma\) track is updated while secondary tracks, produced in the physics interactions, are pushed to \sphinxcode{\sphinxupquote{theTrackStack}} (if any) and the required simulation results are collected/updated into \sphinxcode{\sphinxupquote{theResult}} structure after each individual simulation step.

\sphinxAtStartPar
\begin{quote}\begin{description}
\sphinxlineitem{param theTLData}
\sphinxAtStartPar
a \sphinxcode{\sphinxupquote{G4HepEm}} specific (thread local) object primarily used to obtain all physics related information from \sphinxcode{\sphinxupquote{G4HepEm}} needed to compute a simulation step 

\sphinxlineitem{param theState}
\sphinxAtStartPar
a \sphinxcode{\sphinxupquote{G4HepEm}} specific object that stores pointers to the top level \sphinxcode{\sphinxupquote{G4HepEm}} data structure and parameters that are used by \sphinxcode{\sphinxupquote{G4HepEm}} to provide all physics related infomation needed to compute a simulation step 

\sphinxlineitem{param theTrackStack}
\sphinxAtStartPar
the track stack that is used to store the secondary tracks produced while simulating the entire history o fthe input \(\gamma\) track 

\sphinxlineitem{param theGeometry}
\sphinxAtStartPar
the geometry of the application in which the input track history is simulated 

\sphinxlineitem{param theResult}
\sphinxAtStartPar
the data structure that holds all the infomation needs to be collected during the simulation. It might be updated after each simulation step by calling the \sphinxcode{\sphinxupquote{SteppingAction}} method. 

\sphinxlineitem{param eventID}
\sphinxAtStartPar
ID of the currently simulated event, i.e. the one to which the given input \(\gamma\) track belongs to 

\end{description}\end{quote}


\end{fulllineitems}

\index{SteppingLoop::ElectronStepper (C++ function)@\spxentry{SteppingLoop::ElectronStepper}\spxextra{C++ function}}

\begin{fulllineitems}
\phantomsection\label{\detokenize{Simulation/SimulationCodeDoc:_CPPv4N12SteppingLoop15ElectronStepperER13G4HepEmTLDataR12G4HepEmStateR10TrackStackR8GeometryR7Resultsi}}
\pysigstartsignatures
\pysigstartmultiline
\pysiglinewithargsret{\phantomsection\label{\detokenize{Simulation/SimulationCodeDoc:class_stepping_loop_1ad0de81b62ac3ba13532464c2c01f1b39}}\DUrole{k,k}{static}\DUrole{w,w}{  }\DUrole{kt,kt}{void}\DUrole{w,w}{  }\sphinxbfcode{\sphinxupquote{\DUrole{n,n}{ElectronStepper}}}}{\DUrole{n,n,n}{G4HepEmTLData}\DUrole{w,w}{  }\DUrole{p,p}{\&}\DUrole{n,sig-param,n}{theTLData}\sphinxparamcomma \DUrole{n,n,n}{G4HepEmState}\DUrole{w,w}{  }\DUrole{p,p}{\&}\DUrole{n,sig-param,n}{theState}\sphinxparamcomma {\hyperref[\detokenize{Simulation/SimulationCodeDoc:_CPPv410TrackStack}]{\sphinxcrossref{\DUrole{n,n,n}{TrackStack}}}}\DUrole{w,w}{  }\DUrole{p,p}{\&}\DUrole{n,sig-param,n}{theTrackStack}\sphinxparamcomma {\hyperref[\detokenize{Simulation/SimulationCodeDoc:_CPPv48Geometry}]{\sphinxcrossref{\DUrole{n,n,n}{Geometry}}}}\DUrole{w,w}{  }\DUrole{p,p}{\&}\DUrole{n,sig-param,n}{theGeometry}\sphinxparamcomma {\hyperref[\detokenize{Simulation/SimulationCodeDoc:_CPPv47Results}]{\sphinxcrossref{\DUrole{n,n,n}{Results}}}}\DUrole{w,w}{  }\DUrole{p,p}{\&}\DUrole{n,sig-param,n}{theResult}\sphinxparamcomma \DUrole{kt,kt}{int}\DUrole{w,w}{  }\DUrole{n,sig-param,n}{eventID}}{}
\pysigstopmultiline
\pysigstopsignatures
\sphinxAtStartPar
Stepping loop for simulating the entire history of a \(e^-/e^+\) track. 

\sphinxAtStartPar
The initial state of the \(e^-/e^+\) track is provided in the \sphinxcode{\sphinxupquote{G4HepEmGammaTrack}} field of \sphinxcode{\sphinxupquote{theTLData}} input argument by the caller. The history is simulated then till the end, the state of the \(e^-/e^+\) track is updated while secondary tracks, produced in the physics interactions, are pushed to \sphinxcode{\sphinxupquote{theTrackStack}} (if any) and the required simulation results are collected/updated into \sphinxcode{\sphinxupquote{theResult}} structure after each individual simulation step.

\sphinxAtStartPar
\begin{quote}\begin{description}
\sphinxlineitem{param theTLData}
\sphinxAtStartPar
a \sphinxcode{\sphinxupquote{G4HepEm}} specific (thread local) object primarily used to obtain all physics related information from \sphinxcode{\sphinxupquote{G4HepEm}} needed to compute a simulation step 

\sphinxlineitem{param theState}
\sphinxAtStartPar
a \sphinxcode{\sphinxupquote{G4HepEm}} specific object that stores pointers to the top level \sphinxcode{\sphinxupquote{G4HepEm}} data structure and parameters that are used by \sphinxcode{\sphinxupquote{G4HepEm}} to provide all physics related infomation needed to compute a simulation step 

\sphinxlineitem{param theTrackStack}
\sphinxAtStartPar
the track stack that is used to store the secondary tracks produced while simulating the entire history o fthe input \(e^-/e^+\) track 

\sphinxlineitem{param theGeometry}
\sphinxAtStartPar
the geometry of the application in which the input track history is simulated 

\sphinxlineitem{param theResult}
\sphinxAtStartPar
the data structure that holds all the infomation needs to be collected during the simulation. It might be updated after each simulation step by calling the \sphinxcode{\sphinxupquote{SteppingAction}} method. 

\sphinxlineitem{param eventID}
\sphinxAtStartPar
ID of the currently simulated event, i.e. the one to which the given input \(e^-/e^+\) track belongs to 

\end{description}\end{quote}


\end{fulllineitems}


\end{sphinxuseclass}
\begin{sphinxuseclass}{breathe-sectiondef}\subsubsection*{Private Static Functions}
\index{SteppingLoop::StackSecondaries (C++ function)@\spxentry{SteppingLoop::StackSecondaries}\spxextra{C++ function}}

\begin{fulllineitems}
\phantomsection\label{\detokenize{Simulation/SimulationCodeDoc:_CPPv4N12SteppingLoop16StackSecondariesER13G4HepEmTLDataR10TrackStackR12G4HepEmTrack}}
\pysigstartsignatures
\pysigstartmultiline
\pysiglinewithargsret{\phantomsection\label{\detokenize{Simulation/SimulationCodeDoc:class_stepping_loop_1af0fc2b1e935c4f75824521c1976e5644}}\DUrole{k,k}{static}\DUrole{w,w}{  }\DUrole{kt,kt}{void}\DUrole{w,w}{  }\sphinxbfcode{\sphinxupquote{\DUrole{n,n}{StackSecondaries}}}}{\DUrole{n,n,n}{G4HepEmTLData}\DUrole{w,w}{  }\DUrole{p,p}{\&}\DUrole{n,sig-param,n}{theTLData}\sphinxparamcomma {\hyperref[\detokenize{Simulation/SimulationCodeDoc:_CPPv410TrackStack}]{\sphinxcrossref{\DUrole{n,n,n}{TrackStack}}}}\DUrole{w,w}{  }\DUrole{p,p}{\&}\DUrole{n,sig-param,n}{theTrackStack}\sphinxparamcomma \DUrole{n,n,n}{G4HepEmTrack}\DUrole{w,w}{  }\DUrole{p,p}{\&}\DUrole{n,sig-param,n}{thePrimary}}{}
\pysigstopmultiline
\pysigstopsignatures
\sphinxAtStartPar
Auxiliary method that pushes the secondary track(s), produced by physics interactions at the post\sphinxhyphen{}step point (if any), into the track stack. 

\sphinxAtStartPar
\begin{quote}\begin{description}
\sphinxlineitem{param theTLData}
\sphinxAtStartPar
the \sphinxcode{\sphinxupquote{G4HepEm}} specific (thread local) object that is used by \sphinxcode{\sphinxupquote{G4HepEm}} to deliver the secondary tracks to the caller after calling the its \sphinxcode{\sphinxupquote{Perform}} top level method 

\sphinxlineitem{param theTrackStack}
\sphinxAtStartPar
the track stack that is used to store the secondary tracks produced while simulating the entire history of the input track in the steppers 

\sphinxlineitem{param thePrimary}
\sphinxAtStartPar
the primary track, in its post interaction state (after calling \sphinxcode{\sphinxupquote{G4HepEm}} top level \sphinxcode{\sphinxupquote{Perform}} method), i.e. the one that underwent the physics interaction 

\end{description}\end{quote}


\end{fulllineitems}

\index{SteppingLoop::SteppingAction (C++ function)@\spxentry{SteppingLoop::SteppingAction}\spxextra{C++ function}}

\begin{fulllineitems}
\phantomsection\label{\detokenize{Simulation/SimulationCodeDoc:_CPPv4N12SteppingLoop14SteppingActionER7ResultsRK12G4HepEmTrackPK3Boxdiiii}}
\pysigstartsignatures
\pysigstartmultiline
\pysiglinewithargsret{\phantomsection\label{\detokenize{Simulation/SimulationCodeDoc:class_stepping_loop_1a473d26c5090a12b239c3e2a1ac682243}}\DUrole{k,k}{static}\DUrole{w,w}{  }\DUrole{kt,kt}{void}\DUrole{w,w}{  }\sphinxbfcode{\sphinxupquote{\DUrole{n,n}{SteppingAction}}}}{{\hyperref[\detokenize{Simulation/SimulationCodeDoc:_CPPv47Results}]{\sphinxcrossref{\DUrole{n,n,n}{Results}}}}\DUrole{w,w}{  }\DUrole{p,p}{\&}\DUrole{n,sig-param,n}{theResult}\sphinxparamcomma \DUrole{k,k}{const}\DUrole{w,w}{  }\DUrole{n,n,n}{G4HepEmTrack}\DUrole{w,w}{  }\DUrole{p,p}{\&}\DUrole{n,sig-param,n}{theTrack}\sphinxparamcomma \DUrole{k,k}{const}\DUrole{w,w}{  }{\hyperref[\detokenize{Simulation/SimulationCodeDoc:_CPPv43Box}]{\sphinxcrossref{\DUrole{n,n,n}{Box}}}}\DUrole{w,w}{  }\DUrole{p,p}{*}\DUrole{n,sig-param,n}{currentVolume}\sphinxparamcomma \DUrole{kt,kt}{double}\DUrole{w,w}{  }\DUrole{n,sig-param,n}{currentPhysStepLength}\sphinxparamcomma \DUrole{kt,kt}{int}\DUrole{w,w}{  }\DUrole{n,sig-param,n}{indxLayer}\sphinxparamcomma \DUrole{kt,kt}{int}\DUrole{w,w}{  }\DUrole{n,sig-param,n}{indxAbsorber}\sphinxparamcomma \DUrole{kt,kt}{int}\DUrole{w,w}{  }\DUrole{n,sig-param,n}{eventID}\sphinxparamcomma \DUrole{kt,kt}{int}\DUrole{w,w}{  }\DUrole{n,sig-param,n}{stepID}}{}
\pysigstopmultiline
\pysigstopsignatures
\sphinxAtStartPar
This method is called at the end of each simulation steps to collect some data during the simulation. 

\sphinxAtStartPar
This method provides the possibility of collecting some data after each simulation steps (e.g. energy deposit or length of the step). Among the \sphinxcode{\sphinxupquote{Geant4}} user actions this corresponds to the \sphinxcode{\sphinxupquote{G4UserSteppingAction}}

\sphinxAtStartPar
\begin{quote}\begin{description}
\sphinxlineitem{param theResult}
\sphinxAtStartPar
the data structure that holds all the infomation needs to be collected during the simulation (some fields might be updated) 

\sphinxlineitem{param theTrack}
\sphinxAtStartPar
the primary track, in its post interaction state, i.e. at the end of the step 

\sphinxlineitem{param currentVolume}
\sphinxAtStartPar
pointer to the volume (\sphinxcode{\sphinxupquote{absorber}}/\sphinxcode{\sphinxupquote{gap}}) in which the simulation step was done 

\sphinxlineitem{param currentPhysStepLength}
\sphinxAtStartPar
real (physical) length of the step 

\sphinxlineitem{param indxLayer}
\sphinxAtStartPar
index of the layer in which the step was done 

\sphinxlineitem{param indxAbsorber}
\sphinxAtStartPar
indicates if the step was done in the \sphinxcode{\sphinxupquote{absorber}} (0) or in the \sphinxcode{\sphinxupquote{gap}} (1) 

\sphinxlineitem{param eventID}
\sphinxAtStartPar
ID of the event to which the particle under tracking belongs to 

\sphinxlineitem{param stepID}
\sphinxAtStartPar
ID of this step that was just performed, i.e. number of steps cmpleted so far with with the current track 

\end{description}\end{quote}


\end{fulllineitems}


\end{sphinxuseclass}
\end{fulllineitems}

\index{TrackStack (C++ class)@\spxentry{TrackStack}\spxextra{C++ class}}

\begin{fulllineitems}
\phantomsection\label{\detokenize{Simulation/SimulationCodeDoc:_CPPv410TrackStack}}
\pysigstartsignatures
\pysigstartmultiline
\pysigline{\phantomsection\label{\detokenize{Simulation/SimulationCodeDoc:class_track_stack}}\DUrole{k,k}{class}\DUrole{w,w}{  }\sphinxbfcode{\sphinxupquote{\DUrole{n,n}{TrackStack}}}}
\pysigstopmultiline
\pysigstopsignatures
\sphinxAtStartPar
A simple track\sphinxhyphen{}stack to handle both primary and secondary particle tracks. 

\sphinxAtStartPar
\begin{description}
\sphinxlineitem{\sphinxstylestrong{Author}}
\sphinxAtStartPar
M. Novak 

\sphinxlineitem{\sphinxstylestrong{Date}}
\sphinxAtStartPar
July 2023

\end{description}


\sphinxAtStartPar
This stack holds tracks (all belonging to the same event), that are still to be tracking (i.e. still need to call/instert to the appropriate \sphinxcode{\sphinxupquote{{\hyperref[\detokenize{Simulation/SimulationCodeDoc:class_stepping_loop}]{\sphinxcrossref{\DUrole{std,std-ref}{SteppingLoop}}}}}}):\begin{itemize}
\item {} 
\sphinxAtStartPar
at the begining of each event, (a) primary track is inserted into the stack (inside the \sphinxcode{\sphinxupquote{{\hyperref[\detokenize{Simulation/SimulationCodeDoc:class_event_loop_1a7b1d4b512c5fa676bd990cfaa6d561c9}]{\sphinxcrossref{\DUrole{std,std-ref}{EventLoop::ProcessEvents()}}}}}}) as the very first track (NOTE: assumed to have only one primary per event for simplicity)

\item {} 
\sphinxAtStartPar
all secondaries, generated by the entire simulation of the event, are inserted when created (inside the appropriate \sphinxcode{\sphinxupquote{{\hyperref[\detokenize{Simulation/SimulationCodeDoc:class_stepping_loop}]{\sphinxcrossref{\DUrole{std,std-ref}{SteppingLoop}}}}}}), i.e. pushed and later popped for tracking

\item {} 
\sphinxAtStartPar
the event is completed when the track\sphinxhyphen{}stack becomes empty again

\end{itemize}


\sphinxAtStartPar
A new event can be started then. 

\begin{sphinxuseclass}{breathe-sectiondef}\subsubsection*{Public Functions}
\index{TrackStack::TrackStack (C++ function)@\spxentry{TrackStack::TrackStack}\spxextra{C++ function}}

\begin{fulllineitems}
\phantomsection\label{\detokenize{Simulation/SimulationCodeDoc:_CPPv4N10TrackStack10TrackStackEv}}
\pysigstartsignatures
\pysigstartmultiline
\pysiglinewithargsret{\phantomsection\label{\detokenize{Simulation/SimulationCodeDoc:class_track_stack_1a6d118a8cc1aa0f0c83797e764747cedf}}\sphinxbfcode{\sphinxupquote{\DUrole{n,n}{TrackStack}}}}{}{}
\pysigstopmultiline
\pysigstopsignatures
\sphinxAtStartPar
CTR. 

\end{fulllineitems}

\index{TrackStack::\textasciitilde{}TrackStack (C++ function)@\spxentry{TrackStack::\textasciitilde{}TrackStack}\spxextra{C++ function}}

\begin{fulllineitems}
\phantomsection\label{\detokenize{Simulation/SimulationCodeDoc:_CPPv4N10TrackStackD0Ev}}
\pysigstartsignatures
\pysigstartmultiline
\pysiglinewithargsret{\phantomsection\label{\detokenize{Simulation/SimulationCodeDoc:class_track_stack_1a9762dba449e945f1391e4272b495b4e6}}\DUrole{k,k}{inline}\DUrole{w,w}{  }\sphinxbfcode{\sphinxupquote{\DUrole{n,n}{\textasciitilde{}TrackStack}}}}{}{}
\pysigstopmultiline
\pysigstopsignatures
\sphinxAtStartPar
DTR. 

\end{fulllineitems}

\index{TrackStack::PopInto (C++ function)@\spxentry{TrackStack::PopInto}\spxextra{C++ function}}

\begin{fulllineitems}
\phantomsection\label{\detokenize{Simulation/SimulationCodeDoc:_CPPv4N10TrackStack7PopIntoER12G4HepEmTrack}}
\pysigstartsignatures
\pysigstartmultiline
\pysiglinewithargsret{\phantomsection\label{\detokenize{Simulation/SimulationCodeDoc:class_track_stack_1a528b5fd33f46021e878d4d82dbbedd74}}\DUrole{kt,kt}{int}\DUrole{w,w}{  }\sphinxbfcode{\sphinxupquote{\DUrole{n,n}{PopInto}}}}{\DUrole{n,n,n}{G4HepEmTrack}\DUrole{w,w}{  }\DUrole{p,p}{\&}\DUrole{n,sig-param,n}{track}}{}
\pysigstopmultiline
\pysigstopsignatures
\sphinxAtStartPar
Pops a secondary track from the stack and writes to the input address. 

\sphinxAtStartPar
This method is called from \sphinxcode{\sphinxupquote{{\hyperref[\detokenize{Simulation/SimulationCodeDoc:class_event_loop_1a7b1d4b512c5fa676bd990cfaa6d561c9}]{\sphinxcrossref{\DUrole{std,std-ref}{EventLoop::ProcessEvents()}}}}}} before start tracking a new track. It returns with the original index of the popped track or \sphinxhyphen{}1 when the track is actually empty, i.e. no more track to pop.

\sphinxAtStartPar
\begin{quote}\begin{description}
\sphinxlineitem{param track}
\sphinxAtStartPar
\sphinxstylestrong{{[}inout{]}} the address of the \sphinxcode{\sphinxupquote{G4HepEmTrack}} where the next track should be popped, i.e. copied. 

\end{description}\end{quote}
\begin{quote}\begin{description}
\sphinxlineitem{return}
\sphinxAtStartPar
returns with the original index of the popped track or \sphinxhyphen{}1 if the there are no more tracks in the track 

\end{description}\end{quote}


\end{fulllineitems}

\index{TrackStack::GetTypeOfNextTrack (C++ function)@\spxentry{TrackStack::GetTypeOfNextTrack}\spxextra{C++ function}}

\begin{fulllineitems}
\phantomsection\label{\detokenize{Simulation/SimulationCodeDoc:_CPPv4N10TrackStack18GetTypeOfNextTrackEv}}
\pysigstartsignatures
\pysigstartmultiline
\pysiglinewithargsret{\phantomsection\label{\detokenize{Simulation/SimulationCodeDoc:class_track_stack_1a71cd14abad3d3d0693658e5d4a1315eb}}\DUrole{kt,kt}{int}\DUrole{w,w}{  }\sphinxbfcode{\sphinxupquote{\DUrole{n,n}{GetTypeOfNextTrack}}}}{}{}
\pysigstopmultiline
\pysigstopsignatures
\sphinxAtStartPar
Can provide the type of the next track. 

\sphinxAtStartPar
Returns with an integer that encodes the type of the next track in the stack, i.e. the type of the track that will be popped when calling \sphinxcode{\sphinxupquote{{\hyperref[\detokenize{Simulation/SimulationCodeDoc:class_track_stack_1a528b5fd33f46021e878d4d82dbbedd74}]{\sphinxcrossref{\DUrole{std,std-ref}{PopInto()}}}}}} next time.

\sphinxAtStartPar
\begin{quote}\begin{description}
\sphinxlineitem{return}
\sphinxAtStartPar
an integer indicating the type of the next track:\begin{itemize}
\item {} 
\sphinxAtStartPar
\sphinxhyphen{}1 in case of \(e^-\)

\item {} 
\sphinxAtStartPar
0 in case of \(\gamma\)

\item {} 
\sphinxAtStartPar
+1 in case of \(e^+\)

\item {} 
\sphinxAtStartPar
\sphinxhyphen{}999 if the stack is empty 

\end{itemize}


\end{description}\end{quote}


\end{fulllineitems}

\index{TrackStack::Insert (C++ function)@\spxentry{TrackStack::Insert}\spxextra{C++ function}}

\begin{fulllineitems}
\phantomsection\label{\detokenize{Simulation/SimulationCodeDoc:_CPPv4N10TrackStack6InsertEv}}
\pysigstartsignatures
\pysigstartmultiline
\pysiglinewithargsret{\phantomsection\label{\detokenize{Simulation/SimulationCodeDoc:class_track_stack_1a0f33c1b9f159ba9d6e0953c67cf1e4e2}}\DUrole{n,n,n}{G4HepEmTrack}\DUrole{w,w}{  }\DUrole{p,p}{\&}\sphinxbfcode{\sphinxupquote{\DUrole{n,n}{Insert}}}}{}{}
\pysigstopmultiline
\pysigstopsignatures
\sphinxAtStartPar
Returns a reference to a secondary track that can be used to push a new track into the stack. 

\sphinxAtStartPar
This method is called whenever a new track needs to be inseted into the stack. The provided reference can be used to fill the track infomation (the referenced track is re\sphinxhyphen{}set).

\sphinxAtStartPar
\begin{quote}\begin{description}
\sphinxlineitem{return}
\sphinxAtStartPar
A reference to a \sphinxcode{\sphinxupquote{G4HepEmTrack}} that can be used to add a new track to the stack (by filling in its filed). 

\end{description}\end{quote}


\end{fulllineitems}

\index{TrackStack::Copy (C++ function)@\spxentry{TrackStack::Copy}\spxextra{C++ function}}

\begin{fulllineitems}
\phantomsection\label{\detokenize{Simulation/SimulationCodeDoc:_CPPv4N10TrackStack4CopyER12G4HepEmTrackR12G4HepEmTrack}}
\pysigstartsignatures
\pysigstartmultiline
\pysiglinewithargsret{\phantomsection\label{\detokenize{Simulation/SimulationCodeDoc:class_track_stack_1ae87bf7ff4f6836e10cf3b7e277b0d653}}\DUrole{kt,kt}{void}\DUrole{w,w}{  }\sphinxbfcode{\sphinxupquote{\DUrole{n,n}{Copy}}}}{\DUrole{n,n,n}{G4HepEmTrack}\DUrole{w,w}{  }\DUrole{p,p}{\&}\DUrole{n,sig-param,n}{from}\sphinxparamcomma \DUrole{n,n,n}{G4HepEmTrack}\DUrole{w,w}{  }\DUrole{p,p}{\&}\DUrole{n,sig-param,n}{to}}{}
\pysigstopmultiline
\pysigstopsignatures
\sphinxAtStartPar
Copying the content of the \sphinxcode{\sphinxupquote{from}} to the \sphinxcode{\sphinxupquote{to}} track. 

\end{fulllineitems}

\index{TrackStack::GetNextTrackID (C++ function)@\spxentry{TrackStack::GetNextTrackID}\spxextra{C++ function}}

\begin{fulllineitems}
\phantomsection\label{\detokenize{Simulation/SimulationCodeDoc:_CPPv4N10TrackStack14GetNextTrackIDEv}}
\pysigstartsignatures
\pysigstartmultiline
\pysiglinewithargsret{\phantomsection\label{\detokenize{Simulation/SimulationCodeDoc:class_track_stack_1a4e4f64496afddaacd19dcf99c1008402}}\DUrole{k,k}{inline}\DUrole{w,w}{  }\DUrole{kt,kt}{int}\DUrole{w,w}{  }\sphinxbfcode{\sphinxupquote{\DUrole{n,n}{GetNextTrackID}}}}{}{}
\pysigstopmultiline
\pysigstopsignatures
\sphinxAtStartPar
Returns with the next track ID (track ID is incremented whenever this method is invoked). 

\end{fulllineitems}

\index{TrackStack::ReSetTrackID (C++ function)@\spxentry{TrackStack::ReSetTrackID}\spxextra{C++ function}}

\begin{fulllineitems}
\phantomsection\label{\detokenize{Simulation/SimulationCodeDoc:_CPPv4N10TrackStack12ReSetTrackIDEv}}
\pysigstartsignatures
\pysigstartmultiline
\pysiglinewithargsret{\phantomsection\label{\detokenize{Simulation/SimulationCodeDoc:class_track_stack_1a1fd724f7c0c8b125304786c6f4b0e00c}}\DUrole{k,k}{inline}\DUrole{w,w}{  }\DUrole{kt,kt}{void}\DUrole{w,w}{  }\sphinxbfcode{\sphinxupquote{\DUrole{n,n}{ReSetTrackID}}}}{}{}
\pysigstopmultiline
\pysigstopsignatures
\sphinxAtStartPar
Resets the track ID to zero. 

\end{fulllineitems}


\end{sphinxuseclass}
\begin{sphinxuseclass}{breathe-sectiondef}\subsubsection*{Private Members}
\index{TrackStack::fSize (C++ member)@\spxentry{TrackStack::fSize}\spxextra{C++ member}}

\begin{fulllineitems}
\phantomsection\label{\detokenize{Simulation/SimulationCodeDoc:_CPPv4N10TrackStack5fSizeE}}
\pysigstartsignatures
\pysigstartmultiline
\pysigline{\phantomsection\label{\detokenize{Simulation/SimulationCodeDoc:class_track_stack_1a7c3631cc17083505d626d0ff669c16cf}}\DUrole{kt,kt}{int}\DUrole{w,w}{  }\sphinxbfcode{\sphinxupquote{\DUrole{n,n}{fSize}}}}
\pysigstopmultiline
\pysigstopsignatures
\sphinxAtStartPar
current capacity of the track stack 

\end{fulllineitems}

\index{TrackStack::fCurIndx (C++ member)@\spxentry{TrackStack::fCurIndx}\spxextra{C++ member}}

\begin{fulllineitems}
\phantomsection\label{\detokenize{Simulation/SimulationCodeDoc:_CPPv4N10TrackStack8fCurIndxE}}
\pysigstartsignatures
\pysigstartmultiline
\pysigline{\phantomsection\label{\detokenize{Simulation/SimulationCodeDoc:class_track_stack_1a3917da2d4c2a007a69e1ba9e02af032d}}\DUrole{kt,kt}{int}\DUrole{w,w}{  }\sphinxbfcode{\sphinxupquote{\DUrole{n,n}{fCurIndx}}}}
\pysigstopmultiline
\pysigstopsignatures
\sphinxAtStartPar
number of tracks used from the capacity 

\end{fulllineitems}

\index{TrackStack::fCurrentTrackID (C++ member)@\spxentry{TrackStack::fCurrentTrackID}\spxextra{C++ member}}

\begin{fulllineitems}
\phantomsection\label{\detokenize{Simulation/SimulationCodeDoc:_CPPv4N10TrackStack15fCurrentTrackIDE}}
\pysigstartsignatures
\pysigstartmultiline
\pysigline{\phantomsection\label{\detokenize{Simulation/SimulationCodeDoc:class_track_stack_1a5c45f50a86755ec9e93fb524d2684f89}}\DUrole{kt,kt}{int}\DUrole{w,w}{  }\sphinxbfcode{\sphinxupquote{\DUrole{n,n}{fCurrentTrackID}}}}
\pysigstopmultiline
\pysigstopsignatures
\sphinxAtStartPar
current track ID 

\end{fulllineitems}

\index{TrackStack::fTrackVect (C++ member)@\spxentry{TrackStack::fTrackVect}\spxextra{C++ member}}

\begin{fulllineitems}
\phantomsection\label{\detokenize{Simulation/SimulationCodeDoc:_CPPv4N10TrackStack10fTrackVectE}}
\pysigstartsignatures
\pysigstartmultiline
\pysigline{\phantomsection\label{\detokenize{Simulation/SimulationCodeDoc:class_track_stack_1a3d2e0b43b1ded801b0c98300ac8766d1}}\DUrole{n,n,n}{std}\DUrole{p,p}{::}\DUrole{n,n,n}{vector}\DUrole{p,p}{\textless{}}\DUrole{n,n,n}{G4HepEmTrack}\DUrole{p,p}{\textgreater{}}\DUrole{w,w}{  }\sphinxbfcode{\sphinxupquote{\DUrole{n,n}{fTrackVect}}}}
\pysigstopmultiline
\pysigstopsignatures
\sphinxAtStartPar
the stack as a vector of tracks 

\end{fulllineitems}


\end{sphinxuseclass}
\end{fulllineitems}


\sphinxstepscope

\cleardoublepage
\begingroup
\renewcommand\chapter[1]{\endgroup}
\phantomsection


\chapter{Bibliography}
\label{\detokenize{zzbib:bibliography}}\label{\detokenize{zzbib::doc}}
\sphinxAtStartPar



\chapter{Indices and tables}
\label{\detokenize{index:indices-and-tables}}\begin{itemize}
\item {} 
\sphinxAtStartPar
\DUrole{xref,std,std-ref}{genindex}

\item {} 
\sphinxAtStartPar
\DUrole{xref,std,std-ref}{modindex}

\item {} 
\sphinxAtStartPar
\DUrole{xref,std,std-ref}{search}

\end{itemize}

\begin{sphinxthebibliography}{1}
\bibitem[1]{zzbib:agostinelli2003geant4}
\sphinxAtStartPar
Sea Agostinelli, John Allison, K al Amako, John Apostolakis, H Araujo, Pedro Arce, Makoto Asai, D Axen, Swagato Banerjee, GJNI Barrand, and others. Geant4—a simulation toolkit. \sphinxstyleemphasis{Nuclear instruments and methods in physics research section A: Accelerators, Spectrometers, Detectors and Associated Equipment}, 506(3):250\textendash{}303, 2003.
\bibitem[2]{zzbib:allison2006geant4}
\sphinxAtStartPar
John Allison, Katsuya Amako, JEA Apostolakis, HAAH Araujo, P Arce Dubois, MAAM Asai, GABG Barrand, RACR Capra, SACS Chauvie, RACR Chytracek, and others. Geant4 developments and applications. \sphinxstyleemphasis{IEEE Transactions on nuclear science}, 53(1):270\textendash{}278, 2006.
\bibitem[3]{zzbib:allison2016recent}
\sphinxAtStartPar
John Allison, Katsuya Amako, John Apostolakis, Pedro Arce, Makoto Asai, Tsukasa Aso, Enrico Bagli, A Bagulya, S Banerjee, GJNI Barrand, and others. Recent developments in geant4. \sphinxstyleemphasis{Nuclear instruments and methods in physics research section A: Accelerators, Spectrometers, Detectors and Associated Equipment}, 835:186\textendash{}225, 2016.
\bibitem[4]{zzbib:g4hepem}
\sphinxAtStartPar
Mihaly Novak, Jonas Hahnfeld, and Ben Morgan. mnovak42/g4hepem: The G4HepEm R\&D Project. 2022. accessed on 7 September 2023, https://github.com/mnovak42/g4hepem https://g4hepem.readthedocs.io/en/latest/.
\end{sphinxthebibliography}



\renewcommand{\indexname}{Index}
\printindex
\end{document}